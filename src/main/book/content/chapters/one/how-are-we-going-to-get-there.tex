\section{How are we going to get there}

\subsection{Leading by example}

The principal technique throughout this book is leading by
example. What this means in this case is that the ideas are presented
primarily in terms of a coherent collection of examples that work
together to do something. Namely, these examples function together to
provide a prototypical web-based application with a feature set that
resonates with what application developers are building today and
contemplating building tomorrow.

Let's illustrate this in more detail by telling a story. We imagine a
cloud-based editor for a simple programming language, not unlike
\texttt{Mozilla's bespin} . A user can register with the service and
then create an application project which allows them
\begin{itemize}
   \item to write code in a structured editor that understands the language;
   \item manage files in the application project;
   \item compile the application;
   \item run the application
\end{itemize}

\subsubsection{Our toy language}

For our example we'll need a toy language. Fittingly for a book about
\texttt{Scala} we'll use the $\lambda$-calculus as our toy
language. The core \textit{abstract} syntax of the lambda calculus is
given by the following \textit{EBNF} grammar.

\begin{mathpar}
  \inferrule* [lab=mention] {} {{M,N} ::= x}
  \and
  \inferrule* [lab=abstraction] {} {\;| \; \lambda x . M}
  \and \\
  \inferrule* [lab=application] {} {\;| \; M N}
\end{mathpar} 

To this core let us add some syntactic sugar.

\begin{mathpar}
  \inferrule* [lab=previous] {} {{M,N} ::= ...}
  \and \\
  \inferrule* [lab=let] {} {\;| \; let x = M}
  \and \\
  \inferrule* [lab=seq] {} {\;| \; M;N}
\end{mathpar} 

Now let's wrap this up in concrete syntax.

\begin{mathpar}
  \inferrule* [lab=mention] {} {{M,N} ::= x}
  \and 
  \inferrule* [lab=abstraction] {} {\;| \; \texttt{(} x_1 \texttt{,} ... \texttt{,} x_k \texttt{)} \texttt{=>} M}
  \and \\
  \inferrule* [lab=application] {} {\;| \; M\texttt{(} N_1 \texttt{,} ... \texttt{,} N_k \texttt{)}}
  \and \\
  \inferrule* [lab=let] {} {\;| \; \texttt{val} x \texttt{=} M}
  \and \\
  \inferrule* [lab=seq] {} {\;| \; M \texttt{;} N }
  \and \\
  \inferrule* [lab=group] {} {\;| \; \texttt{{} M \texttt{}} }
\end{mathpar} 