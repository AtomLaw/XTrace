%ch.tex


\chapter{Motivation and Background}
\begin{center}
{\small\em Where are we; how did we get here; and where are we going?}
\end{center}

\paragraph{If on a winter's night a programmer (with apologies to Italo Calvino)} 
You've just picked up the new book by Greg Meredith, Pro
Scala. Perhaps you've heard about it on one of the mailing lists or
seen it advertised on the \texttt{Scala} site or at Amazon. You're
wondering if it's for you. Maybe you've been programming in functional
languages or even \texttt{Scala} for as long as you can remember. Or
maybe you've been a professional programmer for quite some time. Or
maybe you're a manager of programmers, now and you're trying to stay
abreast of the latest technology. Or, maybe you're a futurologist who
looks at technology trends to get a sense of where things are
heading. Whoever you are, if you're like most people, this book is
going to make a lot more sense to you if you've already got about five
to ten thousand hours of either \texttt{Scala} or some other
functional language programming under your belt \footnote{ Now, i've
been told that this is too much to expect of a would-be reader; but,
when i whip out my calculator, i see that (5000 hrs / 25 hrs/wk ) / 52
wks/yr) = 3.84615384615 years. That means that if you've put in under
four years at a hobbyist level, you've met this
requirement. Alternatively, if you've put in less than two years as a
professional working solely in functional languages, you've met the
requirement. Honestly, we don't have to give in to inflationary trends
in the meanings of terms. If we say something is aimed at a pro, we
could mean what we say. }. There may be nuggets in here that provide
some useful insights for people with a different kind of experience;
and, of course, there are those who just take to the ideas without
needing to put in the same number of hours; but, for most, that's
probably the simplest gauge of whether this book is going to make
sense to you at first reading.

On the other hand, just because you've got that sort of experience
under your belt still doesn't mean this book is for you. Maybe you're
just looking for a few tips and tricks to make \texttt{Scala} do what
you want for the program you're writing right now. Or maybe you've got
a nasty perf issue you want to address and are looking here for a
resolution. If that's the case, then maybe this book isn't for you
because this book is really about a point of view, a way of looking at
programming and computation. In some sense this book is all about
programming and complexity management because that's really the issue
that the professional programmer is up against, today. On average the
modern programmer building an Internet-based application is dealing
with no less than a dozen technologies. They are attempting to build
applications with nearly continuous operation, 24x7 availability
servicing 100's to 1000's of concurrent requests. They are overwhelmed
by complexity. What the professional programmer really needs are tools
for complexity management. The principle aim of this book is to serve
that need in that community.

The design patterns expressed in this book have been developed for
nearly \emph{fifty} years to address exactly those concerns. Since
\texttt{Scala} isn't nearly fifty years old you can guess that they
have origins in older technologies, but \texttt{Scala}, it turns out,
is an ideal framework in which both to realize them and to talk about
their ins and outs and pros and cons. However, since they don't
originate in \texttt{Scala}, you can also guess that they have some
significant applicability to the other eleven technologies the modern
professional programmer is juggling.


\section{Where are we}

\subsection{The concurrency squeeze: from the hardware up, from the web down}

It used to be fashionable in academic papers or think tank reports to
predict and then bemoan the imminent demise of Moore's law, to wax on
about the need to ``go sideways'' in hardware design from the number
of cores per die to the number of processors per box. Those days of
polite conversation about the on-coming storm are definitely in our
rear view mirror. Today's developer knows that if her program is
commercially interesting at all then it needs to be web-accessible on
a 24x7 basis; and if it's going to be commercially significant it will
need to support at least 100's if not thousands of concurrent accesses
to its features and functions. Her application is most likely hosted
by some commercial outfit, a Joyent or an EngineYard or an Amazon EC3
or $\ldots$ who are deploying her code over multiple servers each of
which is in turn multi-processor with multiple cores. This means that
from the hardware up and from the web down today's intrepid developer
is dealing with parallelism, concurrency and distribution.

Unfortunately, the methods available in in mainstream programming
languages of dealing with these different aspects of simultaneous
execution are not up to the task of supporting development at this
scale. The core issue is complexity. The modern application developer
is faced with a huge range of concurrency and concurrency control
models, from transactions in the database to message-passing between
server components. Whether to partition her data is no longer an
option, she's thinking hard about \emph{how} to partition her data and
whether or not this ``eventual consistency'' thing is going to
liberate her or bring on a new host of programming nightmares. By
comparison threads packages seem like quaint relics from a time when
concurrent programming was a little hobby project she did after
hours. The modern programmer needs to simplify her life in order to
maintain a competitive level of productivity.

Functional programming provides a sort of transition technology. On
the one hand, it's not that much of a radical departure from
mainstream programming like Java. On the other it offers simple,
uniform model that introduces a number of key features that
considerably improve productivity and maintainability. Java brought
the C/C++ programmer several steps closer to a functional paradigm,
introducing garbage collection, type abstractions such as generics and
other niceties. Languages like \texttt{OCaml}, \texttt{F\#} and
\texttt{Scala} go a step further, bringing the modern developer into
contact with higher order functions, the relationship between types
and pattern matching and powerful abstractions like monads. Yet,
functional programming does not embrace concurrency and distribution
in its foundations. It is not based on a model of computation, like
the actor model or the process calculi, in which the notion of
execution that is fundamentally concurrent. That said, it meshes
nicely with a variety of concurrency programming models. In
particular, the combination of higher order functions (with the
ability to pass functions as arguments and return functions as values)
together with the structuring techniques of monads make models such as
software transactional memory or data flow parallelism quite easy to
integrate, while pattern-matching additionally makes message-passing
style easier to incorporate.

\subsection{Ubiquity of robust, high-performance virtual machines}

Another reality of the modern programmer's life is the ubiquity of
robust, high-performance virtual machines. Both the \texttt{Java}
Virtual Machine (\texttt{JVM}) and the Common Language Runtime
(\texttt{CLR}) provide manage code execution environments that are not
just competitive with their unmanaged counterparts (such as \texttt{C}
and \texttt{C++}), but actually the dominant choice for many
applications. This has two effects that are playing themselves out in
terms of industry trends. Firstly, it provides some level of
insulation between changes in hardware design (from single core per
die to multi-core, for example) that impacts execution model and
language level interface. To illustrate the point note that these
changes in hardware have impacted memory models. This has a much
greater impact on the \texttt{C/C++} family of languages than on
\texttt{Java} because the latter is built on an abstract machine that
hides the underlying memory model. One may, in fact, contemplate an
ironic future in which this abstraction alone causes managed code to
outperform \texttt{C/C++} code because of \texttt{C/C++}'s faulty
assumptions about best use of memory. Secondly, it completely changes
the landscape for language development. By providing a much higher
level and more uniform target for language execution semantics it
lowers the barrier to entry for contending language designs. It is not
surprising, therefore, that we have seen an explosion in language
proposals in the last several years, including \texttt{Clojure},
\texttt{Fortress}, \texttt{Scala}, \texttt{F\#} and many others. It
should not escape notice that all of the languages in that list and
the majority of the proposals coming out are either functional
languages, object-functional languages, or heavily influenced by
functional language design concepts.

\subsection{Advances in functional programming, monads and the awkward squad}

One of the reasons that language design proposals have been so heavily
influenced by functional language design principles is that functional
language design has made demonstrable progress. Since the '80's when
\texttt{Lisp} and it's progeny were thrown out of the industry for
performance failures a lot of excellent work has gone on that has
rectified many of the problems those languages faced. In particular,
while \texttt{Lisp} implementations tried to take a practical approach
to certain aspects of computation, chiefly having to do with
side-effecting operations and i/o, the underlying semantic model did
not seem well-suited to address those kinds of computations. And yet,
not only are side-effecting computations and especially i/o
ubiquitous, using them led (at least initially) to considerably better
performance. Avoiding those operations (sometimes called functional
purity) seemed to be an academic exercise not well suited to writing
``real world'' applications.

However, while many industry shops were throwing out functional
languages, except for niche applications, work was going on that would
reverse this trend. One of the key developments in this was an early
bifurcation of functional language designs at a fairly fundamental
level. The \texttt{Lisp} family of languages are untyped and
dynamic. In the modern world the lack of typing might seem egregiously
unmaintainable, but by comparison to \texttt{C} it was more than made
up for by the kind of dynamic meta-programming that these languages
made possible. Programmers enjoyed a certain kind of productivity
because they could ``go meta'' -- writing programs to write programs
(even dynamically modify them on the fly) -- in a uniform manner. This
sort of feature has become mainstream, as found in \texttt{Ruby} or
even \texttt{Java}'s reflection API, precisely because it is so
extremely useful. Unfortunately, the productivity gains of
meta-programming were not enough to offset the performance shortfalls
at the time.

There was, however, a statically typed branch of functional
programming that began to have traction in certain academic circles
with the development of the \texttt{ML} family of languages -- which
today includes \texttt{OCaml}, which can be considered the direct
ancestor of \texttt{Scala}. One of the very first developments in that
line of investigation was the recognition that data description came
in not just one but \emph{two} flavors: types and \emph{patterns}. The
two flavors, it was recognized, are dual. Types tell the program how
data was built up from its components while patterns tell a program
how to take data apart in terms of its components. These are the
origins of elements in \texttt{Scala}'s design like
\lstinline[language=Scala]!case class!es and the
\lstinline[language=Scala]!match! construct. The \texttt{ML} family of
languages also gave us the first workable parametric polymorphism
(that's generic types to you). 

Still these languages suffered when it came to a compelling and
uniform treatment of side-effecting computations. That all changed
with Haskell. In the mid-80's a young researcher by the name of
Eugenio Moggi observed that an idea previously discovered in a then
obscure branch of mathematics (called category theory) offered a way
to \emph{structure} functional programs to allow them to deal with
side-effecting computations in uniform and compelling
manner. Essentially, the notion of a monad (as it was called in the
category theory literature) provided a language level abstraction for
structuring side-effecting computations in a functional setting. In
today's parlance, he found a domain specific language, a DSL, for
organizing side-effecting computations in an ambient (or hosting)
functional language. Once Moggi made this discovery another
researcher, Phil Wadler, realized that this DSL had a couple of
different ``presentations'' that were almost immediately
understandable by the average programmer. One presentation, called
comprehensions (after it's counter part in set theory), could be
understood directly in terms of a very familiar construct
\lstinline[language=SQL]!SELECT ... FROM ... WHERE ...!; while the
other, dubbed \lstinline[language=Haskell]!do!-notation by the
\texttt{Haskell} community, provided operations that behaved
remarkably like sequencing and assignment. \texttt{Haskell} offers
syntactic sugar to support the latter while the former has been
adopted in both \texttt{XQuery}'s
\lstinline[language=XML]!FLWOR!-expressions and Microsoft's
\texttt{LINQ}

\section{Where are we going}

With a preamble like that it doesn't take much to guess where all this
is heading. More and more we are looking at trends that lead toward
more functional and functionally-based web applications. We need not
look to the growing popularity of cutting-edge frameworks like
\texttt{Lift} to see this trend. Both Javascript (with it's origins in
Self) and Rails must be counted amongst the functionally influenced.

\subsection{A functional web}

Because there are already plenty of excellent functional web
frameworks in the open source community our aim is not to build
another. Rather our aim is to supply a set of design patterns that
will work with most -- in fact are already implicitly at work in many
-- but that when used correctly will reduce complexity.

Specifically, we will look at the organization of the pipeline of a
web-application from the pipeline of HTTP requests through the
application logic to the store and back. We will see how in each case
judicious use of the monadic design pattern provides for significant
leverage in structuring code, making it both simpler, more
maintainable and more robust in the face of change.

To that end we will be looking at

\begin{itemize}
  \item processing HTTP-streams using presence of \emph{delimited}
  continuations to allow for a sophisticated state management
  \item parser combinators for parsing HTTP-requests and higher-level
    application protocols using HTTP as a transport
  \item application domain model as an abstract syntax
  \item zippers as a means of automatically generating navigation
  \item collections and containers in memory
  \item storage, including a new way to approach query and search
\end{itemize}

In each case there is an underlying organization to the computation
that solves the problem. In each case we find an instance of the
monadic design pattern. Whether this apparent universal applicability
is an instance of finding a hammer that turns everything it encounters
into nails or that structuring computation in terms of monads has a
genuine depth remains to be seen. What can be said even at this early
stage of the game is that object-oriented design patterns were
certainly proposed for each of these situations and many others. It
was commonly held that such techniques were not merely universally
applicable, but of genuine utility in every domain of application. The
failure of object-oriented design methods to make good on these claims
might be an argument for caution. Sober assessment of the situation,
however, gives cause for hope.

Unlike the notion monad, objects began as ``folk'' tradition. It was
many years into proposals for object-oriented design methods before
there were commonly accepted formal or mathematical accounts. By
contrast monads began as a mathematical entity. Sequestered away in
category theory the idea was one of a whole zoology of generalizations
of common mathematical entities. It took some time to understand that
both set comprehensions and algebraic data types were instances monads
and that the former was a universal language for the notion. It took
even more time to see the application to structuring
computations. Progress was slow and steady and built from a solid
foundation. This gave the notion an unprecedented level of quality
assurance testing. The category theoretic definition is nearly fifty
years old. If we include the investigation of set comprehensions as a
part of the QA process we add another one hundred years. If we include
the forty years of vigorous use of relational databases and the
\lstinline[language=SQL]!SELECT-FROM-WHERE! construct in the industry,
we see that this was hardly just an academic exercise.

Perhaps more importantly than any of those is the fact that while
object-oriented techniques ultimately failed to be compositional in
any useful way -- inheritance, in fact being at positively at odds
with concurrent composition -- the notion of monad is actually an
attempt to capture the meaning of composition. As we will see in the
upcoming sections, it defines an powerful notion of parametric
composition. This is crucial because in the real world
\emph{composition is the primary means to scaling} -- both in the
sense of performance and in the sense of complexity. As pragmatic
engineers we manage complexity of scale by building larger systems out
of smaller ones. As pragmatic engineers we understand that each time
components are required to interface or synchronize we have the
potential for introducing performance concerns. The parametric form of
composition encapsulated in the notion of monad gives us a language
for talking about both kinds of scaling and connecting the two
ideas. It provides a language for talking about the interplay between
the composition of structure and the composition of the flow of
control. It encapsulates stateful computation. It encapsulates data
structure. In this sense the notion of monad is poised to be the
rational reconstruction of the notion of object. Telling this story
was my motivation for writing this book.

\subsection{DSL-based design}

It has become buzz-word du jour to talk about DSL-based design. So
much so that it's becoming hard to understand what the term means. In
the functional setting the meaning is really quite clear and since the
writing of the Structure and Interpretation of Computer Programs (one
of the seminal texts of functional programming and one of the first to
pioneer the idea of DSL-based design) the meaning has gotten
considerably clearer. In a typed functional setting the design of a
collection of types tailor-made to model and address the operations of
some domain is the basis is effectively the design of an abstract
syntax of a language for computing over the domain.

To see why this must be so, let's begin from the basics. Informally,
DSL-based design means we express our design in terms of a little
mini-language, tailor-made for our application domain. When push
comes to shove, though, if we want to know what DSL-based design means
in practical terms, eventually we have to ask what goes into the
specification of a language. The commonly received wisdom is that a
language is comprised of a \emph{syntax} and a \emph{semantics}. The
syntax carries the structure of the expressions of the language while
the semantics says how to evaluate those expressions to achieve a
result -- typically either to derive a meaning for the expression
(such as this expression denotes that value) or perform an action or
computation indicated by the expression (such as print this string on
the console). Focusing, for the moment, on syntax as the more concrete
of the two elements, we note that syntax is governed by
\emph{grammar}. Whether we're building a concrete syntax, like the
\texttt{ASCII} strings one types to communicate \texttt{Scala}
expressions to the compiler or building an abstract syntax, like the
expression trees of \texttt{LINQ}, syntax is governed by grammar.

What we really want to call out in this discussion is that a
collection of types forming a model of some domain is actually a
grammar for an abstract syntax. This is most readily seen by comparing
the core of the type definition language of modern functional
languages with something like \texttt{EBNF}, the most prevalent
language for defining context-free grammars. At their heart the two
structures are nearly the same. When one is defining a grammar one is
defining a collection of types that model some domain and vice
versa. This is blindingly obvious in \texttt{Haskell}, and is the
essence of techniques like the application of two-level type
decomposition to model grammars. Moreover, while a little harder to
see in \texttt{Scala} it is still there. It is in this sense that
typed functional languages like \texttt{Scala} are very well suited
for DSL-based design. To the extent that the use of \texttt{Scala}
relies on the functional core of the language (not the object-oriented
bits) virtually every domain model is already a kind of DSL in that
it's types define a kind of abstract syntax.

Taking this idea a step further, in most cases such collections of
types are actually representable as a monad. Monads effectively
encapsulate the notion of an algebra -- which in this context is a
category theorist's way of saying a certain kind of collection of
types. If you are at all familiar with parser combinators and perhaps
have heard that these too are facilitated with monadic composition
then the suggestion that there is a deeper link between parsing,
grammars, types and monads might make some sense. On the other hand,
if this seems a little too abstract it will be made much more concrete
in the following sections. For now, we are simply planting the seed of
the idea that monads are not just for structuring side-effecting
computations.

\section{How are we going to get there}

\subsection{Leading by example}

The principal technique throughout this book is leading by
example. What this means in this case is that the ideas are presented
primarily in terms of a coherent collection of examples, rendered as
\texttt{Scala} code, that work together to do something. Namely, these
examples function together to provide a prototypical web-based
application with a feature set that resonates with what application
developers are building today and contemplating building tomorrow.

Let's illustrate this in more detail by telling a story. We imagine a
cloud-based editor for a simple programming language, not unlike
\texttt{Mozilla}'s \texttt{bespin} . A user can register with the
service and then create an application project which allows them
\begin{itemize}
   \item to write code in a structured editor that understands the language;
   \item manage files in the application project;
   \item compile the application;
   \item run the application
\end{itemize}

% Thus, at broad strokes requests from the client app to the server break down into the following categories
% \begin{itemize}
%   \item edits to code
%   \item edits to project structure
%   \item compilation and execution requests
% \end{itemize}

These core capabilities wrap around our little toy programming
language in much the same way a modern IDE might wrap around
development in a more robust, full-featured language. Hence, we want
the capabilities of the application to be partially driven from the
specification of our toy language. For example, if we support some
syntax-highlighting, or syntax-validation on the client, we want that
to be driven from that language spec to the extent that changes to the
language spec ought to result in changes to the behavior of the
highlighting and validation. Thus, at the center of our application is
the specification of our toy language.

\subsubsection{Our toy language}

\paragraph{Abstract syntax}
%For our example we'll need a toy language.
Fittingly for a book about \texttt{Scala} we'll use the
$\lambda$-calculus as our toy language. \footnote{A word to the wise:
  even if you are an old hand at programming language semantics, even
  if you know the $\lambda$-calculus like the back of your hand, you
  are likely to be surprised by some of the things you see in the next
  few sections. Just to make sure that everyone gets a chance to look
  at the formalism as if it were brand new, a few recent theoretical
  developments have been thrown in. So, watch out!} The core
\textit{abstract} syntax of the lambda calculus is given by the
following \textit{EBNF} grammar.

\begin{mathpar}
  \inferrule* [lab=expression] {} {{M,N} ::=}
  \and
  \inferrule* [lab=mention] {} {x}
  \and
  \inferrule* [lab=abstraction] {} {\;| \; \lambda x . M}
  \and
  \inferrule* [lab=application] {} {\;| \; M N}
\end{mathpar} 

Informally, this is really a language of pure variable management. For
example, if the expression $M$ mentions $x$, then $\lambda x. M$ turns
$x$ into a variable in $M$ and provides a means to substitute values
into $M$, via application. Thus, $(\lambda x.M)N$ will result in a new
term, sometimes written $M[N/x]$, in which every occurrence of $x$ has
been replaced by an occurrence of $N$. Thus, $(\lambda x.x)M$ yields
$M$, illustrating the implementation in the $\lambda$-calculus of the
identity function. It turns out to be quite remarkable what you can do
with pure variable management.

\paragraph{A simple-minded representation}
At a syntactic level this has a direct representation as the following
\texttt{Scala} code.

\break
\begin{lstlisting}[language=Scala]
  trait Expressions {
    type Nominal    
    abstract class Expression

    case class Mention( reference : Nominal )
       extends Expression

    case class Abstraction(
       formals : List[Nominal],
       body : Expression
    )  extends Expression

    case class Application(
       operation : Expression,
       actuals : List[Expression]
    )  extends Expression        
  }
\end{lstlisting}

In this representation each \emph{syntactic category}, EXPRESSION, MENTION,
ABSTRACTION and APPLICATION, is represented by a
\lstinline[language=Scala]!trait! or \lstinline[language=Scala]!case class!.
EXPRESSION's are \lstinline[language=Scala]!trait!'s because they
are pure placeholders. The other categories elaborate the syntactic
form, and the elaboration is matched by the
\lstinline[language=Scala]!case class!  structure. Thus, for example,
an ABSTRACTION is modeled by an instance of the
\lstinline[language=Scala]!case class! called
\lstinline[language=Scala]!Abstraction! having members
\lstinline[language=Scala]!formal! for the formal parameter of the
abstraction, and \lstinline[language=Scala]!body! for the
$\lambda$-term under the abstraction that might make use of the
parameter. Similarly, an APPLICATION is modeled by an instance of the
\lstinline[language=Scala]!case class! of the same name having members
\lstinline[language=Scala]!operation! for the expression that will be applied
to the actual parameter called (not surprisingly)
\lstinline[language=Scala]!actual!.

\paragraph{Type parametrization and quotation}
One key aspect of this representation is that we acknowledge that the
abstract syntax is strangely silent on what the \emph{terminals}
are. It doesn't actually say what $x$'s are. Often implementations of
the $\lambda$-calculus will make some choice, such as
\lstinline[language=Scala]!String!s or
\lstinline[language=Scala]!Integers! or some other
representation. With \texttt{Scala}'s type parametrization we can
defer this choice. In fact, to foreshadow some of what's to come, we
illustrate that we never actually have to go outside of the basic
grammar definition to come up with a supply of identifiers.

In the code above we have deferred the choice of identifier. In the
code below we provide several different kinds of identifiers (the term
of art in this context is ``name''), but defer the notion of an
expression by the same trick used to defer the choice of identifiers.

\begin{lstlisting}[language=Scala]
  trait Nominals {
    type Term
    abstract class Name
    case class Transcription( expression : Term )
       extends Name
    case class StringLiteral( str : String )
       extends Name
    case class DeBruijn( outerIndex : Int, innerIndex : Int )
       extends Name
    case class URLLiteral( url : java.net.URL )
       extends Name
  }
\end{lstlisting}

Now we wire the two types together.

\begin{lstlisting}[language=Scala]
  trait ReflectiveGenerators
  extends Expressions with Nominals {
    type Nominal = Name
    type Term = Expression
  }
\end{lstlisting}

This allows us to use \emph{quoted} terms as variables in
$lambda$-terms! The idea is very rich as it begs the question of
whether such variables can be \emph{unquoted} and what that means for
evaluation. Thus, \texttt{Scala}'s type system is already leading
to some pretty interesting places! In fact, this is an instance of a
much deeper design principle lurking here, called two-level type
decomposition, that is enabled by type-level parametricity. We'll talk
more about this in upcoming chapters, but just want to put it on the
backlog.

\paragraph{Some syntactic sugar}
To this core let us add some syntactic sugar.

\begin{mathpar}
  \inferrule* [lab=previous] {} {{M,N} ::= ...}
  \and
  \inferrule* [lab=let] {} {\;| \; let \; x = M \; in \; N}
  \and
  \inferrule* [lab=seq] {} {\;| \; M;N}
\end{mathpar} 

This is sugar because we can reduce $let \; x \; = \; M \; in \; N$ to
$(\lambda x. M) N$ and $ M; N$ to $let \; x \; = \; M \; in \; N$ with
$x$ not occurring in $N$.

\paragraph{Digression: complexity management principle} In terms of
our implementation, the existence of this reduction means that we can
choose to have explicit representation of these syntactic categories
or not. This choice is one of a those design situations that's of
significant interest if our concern is complexity management. [Note:
brief discussion of the relevance of super combinators.]

\paragraph{Concrete syntax}
Now let's wrap this up in concrete syntax.

\begin{mathpar}
  \inferrule* [lab=expression] {} {{M,N} ::=}
  \and
  \inferrule* [lab=mention] {} {x}
  \and
  \inferrule* [lab=abstraction] {} {\;| \; \texttt{(} x_1 \texttt{,} ... \texttt{,} x_k \texttt{)} \; \texttt{=>} \; M}
  \and
  \inferrule* [lab=application] {} {\;| \; M\texttt{(} N_1 \texttt{,} ... \texttt{,} N_k \texttt{)}}
  \and
  \inferrule* [lab=let] {} {\;| \; \texttt{val} \; x \; \texttt{=} \; M \texttt{;} N}
  \and
  \inferrule* [lab=seq] {} {\;| \; M \texttt{;} N }
  \and
  \inferrule* [lab=group] {} {\;| \; \texttt{ \{ } M \texttt{ \} } }
\end{mathpar} 

It doesn't take much squinting to see that this looks a lot like a
subset of \texttt{Scala}, and that's because -- of course! --
functional languages like \texttt{Scala} all share a common core that
is essentially the $\lambda$-calculus. Once you familiarize yourself
with the $\lambda$-calculus as a kind of design pattern you'll see it
poking out everywhere: in \texttt{Clojure} and \texttt{OCaml} and
\texttt{F\#} and \texttt{Scala}. In fact, as we'll see later, just
about any DSL you design that needs a notion of variables could do
worse than simply to crib from this existing and well understood
design pattern.

If you've been following along so far, however, you will spot that
something is actually wrong with this grammar. We still don't have an
actual terminal! \emph{Concrete} syntax is what ``users'' type, so as
soon as we get to concrete syntax we can no longer defer our choices
about identifiers. Let's leave open the door for both ordinary
identifiers -- such as we see in \texttt{Scala} -- and our funny
quoted terms. This means we need to add the following productions to
our grammar.

\begin{mathpar}
  \inferrule* [lab=identifier] {} {{x,y} ::=}
  \and
  \inferrule* [lab=string-id] {} {\;| \; String}
  \and
  \inferrule* [lab=quotation] {} {\;| \; \texttt{@} \texttt{<} M \texttt{>}}
\end{mathpar} 

(The reason we use the \texttt{@} for quotation -- as will become
clear later -- is that when we have both quote and dequote, the former
functions a lot like asking for a \emph{pointer} to a term while the
latter is a lot like dereferencing the pointer.)

\paragraph{Translating concrete syntax to abstract syntax}
The translation from the concrete syntax to the abstract syntax is
compactly expressed as follows. Perhaps the best way to understand
this presentation is in terms of a \texttt{Scala} implementation.

[TBD]

\paragraph{Structural equivalence and Relations or What makes abstract syntax abstract}

Apart from the fact that concrete syntax forces commitment to explicit
representation of terminals, you might be wondering if there are any
other differences between concrete and abstract syntax. It turns out
there are. One of the key properties of abstract syntax is that it
encodes a notion of equality of terms that is not generally
represented in concrete syntax.

It's easier to illustrate the idea in terms of our example. We know
that programs that differ only by a change of bound variable are
essentially the same. Concretely, the program
\lstinline[language=Scala]!( x ) => x + 5! is essentially the same as
the program \lstinline[language=Scala]!( y ) => y + 5!. By
``essentially the same'' we mean that in every evaluation context
where we might put the former if we substitute the latter we will get
the same answer. 

However, this sort of equivalence doesn't have to be all intertwined
with evaluation to be expressed. A little forethought shows we can
achieve some separation of concerns by separating out certain kinds of
\emph{structural} equivalences. Abstract syntax is where we express
structural equivalence (often written using $\equiv$, for example $M \equiv N$).
In terms of our example we can actually calculate when two
$\lambda$-terms differ only by a change of \emph{bound} variable,
where by bound variable we just mean a variable mention in a term also
using the variable as formal parameter of an abstraction.

Since we'll need that notion to express this structural equivalence,
let's write some code to clarify the idea, but because it will be more
convenient, let's calculate the variables not occurring bound,
i.e. the \emph{free} variables of a $\lambda$-term.

\begin{lstlisting}[language=Scala]
    def freeVariables( term : Expression ) : Set[Nominal] = {
      term match {
        case Mention( reference ) => Set( reference )
        case Abstraction( formals, body ) =>
	   freeVariables( body ) &~ formals.toSet
        case Application( operation, actuals ) =>
	   ( freeVariables( operation ) /: actuals )(
           { ( acc, elem ) => acc ++ freeVariables( elem ) } 
           )
      }
    }
\end{lstlisting}

In addition to this idea we'll need to represent exchanging bound
variables. A traditional way to approach this is in terms of
substituting a term (including variables) for a variable. A crucial
point is that we need to avoid unwanted variable capture. For example,
suppose we need to substitute $y$ for $x$ in a term of the form
$\lambda y.(x y)$. Doing so blindly will result in a term of the form
$\lambda y.(y y)$; but, now the first $y$ is bound by the abstraction
-- probably not what we want. To avoid this -- using structural
equivalence! -- we can move the bound variable ``out of the
way''. That is, we can first change the term to an equivalent one, say
$\lambda u.(x u)$. Now, we can make the substitution, resulting in
$\lambda u.(y u)$. This is what's called capture-avoiding
substitution. Central to this trick is the ability to come up with a
fresh variable, one that doesn't occur in a term. Obviously, any
implementation of this trick is going to depend implicitly on the
internal structure of names. Until we have such a structure in hand we
have to defer the implementation, but we mark it with a placeholder.

\begin{lstlisting}[language=Scala]
  def fresh( terms : List[Expression] ) : Nominal
\end{lstlisting}

Now we can write in \texttt{Scala} our definition of substitution.

\break
\begin{lstlisting}[language=Scala]
   def substitute(
    term : Expression,
    actuals : List[Expression],
    formals : List[Nominal]
  ) : Expression = {
    term match {
      case Mention( ref ) => {
	formals.indexOf( ref ) match {
	  case -1 => term
	  case i => actuals( i )
	}
      }
      case Abstraction( fmls, body ) => {
	val fmlsN = fmls.map(
	  {
	    ( fml ) => {
	      formals.indexOf( fml ) match {
		case -1 => fml
		case i => fresh( List( body ) )
	      }
	    }
	  }	      
	)
	val bodyN =
	  substitute(
	    body,
	    fmlsN.map( _ => Mention( _ ) ),
	    fmlsN
	  )
	Abstraction(
	  fmlsN,
	  substitute( bodyN, actuals, formals )
	)
      }
      case Application( op, actls ) => {
	Application(
	  substitute( op, actuals, formals ),
	  actls.map( _ => substitute( _, actuals, formals ) )
	)
      }
    }
  }
\end{lstlisting}

With this code in hand we have what we need to express the structural
equivalence of terms.

\begin{lstlisting}[language=Scala]  
  def `=a=`(
       term1 : Expression,
       term2 : Expression
   ) : Boolean = {
   ( term1, term2 ) match {
      case (
	Mention( ref1 ),
	Mention( ref2 )
      ) => {
	ref1 == ref2
      }
      case (
	Abstraction( fmls1, body1 ), Abstraction( fmls2, body2 )
      ) => {
	if ( fmls1.length == fmls2.length ) {
	  val freshFmls =
	    fmls1.map(
	      { ( fml ) => Mention( fresh( List( body1, body2 ) ) ) }
	    )
	  `=a=`(
	    substitute( body1, freshFmls, fmls1 ),
	    substitute( body2, freshFmls, fmls2 )
	  )
	}
	else false
      }      
      case (
	Application( op1, actls1 ),
	Application( op2, actls2 )
      ) => {
	( `=a=`( op1, op2 ) /: actls1.zip actls2 )(
	  { ( acc, actlPair ) =>
	    acc && `=a=`( actlPair._1, actlPair._2 )
	 }
	)
      }
    }
  }
\end{lstlisting}

This is actually some significant piece of machinery just to be able
to calculate what we mean when we write $M[N/x]$ and $M \equiv N$.
People have wondered if this sort of machinery could be reasonably
factored so that it could be mixed into a variety of variable-binding
capabilities. It turns out that this is possible and is at the root of
a whole family of language design proposals that began with Jamie
Gabbay's \texttt{FreshML}.

\paragraph{Digression: the internal structure of the type of variables}
If you've been paying attention, you will note that there's something
very funny going on in the calculation of
\lstinline[language=Scala]!freeVariables!. To actually perform the
\lstinline[language=Scala]!remove! or the
\lstinline[language=Scala]!union! we have to have equality defined on
variables. Now, this works fine for
\lstinline[language=Scala]!String!s, but what about
\lstinline[language=Scala]!Quotations!?

The question reveals something quite startling about the types of
variables. Clearly, the type has to include a definition of
equality. Now, if we want to have an inexhaustible supply of
variables, then the definition of equality of variables must make use
of the ``internal'' structure of the variables. For example, checking
equality of \lstinline[language=Scala]!String!s means checking the
equality of the respective sequences of characters. There are a finite
set of characters out of which all \lstinline[language=Scala]!String!s
are built and so eventually the calculation of equality grounds
out. The same would be true if we used
\lstinline[language=Scala]!Integer!s as ``variables''. If our type of
variables didn't have some evident internal structure (like a
\lstinline[language=Scala]!String! has characters or an
\lstinline[language=Scala]!Integer! has arithmetic structure) and yet
it was to provide an endless supply of variables, then the equality
check could only be an \emph{infinite} table -- which wouldn't fit
inside a computer. So, the type of variables must also support some
internal structure, i.e. it must be a container type!

Fortunately, our \lstinline[language=Scala]!Quotation!s are
containers, by definition. However, they face another challenge: are
the \lstinline[language=Scala]!Quotation!s of two structurally
equivalent terms equal as variables? If they are then there is a
mutual recursion! Equivalence of terms depends on equality of
\lstinline[language=Scala]!Quotation!s which depends on equivalence of
terms. It turns out that we have cleverly arranged things so that this
recursion will bottom out. The key property of the structure of the
abstract syntax that makes this work is that there is an alternation:
quotation, term constructor, quotation, ... . Each recursive call will
lower the number of quotes, until we reach 0.

\paragraph{Evaluation -- aka operational semantics}

Here's the code

\begin{lstlisting}[language=Scala]
  trait Values {
    type Environment
    type Expression
    abstract class Value
    case class Closure(
       fn : List[Value] => Value
    ) extends Value
    case class Quantity( quantity : Int )
      extends Value
 }

 trait Reduction extends Expressions with Values {
   type Dereferencer = {def apply( m : Mention ) : Value }
   type Expansionist =
   {def extend(
     fmls : List[Mention],
     actls : List[Value]
     ) : Dereferencer}
   type Environment <: (Dereferencer with Expansionist)
   type Applicator = Expression => List[Value] => Value

   val initialApplicator : Applicator =
   { ( xpr : Expression ) => {
       ( actls : List[Value] ) => {
         xpr match {
           case IntegerExpression( i ) => Quantity( i )
           case _ => throw new Exception( "why are we here?" )
         }
       }
     }
   }
   
   def reduce(
      applicator : Applicator,
      environment : Environment
   ) : Expression => Value = { 
     case IntegerExpression( i ) => Quantity( i )
     case Mention( v ) => environment( Mention( v ) )
     case Abstraction( fmls, body ) =>
     Closure(
        { ( actuals : List[Value] ) => {
            val keys : List[Mention] =
 	    fmls.map( { ( fml : Nominal ) => Mention( fml ) });
            reduce(
	       applicator,
               environment.extend(
                  keys,
                  actuals ).asInstanceOf[Environment]
            )( body )
	  }
        }
     )
     case Application(
        operator : Expression,
        actuals : List[Expression]
     ) => {
	reduce( applicator, environment )( operator ) match {
	  case Closure( fn ) => {
	    fn.apply(
	      (actuals
	       map
	       {( actual : Expression) =>
		 (reduce( applicator, environment ))( actual )})
	    )
	  }
	  case _ =>
             throw new Exception( "attempt to apply non function" )
	}
      }
    case _ => throw new Exception( "not implemented, yet" )
  }
}
\end{lstlisting}

\subsubsection{What goes into a language definition}

Before moving to the next chapter it is important to digest what we've
done here. Since we've called out DSL-based design as a methodology
worthy of attention, what does our little foray into defining a
language tell us about language definition? It turns out that this is
really part of folk lore in the programming language semantics
community. At this point in time one of the commonly accepted
presentations of a language definition has three components:

\begin{itemize}
  \item syntax -- usually given in terms of some variant of BNF
  \item structural equivalence -- usually given in terms of a set of relations
  \item operational semantics -- usually given as a set of conditional
    rewrite rules, such as might be expressed in SOS format.
\end{itemize}

That's exactly what we see here. Our toy language can be completely
characterized by the following aforementioned half-page specification. 

\paragraph{Syntax}

\begin{mathpar}
  \inferrule* [lab=expression] {} {{M,N} ::=}
  \and
  \inferrule* [lab=mention] {} {x}
  \and
  \inferrule* [lab=abstraction] {} {\;| \; \lambda x . M}
  \and
  \inferrule* [lab=application] {} {\;| \; M N}
\end{mathpar} 

\paragraph{Structural equivalence}

\begin{mathpar}
  \inferrule*[left=$\alpha$-equivalence] { y \notin \mathcal{FN}( M ) } { \lambda x . M = \lambda y. M[y/x] }
\end{mathpar}

where

\begin{mathpar}
  \inferrule* {} {\mathcal{FN}( x ) = x} \and
  \inferrule* {} {\mathcal{FN}( \lambda x . M ) = \mathcal{FN}( M ) \backslash \{ x \} } \and
  \inferrule* {} {\mathcal{FN}( M N ) = \mathcal{FN}( M ) \cup \mathcal{FN}( N ) }
\end{mathpar}

and we write $M[y/x]$ to mean substitute a $y$ for every occurrence of $x$ in $M$.

\paragraph{Operational semantics}

\begin{mathpar}
  \inferrule*[left=$\beta$-reduction] {} {( \lambda x. M )N \to M[N/y] } \and
  \inferrule*[left=struct] { M \equiv M', M' \to N', N' \equiv N} {M \to N}
\end{mathpar}

\paragraph{Discussion}
This specification leaves open some questions regarding order or
evaluation. In this sense it's a kind of proto-specification. For
example, to get a left-most evaluation order you could add the rule

\begin{mathpar}
  \inferrule*[left=leftmost] { M \to M'} {M N \to M' N}
\end{mathpar}

The \texttt{Scala} code we wrote in the preceding sections is
essentially an elaboration of this specification. While this notation
is clearly more compact, it is not hard to recognize that the code
follows this structure very closely.



% \section{Existence problems}
% We begin with some metamathematics.
% All problems about the existence of maps can be cast into one of the
% following two forms, which are in a sense mutually dual.

% \noindent
% {\bf The Extension Problem}\index{extension problem} \    %%% NB index entry tag
% Given an inclusion $ A \stackrel{i}{\hookrightarrow} X $, and a map
% $ A \stackrel{f}{\rightarrow} Y $,
% does there exist a map $f^{\dagger}:X\to Y$ such that
% $f^{\dagger}$ agrees with $f$ on $A$?

% Here the appropriate source category for maps should be clear from the
% context and, moreover, commutativity through a
% candidate $f^{\dagger}$ is precisely
% the restriction requirement; that is,
% $$f^{\dagger}   :  f^{\dagger}\circ i = f^{\dagger}|_A = f\,. $$
% If such an $f^{\dagger}$ exists\footnote{${}^{\dagger}$ suggests striving
% for perfection, crusading}, then it is called an {\bf
% extension}\index{extension!of a map|bi} of $f$ and is said to {\bf
% extend}\index{extend|bi} $f$. In any diagrams, the presence of
% a dotted arrow or an arrow carrying a ? indicates a pious hope, in no way
% begging the question of its existence. Note that we shall usually
% omit $\circ$ from composite maps.

% \noindent
% {\bf The Lifting Problem}\index{lifting problem} \
% Given a pair of maps $E \stackrel{p}{\rightarrow}B$ and $X \stackrel{f}
% {\rightarrow} B $,
% does there exist a map $f^{\circ} : X \to E$, with
% $pf^{\circ} = f  $?


% That {\em all\/} existence problems about maps are essentially of one
% type or
% the other from these two is seen as follows. Evidently, all existence problems
% are representable by triangular diagrams\index{triangular diagrams} and it
% is easily seen that there are only these six possibilities:
% \begin{center}\begin{picture}(300,70)  %augch2 75
% \put(5,60){\vector(1,0){30}}
% \put(55,60){\vector(1,0){30}}
% \put(135,60){\vector(-1,0){30}}
% \put(185,60){\vector(-1,0){30}}
% \put(235,60){\vector(-1,0){30}}
% \put(285,60){\vector(-1,0){30}}
% \put(0,55){\vector(0,-1){30}}
% \put(50,55){\vector(0,-1){30}}
% \put(100,25){\vector(0,1){30}}
% \put(150,25){\vector(0,1){30}}
% \put(200,55){\vector(0,-1){30}}
% \put(250,55){\vector(0,-1){30}}
% \put(28,33){\small ?}
% \put(78,33){\small ?}
% \put(128,33){\small ?}
% \put(178,33){\small ?}
% \put(228,33){\small ?}
% \put(278,33){\small ?}
% \put(10,3){\bf 1}
% \put(60,3){\bf 2}
% \put(110,3){\bf 3}
% \put(160,3){\bf 4}
% \put(210,3){\bf 5}
% \put(260,3){\bf 6}
% \put(35,55){\vector(-1,-1){30}}
% \put(155,25){\vector(1,1){30}}
% \put(135,55){\vector(-1,-1){30}}
% \put(55,25){\vector(1,1){30}}
% \put(235,55){\vector(-1,-1){30}}
% \put(255,25){\vector(1,1){30}}
% \end{picture}\end{center}



% \begin{figure}
% \begin{picture}(300,220)(0,0)
% \put(-20,-20){\resizebox{20 cm}{!}{\includegraphics{3dpdf}}}
% \put(260,-10){\resizebox{15 cm}{!}{\includegraphics{contpdf}}}
% \put(220,80){$\beta$}
% \put(400,-10){$N$}
% \put(260,170){$\beta$}
% \put(90,15){$N$}
% \end{picture}
% \caption{{\em The log-gamma family of densities with central mean
% $<N> \, = \frac{1}{2}$ as a surface and as a contour plot. }}
% \label{pdf}
% \end{figure}

\newpage
