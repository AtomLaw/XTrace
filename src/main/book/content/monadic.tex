
\documentclass[12pt,leqno]{book}
\usepackage{amsmath,amssymb,amsfonts} % Typical maths resource packages
\usepackage{graphicx}                 % Packages to allow inclusion of graphics
\usepackage{color}                    % For creating coloured text and background
\usepackage{hyperref}                 % For creating hyperlinks in cross references
\usepackage{makeidx}                  % For indexing
\usepackage{listings}                 % For code listing
\usepackage{mathpartir}               % For grammars, rules, etc
\usepackage{bcprules}                 % For other kinds of rules
\usepackage{diagrams}                 % For commutative diagrams

\lstloadlanguages{Scala,Java,Haskell,XML,bash,HTML,SQL}

\parindent 1cm
\parskip 0.2cm
\topmargin 0.2cm
\oddsidemargin 1cm
\evensidemargin 0.5cm
\textwidth 15cm
\textheight 21cm

\newtheorem{theorem}{Theorem}[section]
\newtheorem{proposition}[theorem]{Proposition}
\newtheorem{corollary}[theorem]{Corollary}
\newtheorem{lemma}[theorem]{Lemma}
\newtheorem{remark}[theorem]{Remark}
\newtheorem{definition}[theorem]{Definition}


\def\R{\mathbb{ R}}
\def\S{\mathbb{ S}}
\def\I{\mathbb{ I}}

\def\Scala{\texttt{Scala}}
\def\ScalaCheck{\texttt{ScalaCheck}}
\def\Haskell{\texttt{Haskell}}
\def\XML{\texttt{XML}}


\makeindex


\title{Pro Scala: Monadic Design Patterns for the Web}

\author{L.G. Meredith  \\
{\small\em \copyright \  Draft date \today }}

 \date{ }
\begin{document}
\lstset{language=Haskell}
\maketitle
 \addcontentsline{toc}{chapter}{Contents}
\pagenumbering{roman}
\tableofcontents
\listoffigures
\listoftables
\chapter*{Preface}\normalsize
  \addcontentsline{toc}{chapter}{Preface}
\pagestyle{plain}
% The book root file {\tt bookex.tex} gives a basic example of how to
% use \LaTeX \ for preparation of a book. Note that all
% \LaTeX \ commands begin with a
% backslash.

% Each
% Chapter, Appendix and the Index is made as a {\tt *.tex} file and is
% called in by the {\tt include} command---thus {\tt ch1.tex} is
% the name here of the file containing Chapter~1. The inclusion of any
% particular file can be suppressed by prefixing the line by a
% percent sign.


%  Do not put an {\tt end{document}} command at the end of chapter files;
% just one such command is needed at the end of the book.

% Note the tag used to make an index entry. You may need to consult Lamport's
% book~\cite{lamport} for details of the procedure to make the index input
% file; \LaTeX \ will create a pre-index by listing all the tagged
% items in the file {\tt bookex.idx} then you edit this into
% a {\tt theindex} environment, as {\tt index.tex}.

The book you hold in your hands, Dear Reader, is not at all what you expected...



\pagestyle{headings}
\pagenumbering{arabic}

%ch.tex


\chapter{The semantic web}
\begin{center}
{\small\em Where are we; how did we get here; and where are we going?}
\end{center}

TBD
\section{How our web framework enables different kinds of application queries}

\subsubsection{An alternative presentation}

If you recall, there's an alternative way to present monads that are
algebras, like our monoid monad. Algebras are presented in terms of
generators and relations. In our case the generators presentation is
really just a grammar for monoid expressions.

\begin{mathpar}
  \inferrule* [lab=expression] {} {{m,n} ::=}
  \and
  \inferrule* [lab=identity element] {} {e}
  \and
  \inferrule* [lab=generators] {} {\;| \; g_1 \; | \; ... \; | \; g_n}
  \and
  \inferrule* [lab=monoid-multiplication] {} {\;| \; m * n}
\end{mathpar} 

This is subject to the following constraints, meaning that we will
treat syntactic expressions of certain forms as denoting the same
element of the monoid. To emphasize the nearly purely syntactic role
of these constraints we will use a different symbol for the
constraints. We also use the same symbol, $\equiv$, for the smallest equivalence
relation respecting these constraints.

\begin{mathpar}
  \inferrule* [lab=identity laws] {} {m * e \equiv m \equiv e * m}
  \and
  \inferrule* [lab=associativity] {} {m_1 * (m_2 * m_3) \equiv (m_1 * m_2) * m_3}
\end{mathpar} 

\paragraph{Logic: the set monad as an algebra}
In a similar manner, there is a language associated with the monad of
sets \emph{considered as an algebra}. This language is very familiar
to most programmers.

\begin{mathpar}
  \inferrule* [lab=expression] {} {{c,d} ::=}
  \and
  \inferrule* [lab=identity verity] {} {true}
  \and
  \inferrule* [lab=negation] {} {\;| \; \neg c}
  \and
  \inferrule* [lab=conjunction] {} {\;| \; c \& d}
\end{mathpar} 

Now, if we had a specific set in hand, say $L$ (which we'll call a
universe in the sequel), we can interpret the expressions in the this
language, aka formulae, in terms of operations on subsets of that
set. As with our compiler for the concrete syntax of the
$lambda$-calculus in chapter 1, we can express this translation very
compactly as

\begin{mathpar}
  \inferrule* {} {\meaningof{true} = L}
  \and
  \inferrule* {} {\meaningof{\neg c} = L \backslash c}
  \and 
  \inferrule* {} {\meaningof{c \& d} = \meaningof{c} \cap \meaningof{d}}
\end{mathpar}

Now, what's happening when we pull the monoid monad through the set
monad via a distributive map is this. First, the monoid monad
furnishes the universe, $L$, as the set of expressions generated by
the grammar. We'll denote this by $L(m)$. Then, we enrich the set of
formulae by the operations of the monoid \emph{acting on sets}.

\begin{mathpar}
  \inferrule* [lab=expression] {} {{c,d} ::=}
  \and
  \inferrule* [lab=identity verity] {} {true}
  \and
  \inferrule* [lab=negation] {} {\;| \; \neg c}
  \and
  \inferrule* [lab=conjunction] {} {\;| \; c \& d}
  \and
  \inferrule* [lab=identity verity] {} {\bf{e}}
  \and
  \inferrule* [lab=negation] {} {\;| \; \bf{g_1} \; | \; ... \; | \; \bf{g_n}}
  \and
  \inferrule* [lab=conjunction] {} {\;| \; c * d}
\end{mathpar} 

The identity element, $e$ and the generators of the monoid, $g_1$,
..., $g_n$, can be considered $0$-ary operations in the same way that
we usually consider constants as $0$-ary operations. To avoid
confusion between these elements and the \emph{logical formulae} that
pick them out of the crowd, we write the logical formulae in
$\bf{boldface}$.

Now, we can write our distributive map. Surprisingly, it is exactly a
meaning for our logic!

\begin{mathpar}
  \inferrule* {} {\meaningof{true} = L(m)}
  \and
  \inferrule* {} {\meaningof{\neg c} = L(m) \backslash c}
  \and 
  \inferrule* {} {\meaningof{c \& d} = \meaningof{c} \cap \meaningof{d}}
  \and
  \inferrule* {} {\meaningof{\bf{e}} = \{ m \; \in \; L(m) \; | \; m \equiv e \}}
  \and
  \inferrule* {} {\meaningof{\bf{g_i}} = \{ m \; \in \; L(m) \; | \; m \equiv g_i \}}
  \and
  \inferrule* {} {\meaningof{c*d} = \{ m \; \in \; L(m) \; | \; m \equiv m_1 * m_2, m_1 \; \in \; \meaningof{c}, m_2 \; \in \; \meaningof{d} \}}
\end{mathpar}

\paragraph{Primes: an application}
Before going any further, let's look at an example of how to use these
new operators. Suppose we wanted to pick out all the elements of the
monoid that were not expressible as a composition of other
elements. Obviously, for monoids with a finite set of generators, this
is exactly just the generators, so we could write $\bf{g_1} || ... ||
\bf{g_n}$\footnote{We get the disjunction, $||$, by the usual DeMorgan
  translation: $c || d \stackrel{def}{=} \neg( \neg c \& \neg
  d)$}. However, when the set of generators is not finite, as it is
when the monoid is the integers under multiplication, we need another
way to write this down. That's where our other operators come in
handy. A moment's thought suggests that we could say that since $true$
denotes any possible element in the monoid, an element is not a
composition using negation plus our composition formula, i.e. $\neg
(true * true)$. This is a little overkill, however. We just want to
eliminate non-trivial compositions. We know how to express the
identity element, that's $\bf{e}$, so we are interested in those
elements that are not the identity, i.e. $\neg \bf{e}$. Then a formula
that eliminates compositions of non-trivial elements is spelled out
$\neg (\neg e * \neg e)$. Finally, we want to eliminate the identity
as a solution. So, we arrive at $\neg (\neg e * \neg e) \& \neg
e$. There, that formula picks out the \emph{primes} of \emph{any}
monoid.

\paragraph{Summary}

What have we done? We've illustrated a specific distributive map, one
that pulls the set monad through the monoid monad. We've shown that
this particular distributive map coincides with giving a semantics to
a particular logic, one whose structure is derived solely from the
shape of the collection monad, i.e. set, and the shape of the term
language, in this case monoid.

\subsubsection{Iterating the design pattern}

The whole point of working in this manner is that by virtue of its
compositional structure it provides a much higher level of abstraction
and greater opportunities for reuse. To illustrate the point, we will
now iterate the construction using our toy language, the
$lambda$-calculus, as the term language. As we saw in chapter 1, the
$lambda$-calculus also has a generators and relations
presentation. Unlike a monoid, however, the lambda calculus has
another piece of machinery: reduction! In addition to structural
equivalence of terms (which is a bi-directional relation) there is the
$beta$-reduction rule that captures the \emph{behavioral} aspect of
the lambda calculus.

It is key to understand this underlying structure of language
definitions. In essence, when a DSL is purely about structure it is
presented entirely in terms of generators (read a grammar) and
relations (like the monoid laws). When the DSL is also about behavior,
i.e. the terms in the language somehow express some kind of
computation, then the language has a third component, some kind of
reduction relation. \footnote{In some sense this is one of the central
  contributions of the theory of computation back to
  mathematics. Algebraists have known for a long time about generators
  and relations presentations of algebraic structures (of which
  algebraic data types are a subset). This collective wisdom is
  studied, for example, in the field of universal
  algebra. Computational models like the $lambda$-calculus and more
  recently the process calculi, like Milner's $\pi$-calculus or
  Cardelli and Gordon's ambient calculus, take this presentation one
  step further and add a set of conditional rewrite rules to express
  the computational content of the model. It was Milner who first
  recognized this particular decomposition of language definitions in
  his seminal paper, Functions as Processes, where he reformulated the
  presentation $\pi$-calculus along these lines.} This organization,
this common factoring of the specification of a language, makes it
possible to factor code that handles a wide range of semantic
features. The logic we derive below provides a great example.

\section{Searching for programs}

TBD

% \section{Existence problems}
% We begin with some metamathematics.
% All problems about the existence of maps can be cast into one of the
% following two forms, which are in a sense mutually dual.

% \noindent
% {\bf The Extension Problem}\index{extension problem} \    %%% NB index entry tag
% Given an inclusion $ A \stackrel{i}{\hookrightarrow} X $, and a map
% $ A \stackrel{f}{\rightarrow} Y $,
% does there exist a map $f^{\dagger}:X\to Y$ such that
% $f^{\dagger}$ agrees with $f$ on $A$?

% Here the appropriate source category for maps should be clear from the
% context and, moreover, commutativity through a
% candidate $f^{\dagger}$ is precisely
% the restriction requirement; that is,
% $$f^{\dagger}   :  f^{\dagger}\circ i = f^{\dagger}|_A = f\,. $$
% If such an $f^{\dagger}$ exists\footnote{${}^{\dagger}$ suggests striving
% for perfection, crusading}, then it is called an {\bf
% extension}\index{extension!of a map|bi} of $f$ and is said to {\bf
% extend}\index{extend|bi} $f$. In any diagrams, the presence of
% a dotted arrow or an arrow carrying a ? indicates a pious hope, in no way
% begging the question of its existence. Note that we shall usually
% omit $\circ$ from composite maps.

% \noindent
% {\bf The Lifting Problem}\index{lifting problem} \
% Given a pair of maps $E \stackrel{p}{\rightarrow}B$ and $X \stackrel{f}
% {\rightarrow} B $,
% does there exist a map $f^{\circ} : X \to E$, with
% $pf^{\circ} = f  $?


% That {\em all\/} existence problems about maps are essentially of one
% type or
% the other from these two is seen as follows. Evidently, all existence problems
% are representable by triangular diagrams\index{triangular diagrams} and it
% is easily seen that there are only these six possibilities:
% \begin{center}\begin{picture}(300,70)  %augch2 75
% \put(5,60){\vector(1,0){30}}
% \put(55,60){\vector(1,0){30}}
% \put(135,60){\vector(-1,0){30}}
% \put(185,60){\vector(-1,0){30}}
% \put(235,60){\vector(-1,0){30}}
% \put(285,60){\vector(-1,0){30}}
% \put(0,55){\vector(0,-1){30}}
% \put(50,55){\vector(0,-1){30}}
% \put(100,25){\vector(0,1){30}}
% \put(150,25){\vector(0,1){30}}
% \put(200,55){\vector(0,-1){30}}
% \put(250,55){\vector(0,-1){30}}
% \put(28,33){\small ?}
% \put(78,33){\small ?}
% \put(128,33){\small ?}
% \put(178,33){\small ?}
% \put(228,33){\small ?}
% \put(278,33){\small ?}
% \put(10,3){\bf 1}
% \put(60,3){\bf 2}
% \put(110,3){\bf 3}
% \put(160,3){\bf 4}
% \put(210,3){\bf 5}
% \put(260,3){\bf 6}
% \put(35,55){\vector(-1,-1){30}}
% \put(155,25){\vector(1,1){30}}
% \put(135,55){\vector(-1,-1){30}}
% \put(55,25){\vector(1,1){30}}
% \put(235,55){\vector(-1,-1){30}}
% \put(255,25){\vector(1,1){30}}
% \end{picture}\end{center}



% \begin{figure}
% \begin{picture}(300,220)(0,0)
% \put(-20,-20){\resizebox{20 cm}{!}{\includegraphics{3dpdf}}}
% \put(260,-10){\resizebox{15 cm}{!}{\includegraphics{contpdf}}}
% \put(220,80){$\beta$}
% \put(400,-10){$N$}
% \put(260,170){$\beta$}
% \put(90,15){$N$}
% \end{picture}
% \caption{{\em The log-gamma family of densities with central mean
% $<N> \, = \frac{1}{2}$ as a surface and as a contour plot. }}
% \label{pdf}
% \end{figure}

\newpage

%ch.tex


\chapter{The semantic web}
\begin{center}
{\small\em Where are we; how did we get here; and where are we going?}
\end{center}

TBD
\section{How our web framework enables different kinds of application queries}

\subsubsection{An alternative presentation}

If you recall, there's an alternative way to present monads that are
algebras, like our monoid monad. Algebras are presented in terms of
generators and relations. In our case the generators presentation is
really just a grammar for monoid expressions.

\begin{mathpar}
  \inferrule* [lab=expression] {} {{m,n} ::=}
  \and
  \inferrule* [lab=identity element] {} {e}
  \and
  \inferrule* [lab=generators] {} {\;| \; g_1 \; | \; ... \; | \; g_n}
  \and
  \inferrule* [lab=monoid-multiplication] {} {\;| \; m * n}
\end{mathpar} 

This is subject to the following constraints, meaning that we will
treat syntactic expressions of certain forms as denoting the same
element of the monoid. To emphasize the nearly purely syntactic role
of these constraints we will use a different symbol for the
constraints. We also use the same symbol, $\equiv$, for the smallest equivalence
relation respecting these constraints.

\begin{mathpar}
  \inferrule* [lab=identity laws] {} {m * e \equiv m \equiv e * m}
  \and
  \inferrule* [lab=associativity] {} {m_1 * (m_2 * m_3) \equiv (m_1 * m_2) * m_3}
\end{mathpar} 

\paragraph{Logic: the set monad as an algebra}
In a similar manner, there is a language associated with the monad of
sets \emph{considered as an algebra}. This language is very familiar
to most programmers.

\begin{mathpar}
  \inferrule* [lab=expression] {} {{c,d} ::=}
  \and
  \inferrule* [lab=identity verity] {} {true}
  \and
  \inferrule* [lab=negation] {} {\;| \; \neg c}
  \and
  \inferrule* [lab=conjunction] {} {\;| \; c \& d}
\end{mathpar} 

Now, if we had a specific set in hand, say $L$ (which we'll call a
universe in the sequel), we can interpret the expressions in the this
language, aka formulae, in terms of operations on subsets of that
set. As with our compiler for the concrete syntax of the
$lambda$-calculus in chapter 1, we can express this translation very
compactly as

\begin{mathpar}
  \inferrule* {} {\meaningof{true} = L}
  \and
  \inferrule* {} {\meaningof{\neg c} = L \backslash c}
  \and 
  \inferrule* {} {\meaningof{c \& d} = \meaningof{c} \cap \meaningof{d}}
\end{mathpar}

Now, what's happening when we pull the monoid monad through the set
monad via a distributive map is this. First, the monoid monad
furnishes the universe, $L$, as the set of expressions generated by
the grammar. We'll denote this by $L(m)$. Then, we enrich the set of
formulae by the operations of the monoid \emph{acting on sets}.

\begin{mathpar}
  \inferrule* [lab=expression] {} {{c,d} ::=}
  \and
  \inferrule* [lab=identity verity] {} {true}
  \and
  \inferrule* [lab=negation] {} {\;| \; \neg c}
  \and
  \inferrule* [lab=conjunction] {} {\;| \; c \& d}
  \and
  \inferrule* [lab=identity verity] {} {\bf{e}}
  \and
  \inferrule* [lab=negation] {} {\;| \; \bf{g_1} \; | \; ... \; | \; \bf{g_n}}
  \and
  \inferrule* [lab=conjunction] {} {\;| \; c * d}
\end{mathpar} 

The identity element, $e$ and the generators of the monoid, $g_1$,
..., $g_n$, can be considered $0$-ary operations in the same way that
we usually consider constants as $0$-ary operations. To avoid
confusion between these elements and the \emph{logical formulae} that
pick them out of the crowd, we write the logical formulae in
$\bf{boldface}$.

Now, we can write our distributive map. Surprisingly, it is exactly a
meaning for our logic!

\begin{mathpar}
  \inferrule* {} {\meaningof{true} = L(m)}
  \and
  \inferrule* {} {\meaningof{\neg c} = L(m) \backslash c}
  \and 
  \inferrule* {} {\meaningof{c \& d} = \meaningof{c} \cap \meaningof{d}}
  \and
  \inferrule* {} {\meaningof{\bf{e}} = \{ m \; \in \; L(m) \; | \; m \equiv e \}}
  \and
  \inferrule* {} {\meaningof{\bf{g_i}} = \{ m \; \in \; L(m) \; | \; m \equiv g_i \}}
  \and
  \inferrule* {} {\meaningof{c*d} = \{ m \; \in \; L(m) \; | \; m \equiv m_1 * m_2, m_1 \; \in \; \meaningof{c}, m_2 \; \in \; \meaningof{d} \}}
\end{mathpar}

\paragraph{Primes: an application}
Before going any further, let's look at an example of how to use these
new operators. Suppose we wanted to pick out all the elements of the
monoid that were not expressible as a composition of other
elements. Obviously, for monoids with a finite set of generators, this
is exactly just the generators, so we could write $\bf{g_1} || ... ||
\bf{g_n}$\footnote{We get the disjunction, $||$, by the usual DeMorgan
  translation: $c || d \stackrel{def}{=} \neg( \neg c \& \neg
  d)$}. However, when the set of generators is not finite, as it is
when the monoid is the integers under multiplication, we need another
way to write this down. That's where our other operators come in
handy. A moment's thought suggests that we could say that since $true$
denotes any possible element in the monoid, an element is not a
composition using negation plus our composition formula, i.e. $\neg
(true * true)$. This is a little overkill, however. We just want to
eliminate non-trivial compositions. We know how to express the
identity element, that's $\bf{e}$, so we are interested in those
elements that are not the identity, i.e. $\neg \bf{e}$. Then a formula
that eliminates compositions of non-trivial elements is spelled out
$\neg (\neg e * \neg e)$. Finally, we want to eliminate the identity
as a solution. So, we arrive at $\neg (\neg e * \neg e) \& \neg
e$. There, that formula picks out the \emph{primes} of \emph{any}
monoid.

\paragraph{Summary}

What have we done? We've illustrated a specific distributive map, one
that pulls the set monad through the monoid monad. We've shown that
this particular distributive map coincides with giving a semantics to
a particular logic, one whose structure is derived solely from the
shape of the collection monad, i.e. set, and the shape of the term
language, in this case monoid.

\subsubsection{Iterating the design pattern}

The whole point of working in this manner is that by virtue of its
compositional structure it provides a much higher level of abstraction
and greater opportunities for reuse. To illustrate the point, we will
now iterate the construction using our toy language, the
$lambda$-calculus, as the term language. As we saw in chapter 1, the
$lambda$-calculus also has a generators and relations
presentation. Unlike a monoid, however, the lambda calculus has
another piece of machinery: reduction! In addition to structural
equivalence of terms (which is a bi-directional relation) there is the
$beta$-reduction rule that captures the \emph{behavioral} aspect of
the lambda calculus.

It is key to understand this underlying structure of language
definitions. In essence, when a DSL is purely about structure it is
presented entirely in terms of generators (read a grammar) and
relations (like the monoid laws). When the DSL is also about behavior,
i.e. the terms in the language somehow express some kind of
computation, then the language has a third component, some kind of
reduction relation. \footnote{In some sense this is one of the central
  contributions of the theory of computation back to
  mathematics. Algebraists have known for a long time about generators
  and relations presentations of algebraic structures (of which
  algebraic data types are a subset). This collective wisdom is
  studied, for example, in the field of universal
  algebra. Computational models like the $lambda$-calculus and more
  recently the process calculi, like Milner's $\pi$-calculus or
  Cardelli and Gordon's ambient calculus, take this presentation one
  step further and add a set of conditional rewrite rules to express
  the computational content of the model. It was Milner who first
  recognized this particular decomposition of language definitions in
  his seminal paper, Functions as Processes, where he reformulated the
  presentation $\pi$-calculus along these lines.} This organization,
this common factoring of the specification of a language, makes it
possible to factor code that handles a wide range of semantic
features. The logic we derive below provides a great example.

\section{Searching for programs}

TBD

% \section{Existence problems}
% We begin with some metamathematics.
% All problems about the existence of maps can be cast into one of the
% following two forms, which are in a sense mutually dual.

% \noindent
% {\bf The Extension Problem}\index{extension problem} \    %%% NB index entry tag
% Given an inclusion $ A \stackrel{i}{\hookrightarrow} X $, and a map
% $ A \stackrel{f}{\rightarrow} Y $,
% does there exist a map $f^{\dagger}:X\to Y$ such that
% $f^{\dagger}$ agrees with $f$ on $A$?

% Here the appropriate source category for maps should be clear from the
% context and, moreover, commutativity through a
% candidate $f^{\dagger}$ is precisely
% the restriction requirement; that is,
% $$f^{\dagger}   :  f^{\dagger}\circ i = f^{\dagger}|_A = f\,. $$
% If such an $f^{\dagger}$ exists\footnote{${}^{\dagger}$ suggests striving
% for perfection, crusading}, then it is called an {\bf
% extension}\index{extension!of a map|bi} of $f$ and is said to {\bf
% extend}\index{extend|bi} $f$. In any diagrams, the presence of
% a dotted arrow or an arrow carrying a ? indicates a pious hope, in no way
% begging the question of its existence. Note that we shall usually
% omit $\circ$ from composite maps.

% \noindent
% {\bf The Lifting Problem}\index{lifting problem} \
% Given a pair of maps $E \stackrel{p}{\rightarrow}B$ and $X \stackrel{f}
% {\rightarrow} B $,
% does there exist a map $f^{\circ} : X \to E$, with
% $pf^{\circ} = f  $?


% That {\em all\/} existence problems about maps are essentially of one
% type or
% the other from these two is seen as follows. Evidently, all existence problems
% are representable by triangular diagrams\index{triangular diagrams} and it
% is easily seen that there are only these six possibilities:
% \begin{center}\begin{picture}(300,70)  %augch2 75
% \put(5,60){\vector(1,0){30}}
% \put(55,60){\vector(1,0){30}}
% \put(135,60){\vector(-1,0){30}}
% \put(185,60){\vector(-1,0){30}}
% \put(235,60){\vector(-1,0){30}}
% \put(285,60){\vector(-1,0){30}}
% \put(0,55){\vector(0,-1){30}}
% \put(50,55){\vector(0,-1){30}}
% \put(100,25){\vector(0,1){30}}
% \put(150,25){\vector(0,1){30}}
% \put(200,55){\vector(0,-1){30}}
% \put(250,55){\vector(0,-1){30}}
% \put(28,33){\small ?}
% \put(78,33){\small ?}
% \put(128,33){\small ?}
% \put(178,33){\small ?}
% \put(228,33){\small ?}
% \put(278,33){\small ?}
% \put(10,3){\bf 1}
% \put(60,3){\bf 2}
% \put(110,3){\bf 3}
% \put(160,3){\bf 4}
% \put(210,3){\bf 5}
% \put(260,3){\bf 6}
% \put(35,55){\vector(-1,-1){30}}
% \put(155,25){\vector(1,1){30}}
% \put(135,55){\vector(-1,-1){30}}
% \put(55,25){\vector(1,1){30}}
% \put(235,55){\vector(-1,-1){30}}
% \put(255,25){\vector(1,1){30}}
% \end{picture}\end{center}



% \begin{figure}
% \begin{picture}(300,220)(0,0)
% \put(-20,-20){\resizebox{20 cm}{!}{\includegraphics{3dpdf}}}
% \put(260,-10){\resizebox{15 cm}{!}{\includegraphics{contpdf}}}
% \put(220,80){$\beta$}
% \put(400,-10){$N$}
% \put(260,170){$\beta$}
% \put(90,15){$N$}
% \end{picture}
% \caption{{\em The log-gamma family of densities with central mean
% $<N> \, = \frac{1}{2}$ as a surface and as a contour plot. }}
% \label{pdf}
% \end{figure}

\newpage

%ch.tex


\chapter{The semantic web}
\begin{center}
{\small\em Where are we; how did we get here; and where are we going?}
\end{center}

TBD
\section{How our web framework enables different kinds of application queries}

\subsubsection{An alternative presentation}

If you recall, there's an alternative way to present monads that are
algebras, like our monoid monad. Algebras are presented in terms of
generators and relations. In our case the generators presentation is
really just a grammar for monoid expressions.

\begin{mathpar}
  \inferrule* [lab=expression] {} {{m,n} ::=}
  \and
  \inferrule* [lab=identity element] {} {e}
  \and
  \inferrule* [lab=generators] {} {\;| \; g_1 \; | \; ... \; | \; g_n}
  \and
  \inferrule* [lab=monoid-multiplication] {} {\;| \; m * n}
\end{mathpar} 

This is subject to the following constraints, meaning that we will
treat syntactic expressions of certain forms as denoting the same
element of the monoid. To emphasize the nearly purely syntactic role
of these constraints we will use a different symbol for the
constraints. We also use the same symbol, $\equiv$, for the smallest equivalence
relation respecting these constraints.

\begin{mathpar}
  \inferrule* [lab=identity laws] {} {m * e \equiv m \equiv e * m}
  \and
  \inferrule* [lab=associativity] {} {m_1 * (m_2 * m_3) \equiv (m_1 * m_2) * m_3}
\end{mathpar} 

\paragraph{Logic: the set monad as an algebra}
In a similar manner, there is a language associated with the monad of
sets \emph{considered as an algebra}. This language is very familiar
to most programmers.

\begin{mathpar}
  \inferrule* [lab=expression] {} {{c,d} ::=}
  \and
  \inferrule* [lab=identity verity] {} {true}
  \and
  \inferrule* [lab=negation] {} {\;| \; \neg c}
  \and
  \inferrule* [lab=conjunction] {} {\;| \; c \& d}
\end{mathpar} 

Now, if we had a specific set in hand, say $L$ (which we'll call a
universe in the sequel), we can interpret the expressions in the this
language, aka formulae, in terms of operations on subsets of that
set. As with our compiler for the concrete syntax of the
$lambda$-calculus in chapter 1, we can express this translation very
compactly as

\begin{mathpar}
  \inferrule* {} {\meaningof{true} = L}
  \and
  \inferrule* {} {\meaningof{\neg c} = L \backslash c}
  \and 
  \inferrule* {} {\meaningof{c \& d} = \meaningof{c} \cap \meaningof{d}}
\end{mathpar}

Now, what's happening when we pull the monoid monad through the set
monad via a distributive map is this. First, the monoid monad
furnishes the universe, $L$, as the set of expressions generated by
the grammar. We'll denote this by $L(m)$. Then, we enrich the set of
formulae by the operations of the monoid \emph{acting on sets}.

\begin{mathpar}
  \inferrule* [lab=expression] {} {{c,d} ::=}
  \and
  \inferrule* [lab=identity verity] {} {true}
  \and
  \inferrule* [lab=negation] {} {\;| \; \neg c}
  \and
  \inferrule* [lab=conjunction] {} {\;| \; c \& d}
  \and
  \inferrule* [lab=identity verity] {} {\bf{e}}
  \and
  \inferrule* [lab=negation] {} {\;| \; \bf{g_1} \; | \; ... \; | \; \bf{g_n}}
  \and
  \inferrule* [lab=conjunction] {} {\;| \; c * d}
\end{mathpar} 

The identity element, $e$ and the generators of the monoid, $g_1$,
..., $g_n$, can be considered $0$-ary operations in the same way that
we usually consider constants as $0$-ary operations. To avoid
confusion between these elements and the \emph{logical formulae} that
pick them out of the crowd, we write the logical formulae in
$\bf{boldface}$.

Now, we can write our distributive map. Surprisingly, it is exactly a
meaning for our logic!

\begin{mathpar}
  \inferrule* {} {\meaningof{true} = L(m)}
  \and
  \inferrule* {} {\meaningof{\neg c} = L(m) \backslash c}
  \and 
  \inferrule* {} {\meaningof{c \& d} = \meaningof{c} \cap \meaningof{d}}
  \and
  \inferrule* {} {\meaningof{\bf{e}} = \{ m \; \in \; L(m) \; | \; m \equiv e \}}
  \and
  \inferrule* {} {\meaningof{\bf{g_i}} = \{ m \; \in \; L(m) \; | \; m \equiv g_i \}}
  \and
  \inferrule* {} {\meaningof{c*d} = \{ m \; \in \; L(m) \; | \; m \equiv m_1 * m_2, m_1 \; \in \; \meaningof{c}, m_2 \; \in \; \meaningof{d} \}}
\end{mathpar}

\paragraph{Primes: an application}
Before going any further, let's look at an example of how to use these
new operators. Suppose we wanted to pick out all the elements of the
monoid that were not expressible as a composition of other
elements. Obviously, for monoids with a finite set of generators, this
is exactly just the generators, so we could write $\bf{g_1} || ... ||
\bf{g_n}$\footnote{We get the disjunction, $||$, by the usual DeMorgan
  translation: $c || d \stackrel{def}{=} \neg( \neg c \& \neg
  d)$}. However, when the set of generators is not finite, as it is
when the monoid is the integers under multiplication, we need another
way to write this down. That's where our other operators come in
handy. A moment's thought suggests that we could say that since $true$
denotes any possible element in the monoid, an element is not a
composition using negation plus our composition formula, i.e. $\neg
(true * true)$. This is a little overkill, however. We just want to
eliminate non-trivial compositions. We know how to express the
identity element, that's $\bf{e}$, so we are interested in those
elements that are not the identity, i.e. $\neg \bf{e}$. Then a formula
that eliminates compositions of non-trivial elements is spelled out
$\neg (\neg e * \neg e)$. Finally, we want to eliminate the identity
as a solution. So, we arrive at $\neg (\neg e * \neg e) \& \neg
e$. There, that formula picks out the \emph{primes} of \emph{any}
monoid.

\paragraph{Summary}

What have we done? We've illustrated a specific distributive map, one
that pulls the set monad through the monoid monad. We've shown that
this particular distributive map coincides with giving a semantics to
a particular logic, one whose structure is derived solely from the
shape of the collection monad, i.e. set, and the shape of the term
language, in this case monoid.

\subsubsection{Iterating the design pattern}

The whole point of working in this manner is that by virtue of its
compositional structure it provides a much higher level of abstraction
and greater opportunities for reuse. To illustrate the point, we will
now iterate the construction using our toy language, the
$lambda$-calculus, as the term language. As we saw in chapter 1, the
$lambda$-calculus also has a generators and relations
presentation. Unlike a monoid, however, the lambda calculus has
another piece of machinery: reduction! In addition to structural
equivalence of terms (which is a bi-directional relation) there is the
$beta$-reduction rule that captures the \emph{behavioral} aspect of
the lambda calculus.

It is key to understand this underlying structure of language
definitions. In essence, when a DSL is purely about structure it is
presented entirely in terms of generators (read a grammar) and
relations (like the monoid laws). When the DSL is also about behavior,
i.e. the terms in the language somehow express some kind of
computation, then the language has a third component, some kind of
reduction relation. \footnote{In some sense this is one of the central
  contributions of the theory of computation back to
  mathematics. Algebraists have known for a long time about generators
  and relations presentations of algebraic structures (of which
  algebraic data types are a subset). This collective wisdom is
  studied, for example, in the field of universal
  algebra. Computational models like the $lambda$-calculus and more
  recently the process calculi, like Milner's $\pi$-calculus or
  Cardelli and Gordon's ambient calculus, take this presentation one
  step further and add a set of conditional rewrite rules to express
  the computational content of the model. It was Milner who first
  recognized this particular decomposition of language definitions in
  his seminal paper, Functions as Processes, where he reformulated the
  presentation $\pi$-calculus along these lines.} This organization,
this common factoring of the specification of a language, makes it
possible to factor code that handles a wide range of semantic
features. The logic we derive below provides a great example.

\section{Searching for programs}

TBD

% \section{Existence problems}
% We begin with some metamathematics.
% All problems about the existence of maps can be cast into one of the
% following two forms, which are in a sense mutually dual.

% \noindent
% {\bf The Extension Problem}\index{extension problem} \    %%% NB index entry tag
% Given an inclusion $ A \stackrel{i}{\hookrightarrow} X $, and a map
% $ A \stackrel{f}{\rightarrow} Y $,
% does there exist a map $f^{\dagger}:X\to Y$ such that
% $f^{\dagger}$ agrees with $f$ on $A$?

% Here the appropriate source category for maps should be clear from the
% context and, moreover, commutativity through a
% candidate $f^{\dagger}$ is precisely
% the restriction requirement; that is,
% $$f^{\dagger}   :  f^{\dagger}\circ i = f^{\dagger}|_A = f\,. $$
% If such an $f^{\dagger}$ exists\footnote{${}^{\dagger}$ suggests striving
% for perfection, crusading}, then it is called an {\bf
% extension}\index{extension!of a map|bi} of $f$ and is said to {\bf
% extend}\index{extend|bi} $f$. In any diagrams, the presence of
% a dotted arrow or an arrow carrying a ? indicates a pious hope, in no way
% begging the question of its existence. Note that we shall usually
% omit $\circ$ from composite maps.

% \noindent
% {\bf The Lifting Problem}\index{lifting problem} \
% Given a pair of maps $E \stackrel{p}{\rightarrow}B$ and $X \stackrel{f}
% {\rightarrow} B $,
% does there exist a map $f^{\circ} : X \to E$, with
% $pf^{\circ} = f  $?


% That {\em all\/} existence problems about maps are essentially of one
% type or
% the other from these two is seen as follows. Evidently, all existence problems
% are representable by triangular diagrams\index{triangular diagrams} and it
% is easily seen that there are only these six possibilities:
% \begin{center}\begin{picture}(300,70)  %augch2 75
% \put(5,60){\vector(1,0){30}}
% \put(55,60){\vector(1,0){30}}
% \put(135,60){\vector(-1,0){30}}
% \put(185,60){\vector(-1,0){30}}
% \put(235,60){\vector(-1,0){30}}
% \put(285,60){\vector(-1,0){30}}
% \put(0,55){\vector(0,-1){30}}
% \put(50,55){\vector(0,-1){30}}
% \put(100,25){\vector(0,1){30}}
% \put(150,25){\vector(0,1){30}}
% \put(200,55){\vector(0,-1){30}}
% \put(250,55){\vector(0,-1){30}}
% \put(28,33){\small ?}
% \put(78,33){\small ?}
% \put(128,33){\small ?}
% \put(178,33){\small ?}
% \put(228,33){\small ?}
% \put(278,33){\small ?}
% \put(10,3){\bf 1}
% \put(60,3){\bf 2}
% \put(110,3){\bf 3}
% \put(160,3){\bf 4}
% \put(210,3){\bf 5}
% \put(260,3){\bf 6}
% \put(35,55){\vector(-1,-1){30}}
% \put(155,25){\vector(1,1){30}}
% \put(135,55){\vector(-1,-1){30}}
% \put(55,25){\vector(1,1){30}}
% \put(235,55){\vector(-1,-1){30}}
% \put(255,25){\vector(1,1){30}}
% \end{picture}\end{center}



% \begin{figure}
% \begin{picture}(300,220)(0,0)
% \put(-20,-20){\resizebox{20 cm}{!}{\includegraphics{3dpdf}}}
% \put(260,-10){\resizebox{15 cm}{!}{\includegraphics{contpdf}}}
% \put(220,80){$\beta$}
% \put(400,-10){$N$}
% \put(260,170){$\beta$}
% \put(90,15){$N$}
% \end{picture}
% \caption{{\em The log-gamma family of densities with central mean
% $<N> \, = \frac{1}{2}$ as a surface and as a contour plot. }}
% \label{pdf}
% \end{figure}

\newpage

%ch.tex


\chapter{The semantic web}
\begin{center}
{\small\em Where are we; how did we get here; and where are we going?}
\end{center}

TBD
\section{How our web framework enables different kinds of application queries}

\subsubsection{An alternative presentation}

If you recall, there's an alternative way to present monads that are
algebras, like our monoid monad. Algebras are presented in terms of
generators and relations. In our case the generators presentation is
really just a grammar for monoid expressions.

\begin{mathpar}
  \inferrule* [lab=expression] {} {{m,n} ::=}
  \and
  \inferrule* [lab=identity element] {} {e}
  \and
  \inferrule* [lab=generators] {} {\;| \; g_1 \; | \; ... \; | \; g_n}
  \and
  \inferrule* [lab=monoid-multiplication] {} {\;| \; m * n}
\end{mathpar} 

This is subject to the following constraints, meaning that we will
treat syntactic expressions of certain forms as denoting the same
element of the monoid. To emphasize the nearly purely syntactic role
of these constraints we will use a different symbol for the
constraints. We also use the same symbol, $\equiv$, for the smallest equivalence
relation respecting these constraints.

\begin{mathpar}
  \inferrule* [lab=identity laws] {} {m * e \equiv m \equiv e * m}
  \and
  \inferrule* [lab=associativity] {} {m_1 * (m_2 * m_3) \equiv (m_1 * m_2) * m_3}
\end{mathpar} 

\paragraph{Logic: the set monad as an algebra}
In a similar manner, there is a language associated with the monad of
sets \emph{considered as an algebra}. This language is very familiar
to most programmers.

\begin{mathpar}
  \inferrule* [lab=expression] {} {{c,d} ::=}
  \and
  \inferrule* [lab=identity verity] {} {true}
  \and
  \inferrule* [lab=negation] {} {\;| \; \neg c}
  \and
  \inferrule* [lab=conjunction] {} {\;| \; c \& d}
\end{mathpar} 

Now, if we had a specific set in hand, say $L$ (which we'll call a
universe in the sequel), we can interpret the expressions in the this
language, aka formulae, in terms of operations on subsets of that
set. As with our compiler for the concrete syntax of the
$lambda$-calculus in chapter 1, we can express this translation very
compactly as

\begin{mathpar}
  \inferrule* {} {\meaningof{true} = L}
  \and
  \inferrule* {} {\meaningof{\neg c} = L \backslash c}
  \and 
  \inferrule* {} {\meaningof{c \& d} = \meaningof{c} \cap \meaningof{d}}
\end{mathpar}

Now, what's happening when we pull the monoid monad through the set
monad via a distributive map is this. First, the monoid monad
furnishes the universe, $L$, as the set of expressions generated by
the grammar. We'll denote this by $L(m)$. Then, we enrich the set of
formulae by the operations of the monoid \emph{acting on sets}.

\begin{mathpar}
  \inferrule* [lab=expression] {} {{c,d} ::=}
  \and
  \inferrule* [lab=identity verity] {} {true}
  \and
  \inferrule* [lab=negation] {} {\;| \; \neg c}
  \and
  \inferrule* [lab=conjunction] {} {\;| \; c \& d}
  \and
  \inferrule* [lab=identity verity] {} {\bf{e}}
  \and
  \inferrule* [lab=negation] {} {\;| \; \bf{g_1} \; | \; ... \; | \; \bf{g_n}}
  \and
  \inferrule* [lab=conjunction] {} {\;| \; c * d}
\end{mathpar} 

The identity element, $e$ and the generators of the monoid, $g_1$,
..., $g_n$, can be considered $0$-ary operations in the same way that
we usually consider constants as $0$-ary operations. To avoid
confusion between these elements and the \emph{logical formulae} that
pick them out of the crowd, we write the logical formulae in
$\bf{boldface}$.

Now, we can write our distributive map. Surprisingly, it is exactly a
meaning for our logic!

\begin{mathpar}
  \inferrule* {} {\meaningof{true} = L(m)}
  \and
  \inferrule* {} {\meaningof{\neg c} = L(m) \backslash c}
  \and 
  \inferrule* {} {\meaningof{c \& d} = \meaningof{c} \cap \meaningof{d}}
  \and
  \inferrule* {} {\meaningof{\bf{e}} = \{ m \; \in \; L(m) \; | \; m \equiv e \}}
  \and
  \inferrule* {} {\meaningof{\bf{g_i}} = \{ m \; \in \; L(m) \; | \; m \equiv g_i \}}
  \and
  \inferrule* {} {\meaningof{c*d} = \{ m \; \in \; L(m) \; | \; m \equiv m_1 * m_2, m_1 \; \in \; \meaningof{c}, m_2 \; \in \; \meaningof{d} \}}
\end{mathpar}

\paragraph{Primes: an application}
Before going any further, let's look at an example of how to use these
new operators. Suppose we wanted to pick out all the elements of the
monoid that were not expressible as a composition of other
elements. Obviously, for monoids with a finite set of generators, this
is exactly just the generators, so we could write $\bf{g_1} || ... ||
\bf{g_n}$\footnote{We get the disjunction, $||$, by the usual DeMorgan
  translation: $c || d \stackrel{def}{=} \neg( \neg c \& \neg
  d)$}. However, when the set of generators is not finite, as it is
when the monoid is the integers under multiplication, we need another
way to write this down. That's where our other operators come in
handy. A moment's thought suggests that we could say that since $true$
denotes any possible element in the monoid, an element is not a
composition using negation plus our composition formula, i.e. $\neg
(true * true)$. This is a little overkill, however. We just want to
eliminate non-trivial compositions. We know how to express the
identity element, that's $\bf{e}$, so we are interested in those
elements that are not the identity, i.e. $\neg \bf{e}$. Then a formula
that eliminates compositions of non-trivial elements is spelled out
$\neg (\neg e * \neg e)$. Finally, we want to eliminate the identity
as a solution. So, we arrive at $\neg (\neg e * \neg e) \& \neg
e$. There, that formula picks out the \emph{primes} of \emph{any}
monoid.

\paragraph{Summary}

What have we done? We've illustrated a specific distributive map, one
that pulls the set monad through the monoid monad. We've shown that
this particular distributive map coincides with giving a semantics to
a particular logic, one whose structure is derived solely from the
shape of the collection monad, i.e. set, and the shape of the term
language, in this case monoid.

\subsubsection{Iterating the design pattern}

The whole point of working in this manner is that by virtue of its
compositional structure it provides a much higher level of abstraction
and greater opportunities for reuse. To illustrate the point, we will
now iterate the construction using our toy language, the
$lambda$-calculus, as the term language. As we saw in chapter 1, the
$lambda$-calculus also has a generators and relations
presentation. Unlike a monoid, however, the lambda calculus has
another piece of machinery: reduction! In addition to structural
equivalence of terms (which is a bi-directional relation) there is the
$beta$-reduction rule that captures the \emph{behavioral} aspect of
the lambda calculus.

It is key to understand this underlying structure of language
definitions. In essence, when a DSL is purely about structure it is
presented entirely in terms of generators (read a grammar) and
relations (like the monoid laws). When the DSL is also about behavior,
i.e. the terms in the language somehow express some kind of
computation, then the language has a third component, some kind of
reduction relation. \footnote{In some sense this is one of the central
  contributions of the theory of computation back to
  mathematics. Algebraists have known for a long time about generators
  and relations presentations of algebraic structures (of which
  algebraic data types are a subset). This collective wisdom is
  studied, for example, in the field of universal
  algebra. Computational models like the $lambda$-calculus and more
  recently the process calculi, like Milner's $\pi$-calculus or
  Cardelli and Gordon's ambient calculus, take this presentation one
  step further and add a set of conditional rewrite rules to express
  the computational content of the model. It was Milner who first
  recognized this particular decomposition of language definitions in
  his seminal paper, Functions as Processes, where he reformulated the
  presentation $\pi$-calculus along these lines.} This organization,
this common factoring of the specification of a language, makes it
possible to factor code that handles a wide range of semantic
features. The logic we derive below provides a great example.

\section{Searching for programs}

TBD

% \section{Existence problems}
% We begin with some metamathematics.
% All problems about the existence of maps can be cast into one of the
% following two forms, which are in a sense mutually dual.

% \noindent
% {\bf The Extension Problem}\index{extension problem} \    %%% NB index entry tag
% Given an inclusion $ A \stackrel{i}{\hookrightarrow} X $, and a map
% $ A \stackrel{f}{\rightarrow} Y $,
% does there exist a map $f^{\dagger}:X\to Y$ such that
% $f^{\dagger}$ agrees with $f$ on $A$?

% Here the appropriate source category for maps should be clear from the
% context and, moreover, commutativity through a
% candidate $f^{\dagger}$ is precisely
% the restriction requirement; that is,
% $$f^{\dagger}   :  f^{\dagger}\circ i = f^{\dagger}|_A = f\,. $$
% If such an $f^{\dagger}$ exists\footnote{${}^{\dagger}$ suggests striving
% for perfection, crusading}, then it is called an {\bf
% extension}\index{extension!of a map|bi} of $f$ and is said to {\bf
% extend}\index{extend|bi} $f$. In any diagrams, the presence of
% a dotted arrow or an arrow carrying a ? indicates a pious hope, in no way
% begging the question of its existence. Note that we shall usually
% omit $\circ$ from composite maps.

% \noindent
% {\bf The Lifting Problem}\index{lifting problem} \
% Given a pair of maps $E \stackrel{p}{\rightarrow}B$ and $X \stackrel{f}
% {\rightarrow} B $,
% does there exist a map $f^{\circ} : X \to E$, with
% $pf^{\circ} = f  $?


% That {\em all\/} existence problems about maps are essentially of one
% type or
% the other from these two is seen as follows. Evidently, all existence problems
% are representable by triangular diagrams\index{triangular diagrams} and it
% is easily seen that there are only these six possibilities:
% \begin{center}\begin{picture}(300,70)  %augch2 75
% \put(5,60){\vector(1,0){30}}
% \put(55,60){\vector(1,0){30}}
% \put(135,60){\vector(-1,0){30}}
% \put(185,60){\vector(-1,0){30}}
% \put(235,60){\vector(-1,0){30}}
% \put(285,60){\vector(-1,0){30}}
% \put(0,55){\vector(0,-1){30}}
% \put(50,55){\vector(0,-1){30}}
% \put(100,25){\vector(0,1){30}}
% \put(150,25){\vector(0,1){30}}
% \put(200,55){\vector(0,-1){30}}
% \put(250,55){\vector(0,-1){30}}
% \put(28,33){\small ?}
% \put(78,33){\small ?}
% \put(128,33){\small ?}
% \put(178,33){\small ?}
% \put(228,33){\small ?}
% \put(278,33){\small ?}
% \put(10,3){\bf 1}
% \put(60,3){\bf 2}
% \put(110,3){\bf 3}
% \put(160,3){\bf 4}
% \put(210,3){\bf 5}
% \put(260,3){\bf 6}
% \put(35,55){\vector(-1,-1){30}}
% \put(155,25){\vector(1,1){30}}
% \put(135,55){\vector(-1,-1){30}}
% \put(55,25){\vector(1,1){30}}
% \put(235,55){\vector(-1,-1){30}}
% \put(255,25){\vector(1,1){30}}
% \end{picture}\end{center}



% \begin{figure}
% \begin{picture}(300,220)(0,0)
% \put(-20,-20){\resizebox{20 cm}{!}{\includegraphics{3dpdf}}}
% \put(260,-10){\resizebox{15 cm}{!}{\includegraphics{contpdf}}}
% \put(220,80){$\beta$}
% \put(400,-10){$N$}
% \put(260,170){$\beta$}
% \put(90,15){$N$}
% \end{picture}
% \caption{{\em The log-gamma family of densities with central mean
% $<N> \, = \frac{1}{2}$ as a surface and as a contour plot. }}
% \label{pdf}
% \end{figure}

\newpage

%ch.tex


\chapter{The semantic web}
\begin{center}
{\small\em Where are we; how did we get here; and where are we going?}
\end{center}

TBD
\section{How our web framework enables different kinds of application queries}

\subsubsection{An alternative presentation}

If you recall, there's an alternative way to present monads that are
algebras, like our monoid monad. Algebras are presented in terms of
generators and relations. In our case the generators presentation is
really just a grammar for monoid expressions.

\begin{mathpar}
  \inferrule* [lab=expression] {} {{m,n} ::=}
  \and
  \inferrule* [lab=identity element] {} {e}
  \and
  \inferrule* [lab=generators] {} {\;| \; g_1 \; | \; ... \; | \; g_n}
  \and
  \inferrule* [lab=monoid-multiplication] {} {\;| \; m * n}
\end{mathpar} 

This is subject to the following constraints, meaning that we will
treat syntactic expressions of certain forms as denoting the same
element of the monoid. To emphasize the nearly purely syntactic role
of these constraints we will use a different symbol for the
constraints. We also use the same symbol, $\equiv$, for the smallest equivalence
relation respecting these constraints.

\begin{mathpar}
  \inferrule* [lab=identity laws] {} {m * e \equiv m \equiv e * m}
  \and
  \inferrule* [lab=associativity] {} {m_1 * (m_2 * m_3) \equiv (m_1 * m_2) * m_3}
\end{mathpar} 

\paragraph{Logic: the set monad as an algebra}
In a similar manner, there is a language associated with the monad of
sets \emph{considered as an algebra}. This language is very familiar
to most programmers.

\begin{mathpar}
  \inferrule* [lab=expression] {} {{c,d} ::=}
  \and
  \inferrule* [lab=identity verity] {} {true}
  \and
  \inferrule* [lab=negation] {} {\;| \; \neg c}
  \and
  \inferrule* [lab=conjunction] {} {\;| \; c \& d}
\end{mathpar} 

Now, if we had a specific set in hand, say $L$ (which we'll call a
universe in the sequel), we can interpret the expressions in the this
language, aka formulae, in terms of operations on subsets of that
set. As with our compiler for the concrete syntax of the
$lambda$-calculus in chapter 1, we can express this translation very
compactly as

\begin{mathpar}
  \inferrule* {} {\meaningof{true} = L}
  \and
  \inferrule* {} {\meaningof{\neg c} = L \backslash c}
  \and 
  \inferrule* {} {\meaningof{c \& d} = \meaningof{c} \cap \meaningof{d}}
\end{mathpar}

Now, what's happening when we pull the monoid monad through the set
monad via a distributive map is this. First, the monoid monad
furnishes the universe, $L$, as the set of expressions generated by
the grammar. We'll denote this by $L(m)$. Then, we enrich the set of
formulae by the operations of the monoid \emph{acting on sets}.

\begin{mathpar}
  \inferrule* [lab=expression] {} {{c,d} ::=}
  \and
  \inferrule* [lab=identity verity] {} {true}
  \and
  \inferrule* [lab=negation] {} {\;| \; \neg c}
  \and
  \inferrule* [lab=conjunction] {} {\;| \; c \& d}
  \and
  \inferrule* [lab=identity verity] {} {\bf{e}}
  \and
  \inferrule* [lab=negation] {} {\;| \; \bf{g_1} \; | \; ... \; | \; \bf{g_n}}
  \and
  \inferrule* [lab=conjunction] {} {\;| \; c * d}
\end{mathpar} 

The identity element, $e$ and the generators of the monoid, $g_1$,
..., $g_n$, can be considered $0$-ary operations in the same way that
we usually consider constants as $0$-ary operations. To avoid
confusion between these elements and the \emph{logical formulae} that
pick them out of the crowd, we write the logical formulae in
$\bf{boldface}$.

Now, we can write our distributive map. Surprisingly, it is exactly a
meaning for our logic!

\begin{mathpar}
  \inferrule* {} {\meaningof{true} = L(m)}
  \and
  \inferrule* {} {\meaningof{\neg c} = L(m) \backslash c}
  \and 
  \inferrule* {} {\meaningof{c \& d} = \meaningof{c} \cap \meaningof{d}}
  \and
  \inferrule* {} {\meaningof{\bf{e}} = \{ m \; \in \; L(m) \; | \; m \equiv e \}}
  \and
  \inferrule* {} {\meaningof{\bf{g_i}} = \{ m \; \in \; L(m) \; | \; m \equiv g_i \}}
  \and
  \inferrule* {} {\meaningof{c*d} = \{ m \; \in \; L(m) \; | \; m \equiv m_1 * m_2, m_1 \; \in \; \meaningof{c}, m_2 \; \in \; \meaningof{d} \}}
\end{mathpar}

\paragraph{Primes: an application}
Before going any further, let's look at an example of how to use these
new operators. Suppose we wanted to pick out all the elements of the
monoid that were not expressible as a composition of other
elements. Obviously, for monoids with a finite set of generators, this
is exactly just the generators, so we could write $\bf{g_1} || ... ||
\bf{g_n}$\footnote{We get the disjunction, $||$, by the usual DeMorgan
  translation: $c || d \stackrel{def}{=} \neg( \neg c \& \neg
  d)$}. However, when the set of generators is not finite, as it is
when the monoid is the integers under multiplication, we need another
way to write this down. That's where our other operators come in
handy. A moment's thought suggests that we could say that since $true$
denotes any possible element in the monoid, an element is not a
composition using negation plus our composition formula, i.e. $\neg
(true * true)$. This is a little overkill, however. We just want to
eliminate non-trivial compositions. We know how to express the
identity element, that's $\bf{e}$, so we are interested in those
elements that are not the identity, i.e. $\neg \bf{e}$. Then a formula
that eliminates compositions of non-trivial elements is spelled out
$\neg (\neg e * \neg e)$. Finally, we want to eliminate the identity
as a solution. So, we arrive at $\neg (\neg e * \neg e) \& \neg
e$. There, that formula picks out the \emph{primes} of \emph{any}
monoid.

\paragraph{Summary}

What have we done? We've illustrated a specific distributive map, one
that pulls the set monad through the monoid monad. We've shown that
this particular distributive map coincides with giving a semantics to
a particular logic, one whose structure is derived solely from the
shape of the collection monad, i.e. set, and the shape of the term
language, in this case monoid.

\subsubsection{Iterating the design pattern}

The whole point of working in this manner is that by virtue of its
compositional structure it provides a much higher level of abstraction
and greater opportunities for reuse. To illustrate the point, we will
now iterate the construction using our toy language, the
$lambda$-calculus, as the term language. As we saw in chapter 1, the
$lambda$-calculus also has a generators and relations
presentation. Unlike a monoid, however, the lambda calculus has
another piece of machinery: reduction! In addition to structural
equivalence of terms (which is a bi-directional relation) there is the
$beta$-reduction rule that captures the \emph{behavioral} aspect of
the lambda calculus.

It is key to understand this underlying structure of language
definitions. In essence, when a DSL is purely about structure it is
presented entirely in terms of generators (read a grammar) and
relations (like the monoid laws). When the DSL is also about behavior,
i.e. the terms in the language somehow express some kind of
computation, then the language has a third component, some kind of
reduction relation. \footnote{In some sense this is one of the central
  contributions of the theory of computation back to
  mathematics. Algebraists have known for a long time about generators
  and relations presentations of algebraic structures (of which
  algebraic data types are a subset). This collective wisdom is
  studied, for example, in the field of universal
  algebra. Computational models like the $lambda$-calculus and more
  recently the process calculi, like Milner's $\pi$-calculus or
  Cardelli and Gordon's ambient calculus, take this presentation one
  step further and add a set of conditional rewrite rules to express
  the computational content of the model. It was Milner who first
  recognized this particular decomposition of language definitions in
  his seminal paper, Functions as Processes, where he reformulated the
  presentation $\pi$-calculus along these lines.} This organization,
this common factoring of the specification of a language, makes it
possible to factor code that handles a wide range of semantic
features. The logic we derive below provides a great example.

\section{Searching for programs}

TBD

% \section{Existence problems}
% We begin with some metamathematics.
% All problems about the existence of maps can be cast into one of the
% following two forms, which are in a sense mutually dual.

% \noindent
% {\bf The Extension Problem}\index{extension problem} \    %%% NB index entry tag
% Given an inclusion $ A \stackrel{i}{\hookrightarrow} X $, and a map
% $ A \stackrel{f}{\rightarrow} Y $,
% does there exist a map $f^{\dagger}:X\to Y$ such that
% $f^{\dagger}$ agrees with $f$ on $A$?

% Here the appropriate source category for maps should be clear from the
% context and, moreover, commutativity through a
% candidate $f^{\dagger}$ is precisely
% the restriction requirement; that is,
% $$f^{\dagger}   :  f^{\dagger}\circ i = f^{\dagger}|_A = f\,. $$
% If such an $f^{\dagger}$ exists\footnote{${}^{\dagger}$ suggests striving
% for perfection, crusading}, then it is called an {\bf
% extension}\index{extension!of a map|bi} of $f$ and is said to {\bf
% extend}\index{extend|bi} $f$. In any diagrams, the presence of
% a dotted arrow or an arrow carrying a ? indicates a pious hope, in no way
% begging the question of its existence. Note that we shall usually
% omit $\circ$ from composite maps.

% \noindent
% {\bf The Lifting Problem}\index{lifting problem} \
% Given a pair of maps $E \stackrel{p}{\rightarrow}B$ and $X \stackrel{f}
% {\rightarrow} B $,
% does there exist a map $f^{\circ} : X \to E$, with
% $pf^{\circ} = f  $?


% That {\em all\/} existence problems about maps are essentially of one
% type or
% the other from these two is seen as follows. Evidently, all existence problems
% are representable by triangular diagrams\index{triangular diagrams} and it
% is easily seen that there are only these six possibilities:
% \begin{center}\begin{picture}(300,70)  %augch2 75
% \put(5,60){\vector(1,0){30}}
% \put(55,60){\vector(1,0){30}}
% \put(135,60){\vector(-1,0){30}}
% \put(185,60){\vector(-1,0){30}}
% \put(235,60){\vector(-1,0){30}}
% \put(285,60){\vector(-1,0){30}}
% \put(0,55){\vector(0,-1){30}}
% \put(50,55){\vector(0,-1){30}}
% \put(100,25){\vector(0,1){30}}
% \put(150,25){\vector(0,1){30}}
% \put(200,55){\vector(0,-1){30}}
% \put(250,55){\vector(0,-1){30}}
% \put(28,33){\small ?}
% \put(78,33){\small ?}
% \put(128,33){\small ?}
% \put(178,33){\small ?}
% \put(228,33){\small ?}
% \put(278,33){\small ?}
% \put(10,3){\bf 1}
% \put(60,3){\bf 2}
% \put(110,3){\bf 3}
% \put(160,3){\bf 4}
% \put(210,3){\bf 5}
% \put(260,3){\bf 6}
% \put(35,55){\vector(-1,-1){30}}
% \put(155,25){\vector(1,1){30}}
% \put(135,55){\vector(-1,-1){30}}
% \put(55,25){\vector(1,1){30}}
% \put(235,55){\vector(-1,-1){30}}
% \put(255,25){\vector(1,1){30}}
% \end{picture}\end{center}



% \begin{figure}
% \begin{picture}(300,220)(0,0)
% \put(-20,-20){\resizebox{20 cm}{!}{\includegraphics{3dpdf}}}
% \put(260,-10){\resizebox{15 cm}{!}{\includegraphics{contpdf}}}
% \put(220,80){$\beta$}
% \put(400,-10){$N$}
% \put(260,170){$\beta$}
% \put(90,15){$N$}
% \end{picture}
% \caption{{\em The log-gamma family of densities with central mean
% $<N> \, = \frac{1}{2}$ as a surface and as a contour plot. }}
% \label{pdf}
% \end{figure}

\newpage

%ch.tex


\chapter{The semantic web}
\begin{center}
{\small\em Where are we; how did we get here; and where are we going?}
\end{center}

TBD
\section{How our web framework enables different kinds of application queries}

\subsubsection{An alternative presentation}

If you recall, there's an alternative way to present monads that are
algebras, like our monoid monad. Algebras are presented in terms of
generators and relations. In our case the generators presentation is
really just a grammar for monoid expressions.

\begin{mathpar}
  \inferrule* [lab=expression] {} {{m,n} ::=}
  \and
  \inferrule* [lab=identity element] {} {e}
  \and
  \inferrule* [lab=generators] {} {\;| \; g_1 \; | \; ... \; | \; g_n}
  \and
  \inferrule* [lab=monoid-multiplication] {} {\;| \; m * n}
\end{mathpar} 

This is subject to the following constraints, meaning that we will
treat syntactic expressions of certain forms as denoting the same
element of the monoid. To emphasize the nearly purely syntactic role
of these constraints we will use a different symbol for the
constraints. We also use the same symbol, $\equiv$, for the smallest equivalence
relation respecting these constraints.

\begin{mathpar}
  \inferrule* [lab=identity laws] {} {m * e \equiv m \equiv e * m}
  \and
  \inferrule* [lab=associativity] {} {m_1 * (m_2 * m_3) \equiv (m_1 * m_2) * m_3}
\end{mathpar} 

\paragraph{Logic: the set monad as an algebra}
In a similar manner, there is a language associated with the monad of
sets \emph{considered as an algebra}. This language is very familiar
to most programmers.

\begin{mathpar}
  \inferrule* [lab=expression] {} {{c,d} ::=}
  \and
  \inferrule* [lab=identity verity] {} {true}
  \and
  \inferrule* [lab=negation] {} {\;| \; \neg c}
  \and
  \inferrule* [lab=conjunction] {} {\;| \; c \& d}
\end{mathpar} 

Now, if we had a specific set in hand, say $L$ (which we'll call a
universe in the sequel), we can interpret the expressions in the this
language, aka formulae, in terms of operations on subsets of that
set. As with our compiler for the concrete syntax of the
$lambda$-calculus in chapter 1, we can express this translation very
compactly as

\begin{mathpar}
  \inferrule* {} {\meaningof{true} = L}
  \and
  \inferrule* {} {\meaningof{\neg c} = L \backslash c}
  \and 
  \inferrule* {} {\meaningof{c \& d} = \meaningof{c} \cap \meaningof{d}}
\end{mathpar}

Now, what's happening when we pull the monoid monad through the set
monad via a distributive map is this. First, the monoid monad
furnishes the universe, $L$, as the set of expressions generated by
the grammar. We'll denote this by $L(m)$. Then, we enrich the set of
formulae by the operations of the monoid \emph{acting on sets}.

\begin{mathpar}
  \inferrule* [lab=expression] {} {{c,d} ::=}
  \and
  \inferrule* [lab=identity verity] {} {true}
  \and
  \inferrule* [lab=negation] {} {\;| \; \neg c}
  \and
  \inferrule* [lab=conjunction] {} {\;| \; c \& d}
  \and
  \inferrule* [lab=identity verity] {} {\bf{e}}
  \and
  \inferrule* [lab=negation] {} {\;| \; \bf{g_1} \; | \; ... \; | \; \bf{g_n}}
  \and
  \inferrule* [lab=conjunction] {} {\;| \; c * d}
\end{mathpar} 

The identity element, $e$ and the generators of the monoid, $g_1$,
..., $g_n$, can be considered $0$-ary operations in the same way that
we usually consider constants as $0$-ary operations. To avoid
confusion between these elements and the \emph{logical formulae} that
pick them out of the crowd, we write the logical formulae in
$\bf{boldface}$.

Now, we can write our distributive map. Surprisingly, it is exactly a
meaning for our logic!

\begin{mathpar}
  \inferrule* {} {\meaningof{true} = L(m)}
  \and
  \inferrule* {} {\meaningof{\neg c} = L(m) \backslash c}
  \and 
  \inferrule* {} {\meaningof{c \& d} = \meaningof{c} \cap \meaningof{d}}
  \and
  \inferrule* {} {\meaningof{\bf{e}} = \{ m \; \in \; L(m) \; | \; m \equiv e \}}
  \and
  \inferrule* {} {\meaningof{\bf{g_i}} = \{ m \; \in \; L(m) \; | \; m \equiv g_i \}}
  \and
  \inferrule* {} {\meaningof{c*d} = \{ m \; \in \; L(m) \; | \; m \equiv m_1 * m_2, m_1 \; \in \; \meaningof{c}, m_2 \; \in \; \meaningof{d} \}}
\end{mathpar}

\paragraph{Primes: an application}
Before going any further, let's look at an example of how to use these
new operators. Suppose we wanted to pick out all the elements of the
monoid that were not expressible as a composition of other
elements. Obviously, for monoids with a finite set of generators, this
is exactly just the generators, so we could write $\bf{g_1} || ... ||
\bf{g_n}$\footnote{We get the disjunction, $||$, by the usual DeMorgan
  translation: $c || d \stackrel{def}{=} \neg( \neg c \& \neg
  d)$}. However, when the set of generators is not finite, as it is
when the monoid is the integers under multiplication, we need another
way to write this down. That's where our other operators come in
handy. A moment's thought suggests that we could say that since $true$
denotes any possible element in the monoid, an element is not a
composition using negation plus our composition formula, i.e. $\neg
(true * true)$. This is a little overkill, however. We just want to
eliminate non-trivial compositions. We know how to express the
identity element, that's $\bf{e}$, so we are interested in those
elements that are not the identity, i.e. $\neg \bf{e}$. Then a formula
that eliminates compositions of non-trivial elements is spelled out
$\neg (\neg e * \neg e)$. Finally, we want to eliminate the identity
as a solution. So, we arrive at $\neg (\neg e * \neg e) \& \neg
e$. There, that formula picks out the \emph{primes} of \emph{any}
monoid.

\paragraph{Summary}

What have we done? We've illustrated a specific distributive map, one
that pulls the set monad through the monoid monad. We've shown that
this particular distributive map coincides with giving a semantics to
a particular logic, one whose structure is derived solely from the
shape of the collection monad, i.e. set, and the shape of the term
language, in this case monoid.

\subsubsection{Iterating the design pattern}

The whole point of working in this manner is that by virtue of its
compositional structure it provides a much higher level of abstraction
and greater opportunities for reuse. To illustrate the point, we will
now iterate the construction using our toy language, the
$lambda$-calculus, as the term language. As we saw in chapter 1, the
$lambda$-calculus also has a generators and relations
presentation. Unlike a monoid, however, the lambda calculus has
another piece of machinery: reduction! In addition to structural
equivalence of terms (which is a bi-directional relation) there is the
$beta$-reduction rule that captures the \emph{behavioral} aspect of
the lambda calculus.

It is key to understand this underlying structure of language
definitions. In essence, when a DSL is purely about structure it is
presented entirely in terms of generators (read a grammar) and
relations (like the monoid laws). When the DSL is also about behavior,
i.e. the terms in the language somehow express some kind of
computation, then the language has a third component, some kind of
reduction relation. \footnote{In some sense this is one of the central
  contributions of the theory of computation back to
  mathematics. Algebraists have known for a long time about generators
  and relations presentations of algebraic structures (of which
  algebraic data types are a subset). This collective wisdom is
  studied, for example, in the field of universal
  algebra. Computational models like the $lambda$-calculus and more
  recently the process calculi, like Milner's $\pi$-calculus or
  Cardelli and Gordon's ambient calculus, take this presentation one
  step further and add a set of conditional rewrite rules to express
  the computational content of the model. It was Milner who first
  recognized this particular decomposition of language definitions in
  his seminal paper, Functions as Processes, where he reformulated the
  presentation $\pi$-calculus along these lines.} This organization,
this common factoring of the specification of a language, makes it
possible to factor code that handles a wide range of semantic
features. The logic we derive below provides a great example.

\section{Searching for programs}

TBD

% \section{Existence problems}
% We begin with some metamathematics.
% All problems about the existence of maps can be cast into one of the
% following two forms, which are in a sense mutually dual.

% \noindent
% {\bf The Extension Problem}\index{extension problem} \    %%% NB index entry tag
% Given an inclusion $ A \stackrel{i}{\hookrightarrow} X $, and a map
% $ A \stackrel{f}{\rightarrow} Y $,
% does there exist a map $f^{\dagger}:X\to Y$ such that
% $f^{\dagger}$ agrees with $f$ on $A$?

% Here the appropriate source category for maps should be clear from the
% context and, moreover, commutativity through a
% candidate $f^{\dagger}$ is precisely
% the restriction requirement; that is,
% $$f^{\dagger}   :  f^{\dagger}\circ i = f^{\dagger}|_A = f\,. $$
% If such an $f^{\dagger}$ exists\footnote{${}^{\dagger}$ suggests striving
% for perfection, crusading}, then it is called an {\bf
% extension}\index{extension!of a map|bi} of $f$ and is said to {\bf
% extend}\index{extend|bi} $f$. In any diagrams, the presence of
% a dotted arrow or an arrow carrying a ? indicates a pious hope, in no way
% begging the question of its existence. Note that we shall usually
% omit $\circ$ from composite maps.

% \noindent
% {\bf The Lifting Problem}\index{lifting problem} \
% Given a pair of maps $E \stackrel{p}{\rightarrow}B$ and $X \stackrel{f}
% {\rightarrow} B $,
% does there exist a map $f^{\circ} : X \to E$, with
% $pf^{\circ} = f  $?


% That {\em all\/} existence problems about maps are essentially of one
% type or
% the other from these two is seen as follows. Evidently, all existence problems
% are representable by triangular diagrams\index{triangular diagrams} and it
% is easily seen that there are only these six possibilities:
% \begin{center}\begin{picture}(300,70)  %augch2 75
% \put(5,60){\vector(1,0){30}}
% \put(55,60){\vector(1,0){30}}
% \put(135,60){\vector(-1,0){30}}
% \put(185,60){\vector(-1,0){30}}
% \put(235,60){\vector(-1,0){30}}
% \put(285,60){\vector(-1,0){30}}
% \put(0,55){\vector(0,-1){30}}
% \put(50,55){\vector(0,-1){30}}
% \put(100,25){\vector(0,1){30}}
% \put(150,25){\vector(0,1){30}}
% \put(200,55){\vector(0,-1){30}}
% \put(250,55){\vector(0,-1){30}}
% \put(28,33){\small ?}
% \put(78,33){\small ?}
% \put(128,33){\small ?}
% \put(178,33){\small ?}
% \put(228,33){\small ?}
% \put(278,33){\small ?}
% \put(10,3){\bf 1}
% \put(60,3){\bf 2}
% \put(110,3){\bf 3}
% \put(160,3){\bf 4}
% \put(210,3){\bf 5}
% \put(260,3){\bf 6}
% \put(35,55){\vector(-1,-1){30}}
% \put(155,25){\vector(1,1){30}}
% \put(135,55){\vector(-1,-1){30}}
% \put(55,25){\vector(1,1){30}}
% \put(235,55){\vector(-1,-1){30}}
% \put(255,25){\vector(1,1){30}}
% \end{picture}\end{center}



% \begin{figure}
% \begin{picture}(300,220)(0,0)
% \put(-20,-20){\resizebox{20 cm}{!}{\includegraphics{3dpdf}}}
% \put(260,-10){\resizebox{15 cm}{!}{\includegraphics{contpdf}}}
% \put(220,80){$\beta$}
% \put(400,-10){$N$}
% \put(260,170){$\beta$}
% \put(90,15){$N$}
% \end{picture}
% \caption{{\em The log-gamma family of densities with central mean
% $<N> \, = \frac{1}{2}$ as a surface and as a contour plot. }}
% \label{pdf}
% \end{figure}

\newpage

%ch.tex


\chapter{The semantic web}
\begin{center}
{\small\em Where are we; how did we get here; and where are we going?}
\end{center}

TBD
\section{How our web framework enables different kinds of application queries}

\subsubsection{An alternative presentation}

If you recall, there's an alternative way to present monads that are
algebras, like our monoid monad. Algebras are presented in terms of
generators and relations. In our case the generators presentation is
really just a grammar for monoid expressions.

\begin{mathpar}
  \inferrule* [lab=expression] {} {{m,n} ::=}
  \and
  \inferrule* [lab=identity element] {} {e}
  \and
  \inferrule* [lab=generators] {} {\;| \; g_1 \; | \; ... \; | \; g_n}
  \and
  \inferrule* [lab=monoid-multiplication] {} {\;| \; m * n}
\end{mathpar} 

This is subject to the following constraints, meaning that we will
treat syntactic expressions of certain forms as denoting the same
element of the monoid. To emphasize the nearly purely syntactic role
of these constraints we will use a different symbol for the
constraints. We also use the same symbol, $\equiv$, for the smallest equivalence
relation respecting these constraints.

\begin{mathpar}
  \inferrule* [lab=identity laws] {} {m * e \equiv m \equiv e * m}
  \and
  \inferrule* [lab=associativity] {} {m_1 * (m_2 * m_3) \equiv (m_1 * m_2) * m_3}
\end{mathpar} 

\paragraph{Logic: the set monad as an algebra}
In a similar manner, there is a language associated with the monad of
sets \emph{considered as an algebra}. This language is very familiar
to most programmers.

\begin{mathpar}
  \inferrule* [lab=expression] {} {{c,d} ::=}
  \and
  \inferrule* [lab=identity verity] {} {true}
  \and
  \inferrule* [lab=negation] {} {\;| \; \neg c}
  \and
  \inferrule* [lab=conjunction] {} {\;| \; c \& d}
\end{mathpar} 

Now, if we had a specific set in hand, say $L$ (which we'll call a
universe in the sequel), we can interpret the expressions in the this
language, aka formulae, in terms of operations on subsets of that
set. As with our compiler for the concrete syntax of the
$lambda$-calculus in chapter 1, we can express this translation very
compactly as

\begin{mathpar}
  \inferrule* {} {\meaningof{true} = L}
  \and
  \inferrule* {} {\meaningof{\neg c} = L \backslash c}
  \and 
  \inferrule* {} {\meaningof{c \& d} = \meaningof{c} \cap \meaningof{d}}
\end{mathpar}

Now, what's happening when we pull the monoid monad through the set
monad via a distributive map is this. First, the monoid monad
furnishes the universe, $L$, as the set of expressions generated by
the grammar. We'll denote this by $L(m)$. Then, we enrich the set of
formulae by the operations of the monoid \emph{acting on sets}.

\begin{mathpar}
  \inferrule* [lab=expression] {} {{c,d} ::=}
  \and
  \inferrule* [lab=identity verity] {} {true}
  \and
  \inferrule* [lab=negation] {} {\;| \; \neg c}
  \and
  \inferrule* [lab=conjunction] {} {\;| \; c \& d}
  \and
  \inferrule* [lab=identity verity] {} {\bf{e}}
  \and
  \inferrule* [lab=negation] {} {\;| \; \bf{g_1} \; | \; ... \; | \; \bf{g_n}}
  \and
  \inferrule* [lab=conjunction] {} {\;| \; c * d}
\end{mathpar} 

The identity element, $e$ and the generators of the monoid, $g_1$,
..., $g_n$, can be considered $0$-ary operations in the same way that
we usually consider constants as $0$-ary operations. To avoid
confusion between these elements and the \emph{logical formulae} that
pick them out of the crowd, we write the logical formulae in
$\bf{boldface}$.

Now, we can write our distributive map. Surprisingly, it is exactly a
meaning for our logic!

\begin{mathpar}
  \inferrule* {} {\meaningof{true} = L(m)}
  \and
  \inferrule* {} {\meaningof{\neg c} = L(m) \backslash c}
  \and 
  \inferrule* {} {\meaningof{c \& d} = \meaningof{c} \cap \meaningof{d}}
  \and
  \inferrule* {} {\meaningof{\bf{e}} = \{ m \; \in \; L(m) \; | \; m \equiv e \}}
  \and
  \inferrule* {} {\meaningof{\bf{g_i}} = \{ m \; \in \; L(m) \; | \; m \equiv g_i \}}
  \and
  \inferrule* {} {\meaningof{c*d} = \{ m \; \in \; L(m) \; | \; m \equiv m_1 * m_2, m_1 \; \in \; \meaningof{c}, m_2 \; \in \; \meaningof{d} \}}
\end{mathpar}

\paragraph{Primes: an application}
Before going any further, let's look at an example of how to use these
new operators. Suppose we wanted to pick out all the elements of the
monoid that were not expressible as a composition of other
elements. Obviously, for monoids with a finite set of generators, this
is exactly just the generators, so we could write $\bf{g_1} || ... ||
\bf{g_n}$\footnote{We get the disjunction, $||$, by the usual DeMorgan
  translation: $c || d \stackrel{def}{=} \neg( \neg c \& \neg
  d)$}. However, when the set of generators is not finite, as it is
when the monoid is the integers under multiplication, we need another
way to write this down. That's where our other operators come in
handy. A moment's thought suggests that we could say that since $true$
denotes any possible element in the monoid, an element is not a
composition using negation plus our composition formula, i.e. $\neg
(true * true)$. This is a little overkill, however. We just want to
eliminate non-trivial compositions. We know how to express the
identity element, that's $\bf{e}$, so we are interested in those
elements that are not the identity, i.e. $\neg \bf{e}$. Then a formula
that eliminates compositions of non-trivial elements is spelled out
$\neg (\neg e * \neg e)$. Finally, we want to eliminate the identity
as a solution. So, we arrive at $\neg (\neg e * \neg e) \& \neg
e$. There, that formula picks out the \emph{primes} of \emph{any}
monoid.

\paragraph{Summary}

What have we done? We've illustrated a specific distributive map, one
that pulls the set monad through the monoid monad. We've shown that
this particular distributive map coincides with giving a semantics to
a particular logic, one whose structure is derived solely from the
shape of the collection monad, i.e. set, and the shape of the term
language, in this case monoid.

\subsubsection{Iterating the design pattern}

The whole point of working in this manner is that by virtue of its
compositional structure it provides a much higher level of abstraction
and greater opportunities for reuse. To illustrate the point, we will
now iterate the construction using our toy language, the
$lambda$-calculus, as the term language. As we saw in chapter 1, the
$lambda$-calculus also has a generators and relations
presentation. Unlike a monoid, however, the lambda calculus has
another piece of machinery: reduction! In addition to structural
equivalence of terms (which is a bi-directional relation) there is the
$beta$-reduction rule that captures the \emph{behavioral} aspect of
the lambda calculus.

It is key to understand this underlying structure of language
definitions. In essence, when a DSL is purely about structure it is
presented entirely in terms of generators (read a grammar) and
relations (like the monoid laws). When the DSL is also about behavior,
i.e. the terms in the language somehow express some kind of
computation, then the language has a third component, some kind of
reduction relation. \footnote{In some sense this is one of the central
  contributions of the theory of computation back to
  mathematics. Algebraists have known for a long time about generators
  and relations presentations of algebraic structures (of which
  algebraic data types are a subset). This collective wisdom is
  studied, for example, in the field of universal
  algebra. Computational models like the $lambda$-calculus and more
  recently the process calculi, like Milner's $\pi$-calculus or
  Cardelli and Gordon's ambient calculus, take this presentation one
  step further and add a set of conditional rewrite rules to express
  the computational content of the model. It was Milner who first
  recognized this particular decomposition of language definitions in
  his seminal paper, Functions as Processes, where he reformulated the
  presentation $\pi$-calculus along these lines.} This organization,
this common factoring of the specification of a language, makes it
possible to factor code that handles a wide range of semantic
features. The logic we derive below provides a great example.

\section{Searching for programs}

TBD

% \section{Existence problems}
% We begin with some metamathematics.
% All problems about the existence of maps can be cast into one of the
% following two forms, which are in a sense mutually dual.

% \noindent
% {\bf The Extension Problem}\index{extension problem} \    %%% NB index entry tag
% Given an inclusion $ A \stackrel{i}{\hookrightarrow} X $, and a map
% $ A \stackrel{f}{\rightarrow} Y $,
% does there exist a map $f^{\dagger}:X\to Y$ such that
% $f^{\dagger}$ agrees with $f$ on $A$?

% Here the appropriate source category for maps should be clear from the
% context and, moreover, commutativity through a
% candidate $f^{\dagger}$ is precisely
% the restriction requirement; that is,
% $$f^{\dagger}   :  f^{\dagger}\circ i = f^{\dagger}|_A = f\,. $$
% If such an $f^{\dagger}$ exists\footnote{${}^{\dagger}$ suggests striving
% for perfection, crusading}, then it is called an {\bf
% extension}\index{extension!of a map|bi} of $f$ and is said to {\bf
% extend}\index{extend|bi} $f$. In any diagrams, the presence of
% a dotted arrow or an arrow carrying a ? indicates a pious hope, in no way
% begging the question of its existence. Note that we shall usually
% omit $\circ$ from composite maps.

% \noindent
% {\bf The Lifting Problem}\index{lifting problem} \
% Given a pair of maps $E \stackrel{p}{\rightarrow}B$ and $X \stackrel{f}
% {\rightarrow} B $,
% does there exist a map $f^{\circ} : X \to E$, with
% $pf^{\circ} = f  $?


% That {\em all\/} existence problems about maps are essentially of one
% type or
% the other from these two is seen as follows. Evidently, all existence problems
% are representable by triangular diagrams\index{triangular diagrams} and it
% is easily seen that there are only these six possibilities:
% \begin{center}\begin{picture}(300,70)  %augch2 75
% \put(5,60){\vector(1,0){30}}
% \put(55,60){\vector(1,0){30}}
% \put(135,60){\vector(-1,0){30}}
% \put(185,60){\vector(-1,0){30}}
% \put(235,60){\vector(-1,0){30}}
% \put(285,60){\vector(-1,0){30}}
% \put(0,55){\vector(0,-1){30}}
% \put(50,55){\vector(0,-1){30}}
% \put(100,25){\vector(0,1){30}}
% \put(150,25){\vector(0,1){30}}
% \put(200,55){\vector(0,-1){30}}
% \put(250,55){\vector(0,-1){30}}
% \put(28,33){\small ?}
% \put(78,33){\small ?}
% \put(128,33){\small ?}
% \put(178,33){\small ?}
% \put(228,33){\small ?}
% \put(278,33){\small ?}
% \put(10,3){\bf 1}
% \put(60,3){\bf 2}
% \put(110,3){\bf 3}
% \put(160,3){\bf 4}
% \put(210,3){\bf 5}
% \put(260,3){\bf 6}
% \put(35,55){\vector(-1,-1){30}}
% \put(155,25){\vector(1,1){30}}
% \put(135,55){\vector(-1,-1){30}}
% \put(55,25){\vector(1,1){30}}
% \put(235,55){\vector(-1,-1){30}}
% \put(255,25){\vector(1,1){30}}
% \end{picture}\end{center}



% \begin{figure}
% \begin{picture}(300,220)(0,0)
% \put(-20,-20){\resizebox{20 cm}{!}{\includegraphics{3dpdf}}}
% \put(260,-10){\resizebox{15 cm}{!}{\includegraphics{contpdf}}}
% \put(220,80){$\beta$}
% \put(400,-10){$N$}
% \put(260,170){$\beta$}
% \put(90,15){$N$}
% \end{picture}
% \caption{{\em The log-gamma family of densities with central mean
% $<N> \, = \frac{1}{2}$ as a surface and as a contour plot. }}
% \label{pdf}
% \end{figure}

\newpage

%ch.tex


\chapter{The semantic web}
\begin{center}
{\small\em Where are we; how did we get here; and where are we going?}
\end{center}

TBD
\section{How our web framework enables different kinds of application queries}

\subsubsection{An alternative presentation}

If you recall, there's an alternative way to present monads that are
algebras, like our monoid monad. Algebras are presented in terms of
generators and relations. In our case the generators presentation is
really just a grammar for monoid expressions.

\begin{mathpar}
  \inferrule* [lab=expression] {} {{m,n} ::=}
  \and
  \inferrule* [lab=identity element] {} {e}
  \and
  \inferrule* [lab=generators] {} {\;| \; g_1 \; | \; ... \; | \; g_n}
  \and
  \inferrule* [lab=monoid-multiplication] {} {\;| \; m * n}
\end{mathpar} 

This is subject to the following constraints, meaning that we will
treat syntactic expressions of certain forms as denoting the same
element of the monoid. To emphasize the nearly purely syntactic role
of these constraints we will use a different symbol for the
constraints. We also use the same symbol, $\equiv$, for the smallest equivalence
relation respecting these constraints.

\begin{mathpar}
  \inferrule* [lab=identity laws] {} {m * e \equiv m \equiv e * m}
  \and
  \inferrule* [lab=associativity] {} {m_1 * (m_2 * m_3) \equiv (m_1 * m_2) * m_3}
\end{mathpar} 

\paragraph{Logic: the set monad as an algebra}
In a similar manner, there is a language associated with the monad of
sets \emph{considered as an algebra}. This language is very familiar
to most programmers.

\begin{mathpar}
  \inferrule* [lab=expression] {} {{c,d} ::=}
  \and
  \inferrule* [lab=identity verity] {} {true}
  \and
  \inferrule* [lab=negation] {} {\;| \; \neg c}
  \and
  \inferrule* [lab=conjunction] {} {\;| \; c \& d}
\end{mathpar} 

Now, if we had a specific set in hand, say $L$ (which we'll call a
universe in the sequel), we can interpret the expressions in the this
language, aka formulae, in terms of operations on subsets of that
set. As with our compiler for the concrete syntax of the
$lambda$-calculus in chapter 1, we can express this translation very
compactly as

\begin{mathpar}
  \inferrule* {} {\meaningof{true} = L}
  \and
  \inferrule* {} {\meaningof{\neg c} = L \backslash c}
  \and 
  \inferrule* {} {\meaningof{c \& d} = \meaningof{c} \cap \meaningof{d}}
\end{mathpar}

Now, what's happening when we pull the monoid monad through the set
monad via a distributive map is this. First, the monoid monad
furnishes the universe, $L$, as the set of expressions generated by
the grammar. We'll denote this by $L(m)$. Then, we enrich the set of
formulae by the operations of the monoid \emph{acting on sets}.

\begin{mathpar}
  \inferrule* [lab=expression] {} {{c,d} ::=}
  \and
  \inferrule* [lab=identity verity] {} {true}
  \and
  \inferrule* [lab=negation] {} {\;| \; \neg c}
  \and
  \inferrule* [lab=conjunction] {} {\;| \; c \& d}
  \and
  \inferrule* [lab=identity verity] {} {\bf{e}}
  \and
  \inferrule* [lab=negation] {} {\;| \; \bf{g_1} \; | \; ... \; | \; \bf{g_n}}
  \and
  \inferrule* [lab=conjunction] {} {\;| \; c * d}
\end{mathpar} 

The identity element, $e$ and the generators of the monoid, $g_1$,
..., $g_n$, can be considered $0$-ary operations in the same way that
we usually consider constants as $0$-ary operations. To avoid
confusion between these elements and the \emph{logical formulae} that
pick them out of the crowd, we write the logical formulae in
$\bf{boldface}$.

Now, we can write our distributive map. Surprisingly, it is exactly a
meaning for our logic!

\begin{mathpar}
  \inferrule* {} {\meaningof{true} = L(m)}
  \and
  \inferrule* {} {\meaningof{\neg c} = L(m) \backslash c}
  \and 
  \inferrule* {} {\meaningof{c \& d} = \meaningof{c} \cap \meaningof{d}}
  \and
  \inferrule* {} {\meaningof{\bf{e}} = \{ m \; \in \; L(m) \; | \; m \equiv e \}}
  \and
  \inferrule* {} {\meaningof{\bf{g_i}} = \{ m \; \in \; L(m) \; | \; m \equiv g_i \}}
  \and
  \inferrule* {} {\meaningof{c*d} = \{ m \; \in \; L(m) \; | \; m \equiv m_1 * m_2, m_1 \; \in \; \meaningof{c}, m_2 \; \in \; \meaningof{d} \}}
\end{mathpar}

\paragraph{Primes: an application}
Before going any further, let's look at an example of how to use these
new operators. Suppose we wanted to pick out all the elements of the
monoid that were not expressible as a composition of other
elements. Obviously, for monoids with a finite set of generators, this
is exactly just the generators, so we could write $\bf{g_1} || ... ||
\bf{g_n}$\footnote{We get the disjunction, $||$, by the usual DeMorgan
  translation: $c || d \stackrel{def}{=} \neg( \neg c \& \neg
  d)$}. However, when the set of generators is not finite, as it is
when the monoid is the integers under multiplication, we need another
way to write this down. That's where our other operators come in
handy. A moment's thought suggests that we could say that since $true$
denotes any possible element in the monoid, an element is not a
composition using negation plus our composition formula, i.e. $\neg
(true * true)$. This is a little overkill, however. We just want to
eliminate non-trivial compositions. We know how to express the
identity element, that's $\bf{e}$, so we are interested in those
elements that are not the identity, i.e. $\neg \bf{e}$. Then a formula
that eliminates compositions of non-trivial elements is spelled out
$\neg (\neg e * \neg e)$. Finally, we want to eliminate the identity
as a solution. So, we arrive at $\neg (\neg e * \neg e) \& \neg
e$. There, that formula picks out the \emph{primes} of \emph{any}
monoid.

\paragraph{Summary}

What have we done? We've illustrated a specific distributive map, one
that pulls the set monad through the monoid monad. We've shown that
this particular distributive map coincides with giving a semantics to
a particular logic, one whose structure is derived solely from the
shape of the collection monad, i.e. set, and the shape of the term
language, in this case monoid.

\subsubsection{Iterating the design pattern}

The whole point of working in this manner is that by virtue of its
compositional structure it provides a much higher level of abstraction
and greater opportunities for reuse. To illustrate the point, we will
now iterate the construction using our toy language, the
$lambda$-calculus, as the term language. As we saw in chapter 1, the
$lambda$-calculus also has a generators and relations
presentation. Unlike a monoid, however, the lambda calculus has
another piece of machinery: reduction! In addition to structural
equivalence of terms (which is a bi-directional relation) there is the
$beta$-reduction rule that captures the \emph{behavioral} aspect of
the lambda calculus.

It is key to understand this underlying structure of language
definitions. In essence, when a DSL is purely about structure it is
presented entirely in terms of generators (read a grammar) and
relations (like the monoid laws). When the DSL is also about behavior,
i.e. the terms in the language somehow express some kind of
computation, then the language has a third component, some kind of
reduction relation. \footnote{In some sense this is one of the central
  contributions of the theory of computation back to
  mathematics. Algebraists have known for a long time about generators
  and relations presentations of algebraic structures (of which
  algebraic data types are a subset). This collective wisdom is
  studied, for example, in the field of universal
  algebra. Computational models like the $lambda$-calculus and more
  recently the process calculi, like Milner's $\pi$-calculus or
  Cardelli and Gordon's ambient calculus, take this presentation one
  step further and add a set of conditional rewrite rules to express
  the computational content of the model. It was Milner who first
  recognized this particular decomposition of language definitions in
  his seminal paper, Functions as Processes, where he reformulated the
  presentation $\pi$-calculus along these lines.} This organization,
this common factoring of the specification of a language, makes it
possible to factor code that handles a wide range of semantic
features. The logic we derive below provides a great example.

\section{Searching for programs}

TBD

% \section{Existence problems}
% We begin with some metamathematics.
% All problems about the existence of maps can be cast into one of the
% following two forms, which are in a sense mutually dual.

% \noindent
% {\bf The Extension Problem}\index{extension problem} \    %%% NB index entry tag
% Given an inclusion $ A \stackrel{i}{\hookrightarrow} X $, and a map
% $ A \stackrel{f}{\rightarrow} Y $,
% does there exist a map $f^{\dagger}:X\to Y$ such that
% $f^{\dagger}$ agrees with $f$ on $A$?

% Here the appropriate source category for maps should be clear from the
% context and, moreover, commutativity through a
% candidate $f^{\dagger}$ is precisely
% the restriction requirement; that is,
% $$f^{\dagger}   :  f^{\dagger}\circ i = f^{\dagger}|_A = f\,. $$
% If such an $f^{\dagger}$ exists\footnote{${}^{\dagger}$ suggests striving
% for perfection, crusading}, then it is called an {\bf
% extension}\index{extension!of a map|bi} of $f$ and is said to {\bf
% extend}\index{extend|bi} $f$. In any diagrams, the presence of
% a dotted arrow or an arrow carrying a ? indicates a pious hope, in no way
% begging the question of its existence. Note that we shall usually
% omit $\circ$ from composite maps.

% \noindent
% {\bf The Lifting Problem}\index{lifting problem} \
% Given a pair of maps $E \stackrel{p}{\rightarrow}B$ and $X \stackrel{f}
% {\rightarrow} B $,
% does there exist a map $f^{\circ} : X \to E$, with
% $pf^{\circ} = f  $?


% That {\em all\/} existence problems about maps are essentially of one
% type or
% the other from these two is seen as follows. Evidently, all existence problems
% are representable by triangular diagrams\index{triangular diagrams} and it
% is easily seen that there are only these six possibilities:
% \begin{center}\begin{picture}(300,70)  %augch2 75
% \put(5,60){\vector(1,0){30}}
% \put(55,60){\vector(1,0){30}}
% \put(135,60){\vector(-1,0){30}}
% \put(185,60){\vector(-1,0){30}}
% \put(235,60){\vector(-1,0){30}}
% \put(285,60){\vector(-1,0){30}}
% \put(0,55){\vector(0,-1){30}}
% \put(50,55){\vector(0,-1){30}}
% \put(100,25){\vector(0,1){30}}
% \put(150,25){\vector(0,1){30}}
% \put(200,55){\vector(0,-1){30}}
% \put(250,55){\vector(0,-1){30}}
% \put(28,33){\small ?}
% \put(78,33){\small ?}
% \put(128,33){\small ?}
% \put(178,33){\small ?}
% \put(228,33){\small ?}
% \put(278,33){\small ?}
% \put(10,3){\bf 1}
% \put(60,3){\bf 2}
% \put(110,3){\bf 3}
% \put(160,3){\bf 4}
% \put(210,3){\bf 5}
% \put(260,3){\bf 6}
% \put(35,55){\vector(-1,-1){30}}
% \put(155,25){\vector(1,1){30}}
% \put(135,55){\vector(-1,-1){30}}
% \put(55,25){\vector(1,1){30}}
% \put(235,55){\vector(-1,-1){30}}
% \put(255,25){\vector(1,1){30}}
% \end{picture}\end{center}



% \begin{figure}
% \begin{picture}(300,220)(0,0)
% \put(-20,-20){\resizebox{20 cm}{!}{\includegraphics{3dpdf}}}
% \put(260,-10){\resizebox{15 cm}{!}{\includegraphics{contpdf}}}
% \put(220,80){$\beta$}
% \put(400,-10){$N$}
% \put(260,170){$\beta$}
% \put(90,15){$N$}
% \end{picture}
% \caption{{\em The log-gamma family of densities with central mean
% $<N> \, = \frac{1}{2}$ as a surface and as a contour plot. }}
% \label{pdf}
% \end{figure}

\newpage

%ch.tex


\chapter{The semantic web}
\begin{center}
{\small\em Where are we; how did we get here; and where are we going?}
\end{center}

TBD
\section{How our web framework enables different kinds of application queries}

\subsubsection{An alternative presentation}

If you recall, there's an alternative way to present monads that are
algebras, like our monoid monad. Algebras are presented in terms of
generators and relations. In our case the generators presentation is
really just a grammar for monoid expressions.

\begin{mathpar}
  \inferrule* [lab=expression] {} {{m,n} ::=}
  \and
  \inferrule* [lab=identity element] {} {e}
  \and
  \inferrule* [lab=generators] {} {\;| \; g_1 \; | \; ... \; | \; g_n}
  \and
  \inferrule* [lab=monoid-multiplication] {} {\;| \; m * n}
\end{mathpar} 

This is subject to the following constraints, meaning that we will
treat syntactic expressions of certain forms as denoting the same
element of the monoid. To emphasize the nearly purely syntactic role
of these constraints we will use a different symbol for the
constraints. We also use the same symbol, $\equiv$, for the smallest equivalence
relation respecting these constraints.

\begin{mathpar}
  \inferrule* [lab=identity laws] {} {m * e \equiv m \equiv e * m}
  \and
  \inferrule* [lab=associativity] {} {m_1 * (m_2 * m_3) \equiv (m_1 * m_2) * m_3}
\end{mathpar} 

\paragraph{Logic: the set monad as an algebra}
In a similar manner, there is a language associated with the monad of
sets \emph{considered as an algebra}. This language is very familiar
to most programmers.

\begin{mathpar}
  \inferrule* [lab=expression] {} {{c,d} ::=}
  \and
  \inferrule* [lab=identity verity] {} {true}
  \and
  \inferrule* [lab=negation] {} {\;| \; \neg c}
  \and
  \inferrule* [lab=conjunction] {} {\;| \; c \& d}
\end{mathpar} 

Now, if we had a specific set in hand, say $L$ (which we'll call a
universe in the sequel), we can interpret the expressions in the this
language, aka formulae, in terms of operations on subsets of that
set. As with our compiler for the concrete syntax of the
$lambda$-calculus in chapter 1, we can express this translation very
compactly as

\begin{mathpar}
  \inferrule* {} {\meaningof{true} = L}
  \and
  \inferrule* {} {\meaningof{\neg c} = L \backslash c}
  \and 
  \inferrule* {} {\meaningof{c \& d} = \meaningof{c} \cap \meaningof{d}}
\end{mathpar}

Now, what's happening when we pull the monoid monad through the set
monad via a distributive map is this. First, the monoid monad
furnishes the universe, $L$, as the set of expressions generated by
the grammar. We'll denote this by $L(m)$. Then, we enrich the set of
formulae by the operations of the monoid \emph{acting on sets}.

\begin{mathpar}
  \inferrule* [lab=expression] {} {{c,d} ::=}
  \and
  \inferrule* [lab=identity verity] {} {true}
  \and
  \inferrule* [lab=negation] {} {\;| \; \neg c}
  \and
  \inferrule* [lab=conjunction] {} {\;| \; c \& d}
  \and
  \inferrule* [lab=identity verity] {} {\bf{e}}
  \and
  \inferrule* [lab=negation] {} {\;| \; \bf{g_1} \; | \; ... \; | \; \bf{g_n}}
  \and
  \inferrule* [lab=conjunction] {} {\;| \; c * d}
\end{mathpar} 

The identity element, $e$ and the generators of the monoid, $g_1$,
..., $g_n$, can be considered $0$-ary operations in the same way that
we usually consider constants as $0$-ary operations. To avoid
confusion between these elements and the \emph{logical formulae} that
pick them out of the crowd, we write the logical formulae in
$\bf{boldface}$.

Now, we can write our distributive map. Surprisingly, it is exactly a
meaning for our logic!

\begin{mathpar}
  \inferrule* {} {\meaningof{true} = L(m)}
  \and
  \inferrule* {} {\meaningof{\neg c} = L(m) \backslash c}
  \and 
  \inferrule* {} {\meaningof{c \& d} = \meaningof{c} \cap \meaningof{d}}
  \and
  \inferrule* {} {\meaningof{\bf{e}} = \{ m \; \in \; L(m) \; | \; m \equiv e \}}
  \and
  \inferrule* {} {\meaningof{\bf{g_i}} = \{ m \; \in \; L(m) \; | \; m \equiv g_i \}}
  \and
  \inferrule* {} {\meaningof{c*d} = \{ m \; \in \; L(m) \; | \; m \equiv m_1 * m_2, m_1 \; \in \; \meaningof{c}, m_2 \; \in \; \meaningof{d} \}}
\end{mathpar}

\paragraph{Primes: an application}
Before going any further, let's look at an example of how to use these
new operators. Suppose we wanted to pick out all the elements of the
monoid that were not expressible as a composition of other
elements. Obviously, for monoids with a finite set of generators, this
is exactly just the generators, so we could write $\bf{g_1} || ... ||
\bf{g_n}$\footnote{We get the disjunction, $||$, by the usual DeMorgan
  translation: $c || d \stackrel{def}{=} \neg( \neg c \& \neg
  d)$}. However, when the set of generators is not finite, as it is
when the monoid is the integers under multiplication, we need another
way to write this down. That's where our other operators come in
handy. A moment's thought suggests that we could say that since $true$
denotes any possible element in the monoid, an element is not a
composition using negation plus our composition formula, i.e. $\neg
(true * true)$. This is a little overkill, however. We just want to
eliminate non-trivial compositions. We know how to express the
identity element, that's $\bf{e}$, so we are interested in those
elements that are not the identity, i.e. $\neg \bf{e}$. Then a formula
that eliminates compositions of non-trivial elements is spelled out
$\neg (\neg e * \neg e)$. Finally, we want to eliminate the identity
as a solution. So, we arrive at $\neg (\neg e * \neg e) \& \neg
e$. There, that formula picks out the \emph{primes} of \emph{any}
monoid.

\paragraph{Summary}

What have we done? We've illustrated a specific distributive map, one
that pulls the set monad through the monoid monad. We've shown that
this particular distributive map coincides with giving a semantics to
a particular logic, one whose structure is derived solely from the
shape of the collection monad, i.e. set, and the shape of the term
language, in this case monoid.

\subsubsection{Iterating the design pattern}

The whole point of working in this manner is that by virtue of its
compositional structure it provides a much higher level of abstraction
and greater opportunities for reuse. To illustrate the point, we will
now iterate the construction using our toy language, the
$lambda$-calculus, as the term language. As we saw in chapter 1, the
$lambda$-calculus also has a generators and relations
presentation. Unlike a monoid, however, the lambda calculus has
another piece of machinery: reduction! In addition to structural
equivalence of terms (which is a bi-directional relation) there is the
$beta$-reduction rule that captures the \emph{behavioral} aspect of
the lambda calculus.

It is key to understand this underlying structure of language
definitions. In essence, when a DSL is purely about structure it is
presented entirely in terms of generators (read a grammar) and
relations (like the monoid laws). When the DSL is also about behavior,
i.e. the terms in the language somehow express some kind of
computation, then the language has a third component, some kind of
reduction relation. \footnote{In some sense this is one of the central
  contributions of the theory of computation back to
  mathematics. Algebraists have known for a long time about generators
  and relations presentations of algebraic structures (of which
  algebraic data types are a subset). This collective wisdom is
  studied, for example, in the field of universal
  algebra. Computational models like the $lambda$-calculus and more
  recently the process calculi, like Milner's $\pi$-calculus or
  Cardelli and Gordon's ambient calculus, take this presentation one
  step further and add a set of conditional rewrite rules to express
  the computational content of the model. It was Milner who first
  recognized this particular decomposition of language definitions in
  his seminal paper, Functions as Processes, where he reformulated the
  presentation $\pi$-calculus along these lines.} This organization,
this common factoring of the specification of a language, makes it
possible to factor code that handles a wide range of semantic
features. The logic we derive below provides a great example.

\section{Searching for programs}

TBD

% \section{Existence problems}
% We begin with some metamathematics.
% All problems about the existence of maps can be cast into one of the
% following two forms, which are in a sense mutually dual.

% \noindent
% {\bf The Extension Problem}\index{extension problem} \    %%% NB index entry tag
% Given an inclusion $ A \stackrel{i}{\hookrightarrow} X $, and a map
% $ A \stackrel{f}{\rightarrow} Y $,
% does there exist a map $f^{\dagger}:X\to Y$ such that
% $f^{\dagger}$ agrees with $f$ on $A$?

% Here the appropriate source category for maps should be clear from the
% context and, moreover, commutativity through a
% candidate $f^{\dagger}$ is precisely
% the restriction requirement; that is,
% $$f^{\dagger}   :  f^{\dagger}\circ i = f^{\dagger}|_A = f\,. $$
% If such an $f^{\dagger}$ exists\footnote{${}^{\dagger}$ suggests striving
% for perfection, crusading}, then it is called an {\bf
% extension}\index{extension!of a map|bi} of $f$ and is said to {\bf
% extend}\index{extend|bi} $f$. In any diagrams, the presence of
% a dotted arrow or an arrow carrying a ? indicates a pious hope, in no way
% begging the question of its existence. Note that we shall usually
% omit $\circ$ from composite maps.

% \noindent
% {\bf The Lifting Problem}\index{lifting problem} \
% Given a pair of maps $E \stackrel{p}{\rightarrow}B$ and $X \stackrel{f}
% {\rightarrow} B $,
% does there exist a map $f^{\circ} : X \to E$, with
% $pf^{\circ} = f  $?


% That {\em all\/} existence problems about maps are essentially of one
% type or
% the other from these two is seen as follows. Evidently, all existence problems
% are representable by triangular diagrams\index{triangular diagrams} and it
% is easily seen that there are only these six possibilities:
% \begin{center}\begin{picture}(300,70)  %augch2 75
% \put(5,60){\vector(1,0){30}}
% \put(55,60){\vector(1,0){30}}
% \put(135,60){\vector(-1,0){30}}
% \put(185,60){\vector(-1,0){30}}
% \put(235,60){\vector(-1,0){30}}
% \put(285,60){\vector(-1,0){30}}
% \put(0,55){\vector(0,-1){30}}
% \put(50,55){\vector(0,-1){30}}
% \put(100,25){\vector(0,1){30}}
% \put(150,25){\vector(0,1){30}}
% \put(200,55){\vector(0,-1){30}}
% \put(250,55){\vector(0,-1){30}}
% \put(28,33){\small ?}
% \put(78,33){\small ?}
% \put(128,33){\small ?}
% \put(178,33){\small ?}
% \put(228,33){\small ?}
% \put(278,33){\small ?}
% \put(10,3){\bf 1}
% \put(60,3){\bf 2}
% \put(110,3){\bf 3}
% \put(160,3){\bf 4}
% \put(210,3){\bf 5}
% \put(260,3){\bf 6}
% \put(35,55){\vector(-1,-1){30}}
% \put(155,25){\vector(1,1){30}}
% \put(135,55){\vector(-1,-1){30}}
% \put(55,25){\vector(1,1){30}}
% \put(235,55){\vector(-1,-1){30}}
% \put(255,25){\vector(1,1){30}}
% \end{picture}\end{center}



% \begin{figure}
% \begin{picture}(300,220)(0,0)
% \put(-20,-20){\resizebox{20 cm}{!}{\includegraphics{3dpdf}}}
% \put(260,-10){\resizebox{15 cm}{!}{\includegraphics{contpdf}}}
% \put(220,80){$\beta$}
% \put(400,-10){$N$}
% \put(260,170){$\beta$}
% \put(90,15){$N$}
% \end{picture}
% \caption{{\em The log-gamma family of densities with central mean
% $<N> \, = \frac{1}{2}$ as a surface and as a contour plot. }}
% \label{pdf}
% \end{figure}

\newpage

%ch.tex


\chapter{The semantic web}
\begin{center}
{\small\em Where are we; how did we get here; and where are we going?}
\end{center}

TBD
\section{How our web framework enables different kinds of application queries}

\subsubsection{An alternative presentation}

If you recall, there's an alternative way to present monads that are
algebras, like our monoid monad. Algebras are presented in terms of
generators and relations. In our case the generators presentation is
really just a grammar for monoid expressions.

\begin{mathpar}
  \inferrule* [lab=expression] {} {{m,n} ::=}
  \and
  \inferrule* [lab=identity element] {} {e}
  \and
  \inferrule* [lab=generators] {} {\;| \; g_1 \; | \; ... \; | \; g_n}
  \and
  \inferrule* [lab=monoid-multiplication] {} {\;| \; m * n}
\end{mathpar} 

This is subject to the following constraints, meaning that we will
treat syntactic expressions of certain forms as denoting the same
element of the monoid. To emphasize the nearly purely syntactic role
of these constraints we will use a different symbol for the
constraints. We also use the same symbol, $\equiv$, for the smallest equivalence
relation respecting these constraints.

\begin{mathpar}
  \inferrule* [lab=identity laws] {} {m * e \equiv m \equiv e * m}
  \and
  \inferrule* [lab=associativity] {} {m_1 * (m_2 * m_3) \equiv (m_1 * m_2) * m_3}
\end{mathpar} 

\paragraph{Logic: the set monad as an algebra}
In a similar manner, there is a language associated with the monad of
sets \emph{considered as an algebra}. This language is very familiar
to most programmers.

\begin{mathpar}
  \inferrule* [lab=expression] {} {{c,d} ::=}
  \and
  \inferrule* [lab=identity verity] {} {true}
  \and
  \inferrule* [lab=negation] {} {\;| \; \neg c}
  \and
  \inferrule* [lab=conjunction] {} {\;| \; c \& d}
\end{mathpar} 

Now, if we had a specific set in hand, say $L$ (which we'll call a
universe in the sequel), we can interpret the expressions in the this
language, aka formulae, in terms of operations on subsets of that
set. As with our compiler for the concrete syntax of the
$lambda$-calculus in chapter 1, we can express this translation very
compactly as

\begin{mathpar}
  \inferrule* {} {\meaningof{true} = L}
  \and
  \inferrule* {} {\meaningof{\neg c} = L \backslash c}
  \and 
  \inferrule* {} {\meaningof{c \& d} = \meaningof{c} \cap \meaningof{d}}
\end{mathpar}

Now, what's happening when we pull the monoid monad through the set
monad via a distributive map is this. First, the monoid monad
furnishes the universe, $L$, as the set of expressions generated by
the grammar. We'll denote this by $L(m)$. Then, we enrich the set of
formulae by the operations of the monoid \emph{acting on sets}.

\begin{mathpar}
  \inferrule* [lab=expression] {} {{c,d} ::=}
  \and
  \inferrule* [lab=identity verity] {} {true}
  \and
  \inferrule* [lab=negation] {} {\;| \; \neg c}
  \and
  \inferrule* [lab=conjunction] {} {\;| \; c \& d}
  \and
  \inferrule* [lab=identity verity] {} {\bf{e}}
  \and
  \inferrule* [lab=negation] {} {\;| \; \bf{g_1} \; | \; ... \; | \; \bf{g_n}}
  \and
  \inferrule* [lab=conjunction] {} {\;| \; c * d}
\end{mathpar} 

The identity element, $e$ and the generators of the monoid, $g_1$,
..., $g_n$, can be considered $0$-ary operations in the same way that
we usually consider constants as $0$-ary operations. To avoid
confusion between these elements and the \emph{logical formulae} that
pick them out of the crowd, we write the logical formulae in
$\bf{boldface}$.

Now, we can write our distributive map. Surprisingly, it is exactly a
meaning for our logic!

\begin{mathpar}
  \inferrule* {} {\meaningof{true} = L(m)}
  \and
  \inferrule* {} {\meaningof{\neg c} = L(m) \backslash c}
  \and 
  \inferrule* {} {\meaningof{c \& d} = \meaningof{c} \cap \meaningof{d}}
  \and
  \inferrule* {} {\meaningof{\bf{e}} = \{ m \; \in \; L(m) \; | \; m \equiv e \}}
  \and
  \inferrule* {} {\meaningof{\bf{g_i}} = \{ m \; \in \; L(m) \; | \; m \equiv g_i \}}
  \and
  \inferrule* {} {\meaningof{c*d} = \{ m \; \in \; L(m) \; | \; m \equiv m_1 * m_2, m_1 \; \in \; \meaningof{c}, m_2 \; \in \; \meaningof{d} \}}
\end{mathpar}

\paragraph{Primes: an application}
Before going any further, let's look at an example of how to use these
new operators. Suppose we wanted to pick out all the elements of the
monoid that were not expressible as a composition of other
elements. Obviously, for monoids with a finite set of generators, this
is exactly just the generators, so we could write $\bf{g_1} || ... ||
\bf{g_n}$\footnote{We get the disjunction, $||$, by the usual DeMorgan
  translation: $c || d \stackrel{def}{=} \neg( \neg c \& \neg
  d)$}. However, when the set of generators is not finite, as it is
when the monoid is the integers under multiplication, we need another
way to write this down. That's where our other operators come in
handy. A moment's thought suggests that we could say that since $true$
denotes any possible element in the monoid, an element is not a
composition using negation plus our composition formula, i.e. $\neg
(true * true)$. This is a little overkill, however. We just want to
eliminate non-trivial compositions. We know how to express the
identity element, that's $\bf{e}$, so we are interested in those
elements that are not the identity, i.e. $\neg \bf{e}$. Then a formula
that eliminates compositions of non-trivial elements is spelled out
$\neg (\neg e * \neg e)$. Finally, we want to eliminate the identity
as a solution. So, we arrive at $\neg (\neg e * \neg e) \& \neg
e$. There, that formula picks out the \emph{primes} of \emph{any}
monoid.

\paragraph{Summary}

What have we done? We've illustrated a specific distributive map, one
that pulls the set monad through the monoid monad. We've shown that
this particular distributive map coincides with giving a semantics to
a particular logic, one whose structure is derived solely from the
shape of the collection monad, i.e. set, and the shape of the term
language, in this case monoid.

\subsubsection{Iterating the design pattern}

The whole point of working in this manner is that by virtue of its
compositional structure it provides a much higher level of abstraction
and greater opportunities for reuse. To illustrate the point, we will
now iterate the construction using our toy language, the
$lambda$-calculus, as the term language. As we saw in chapter 1, the
$lambda$-calculus also has a generators and relations
presentation. Unlike a monoid, however, the lambda calculus has
another piece of machinery: reduction! In addition to structural
equivalence of terms (which is a bi-directional relation) there is the
$beta$-reduction rule that captures the \emph{behavioral} aspect of
the lambda calculus.

It is key to understand this underlying structure of language
definitions. In essence, when a DSL is purely about structure it is
presented entirely in terms of generators (read a grammar) and
relations (like the monoid laws). When the DSL is also about behavior,
i.e. the terms in the language somehow express some kind of
computation, then the language has a third component, some kind of
reduction relation. \footnote{In some sense this is one of the central
  contributions of the theory of computation back to
  mathematics. Algebraists have known for a long time about generators
  and relations presentations of algebraic structures (of which
  algebraic data types are a subset). This collective wisdom is
  studied, for example, in the field of universal
  algebra. Computational models like the $lambda$-calculus and more
  recently the process calculi, like Milner's $\pi$-calculus or
  Cardelli and Gordon's ambient calculus, take this presentation one
  step further and add a set of conditional rewrite rules to express
  the computational content of the model. It was Milner who first
  recognized this particular decomposition of language definitions in
  his seminal paper, Functions as Processes, where he reformulated the
  presentation $\pi$-calculus along these lines.} This organization,
this common factoring of the specification of a language, makes it
possible to factor code that handles a wide range of semantic
features. The logic we derive below provides a great example.

\section{Searching for programs}

TBD

% \section{Existence problems}
% We begin with some metamathematics.
% All problems about the existence of maps can be cast into one of the
% following two forms, which are in a sense mutually dual.

% \noindent
% {\bf The Extension Problem}\index{extension problem} \    %%% NB index entry tag
% Given an inclusion $ A \stackrel{i}{\hookrightarrow} X $, and a map
% $ A \stackrel{f}{\rightarrow} Y $,
% does there exist a map $f^{\dagger}:X\to Y$ such that
% $f^{\dagger}$ agrees with $f$ on $A$?

% Here the appropriate source category for maps should be clear from the
% context and, moreover, commutativity through a
% candidate $f^{\dagger}$ is precisely
% the restriction requirement; that is,
% $$f^{\dagger}   :  f^{\dagger}\circ i = f^{\dagger}|_A = f\,. $$
% If such an $f^{\dagger}$ exists\footnote{${}^{\dagger}$ suggests striving
% for perfection, crusading}, then it is called an {\bf
% extension}\index{extension!of a map|bi} of $f$ and is said to {\bf
% extend}\index{extend|bi} $f$. In any diagrams, the presence of
% a dotted arrow or an arrow carrying a ? indicates a pious hope, in no way
% begging the question of its existence. Note that we shall usually
% omit $\circ$ from composite maps.

% \noindent
% {\bf The Lifting Problem}\index{lifting problem} \
% Given a pair of maps $E \stackrel{p}{\rightarrow}B$ and $X \stackrel{f}
% {\rightarrow} B $,
% does there exist a map $f^{\circ} : X \to E$, with
% $pf^{\circ} = f  $?


% That {\em all\/} existence problems about maps are essentially of one
% type or
% the other from these two is seen as follows. Evidently, all existence problems
% are representable by triangular diagrams\index{triangular diagrams} and it
% is easily seen that there are only these six possibilities:
% \begin{center}\begin{picture}(300,70)  %augch2 75
% \put(5,60){\vector(1,0){30}}
% \put(55,60){\vector(1,0){30}}
% \put(135,60){\vector(-1,0){30}}
% \put(185,60){\vector(-1,0){30}}
% \put(235,60){\vector(-1,0){30}}
% \put(285,60){\vector(-1,0){30}}
% \put(0,55){\vector(0,-1){30}}
% \put(50,55){\vector(0,-1){30}}
% \put(100,25){\vector(0,1){30}}
% \put(150,25){\vector(0,1){30}}
% \put(200,55){\vector(0,-1){30}}
% \put(250,55){\vector(0,-1){30}}
% \put(28,33){\small ?}
% \put(78,33){\small ?}
% \put(128,33){\small ?}
% \put(178,33){\small ?}
% \put(228,33){\small ?}
% \put(278,33){\small ?}
% \put(10,3){\bf 1}
% \put(60,3){\bf 2}
% \put(110,3){\bf 3}
% \put(160,3){\bf 4}
% \put(210,3){\bf 5}
% \put(260,3){\bf 6}
% \put(35,55){\vector(-1,-1){30}}
% \put(155,25){\vector(1,1){30}}
% \put(135,55){\vector(-1,-1){30}}
% \put(55,25){\vector(1,1){30}}
% \put(235,55){\vector(-1,-1){30}}
% \put(255,25){\vector(1,1){30}}
% \end{picture}\end{center}



% \begin{figure}
% \begin{picture}(300,220)(0,0)
% \put(-20,-20){\resizebox{20 cm}{!}{\includegraphics{3dpdf}}}
% \put(260,-10){\resizebox{15 cm}{!}{\includegraphics{contpdf}}}
% \put(220,80){$\beta$}
% \put(400,-10){$N$}
% \put(260,170){$\beta$}
% \put(90,15){$N$}
% \end{picture}
% \caption{{\em The log-gamma family of densities with central mean
% $<N> \, = \frac{1}{2}$ as a surface and as a contour plot. }}
% \label{pdf}
% \end{figure}

\newpage



\documentclass[12pt,leqno]{book}
\usepackage{amsmath,amssymb,amsfonts} % Typical maths resource packages
\usepackage{graphics}                 % Packages to allow inclusion of graphics
\usepackage{color}                    % For creating coloured text and background
\usepackage{hyperref}                 % For creating hyperlinks in cross references
\usepackage{makeidx}                  % For indexing
\usepackage{listings}                 % For code listing
\usepackage{mathpartir}               % For grammars, rules, etc
\usepackage{bcprules}                 % For other kinds of rules

\lstloadlanguages{Scala,Java,Haskell,XML,bash,HTML,SQL}

\parindent 1cm
\parskip 0.2cm
\topmargin 0.2cm
\oddsidemargin 1cm
\evensidemargin 0.5cm
\textwidth 15cm
\textheight 21cm

\newtheorem{theorem}{Theorem}[section]
\newtheorem{proposition}[theorem]{Proposition}
\newtheorem{corollary}[theorem]{Corollary}
\newtheorem{lemma}[theorem]{Lemma}
\newtheorem{remark}[theorem]{Remark}
\newtheorem{definition}[theorem]{Definition}


\def\R{\mathbb{ R}}
\def\S{\mathbb{ S}}
\def\I{\mathbb{ I}}

\def\Scala{\texttt{Scala}}
\def\ScalaCheck{\texttt{ScalaCheck}}
\def\Haskell{\texttt{Haskell}}
\def\XML{\texttt{XML}}


\makeindex


\title{Pro Scala: Monadic Design Patterns for the Web}

\author{L.G. Meredith  \\
{\small\em \copyright \  Draft date \today }}

 \date{ }
\begin{document}
\lstset{language=Haskell}
\maketitle
 \addcontentsline{toc}{chapter}{Contents}
\pagenumbering{roman}
\tableofcontents
\listoffigures
\listoftables
\chapter*{Preface}\normalsize
  \addcontentsline{toc}{chapter}{Preface}
\pagestyle{plain}
% The book root file {\tt bookex.tex} gives a basic example of how to
% use \LaTeX \ for preparation of a book. Note that all
% \LaTeX \ commands begin with a
% backslash.

% Each
% Chapter, Appendix and the Index is made as a {\tt *.tex} file and is
% called in by the {\tt include} command---thus {\tt ch1.tex} is
% the name here of the file containing Chapter~1. The inclusion of any
% particular file can be suppressed by prefixing the line by a
% percent sign.


%  Do not put an {\tt end{document}} command at the end of chapter files;
% just one such command is needed at the end of the book.

% Note the tag used to make an index entry. You may need to consult Lamport's
% book~\cite{lamport} for details of the procedure to make the index input
% file; \LaTeX \ will create a pre-index by listing all the tagged
% items in the file {\tt bookex.idx} then you edit this into
% a {\tt theindex} environment, as {\tt index.tex}.

The book you hold in your hands, Dear Reader, is not at all what you expected...



\pagestyle{headings}
\pagenumbering{arabic}

%ch.tex


\chapter{The semantic web}
\begin{center}
{\small\em Where are we; how did we get here; and where are we going?}
\end{center}

TBD
\section{How our web framework enables different kinds of application queries}

\subsubsection{An alternative presentation}

If you recall, there's an alternative way to present monads that are
algebras, like our monoid monad. Algebras are presented in terms of
generators and relations. In our case the generators presentation is
really just a grammar for monoid expressions.

\begin{mathpar}
  \inferrule* [lab=expression] {} {{m,n} ::=}
  \and
  \inferrule* [lab=identity element] {} {e}
  \and
  \inferrule* [lab=generators] {} {\;| \; g_1 \; | \; ... \; | \; g_n}
  \and
  \inferrule* [lab=monoid-multiplication] {} {\;| \; m * n}
\end{mathpar} 

This is subject to the following constraints, meaning that we will
treat syntactic expressions of certain forms as denoting the same
element of the monoid. To emphasize the nearly purely syntactic role
of these constraints we will use a different symbol for the
constraints. We also use the same symbol, $\equiv$, for the smallest equivalence
relation respecting these constraints.

\begin{mathpar}
  \inferrule* [lab=identity laws] {} {m * e \equiv m \equiv e * m}
  \and
  \inferrule* [lab=associativity] {} {m_1 * (m_2 * m_3) \equiv (m_1 * m_2) * m_3}
\end{mathpar} 

\paragraph{Logic: the set monad as an algebra}
In a similar manner, there is a language associated with the monad of
sets \emph{considered as an algebra}. This language is very familiar
to most programmers.

\begin{mathpar}
  \inferrule* [lab=expression] {} {{c,d} ::=}
  \and
  \inferrule* [lab=identity verity] {} {true}
  \and
  \inferrule* [lab=negation] {} {\;| \; \neg c}
  \and
  \inferrule* [lab=conjunction] {} {\;| \; c \& d}
\end{mathpar} 

Now, if we had a specific set in hand, say $L$ (which we'll call a
universe in the sequel), we can interpret the expressions in the this
language, aka formulae, in terms of operations on subsets of that
set. As with our compiler for the concrete syntax of the
$lambda$-calculus in chapter 1, we can express this translation very
compactly as

\begin{mathpar}
  \inferrule* {} {\meaningof{true} = L}
  \and
  \inferrule* {} {\meaningof{\neg c} = L \backslash c}
  \and 
  \inferrule* {} {\meaningof{c \& d} = \meaningof{c} \cap \meaningof{d}}
\end{mathpar}

Now, what's happening when we pull the monoid monad through the set
monad via a distributive map is this. First, the monoid monad
furnishes the universe, $L$, as the set of expressions generated by
the grammar. We'll denote this by $L(m)$. Then, we enrich the set of
formulae by the operations of the monoid \emph{acting on sets}.

\begin{mathpar}
  \inferrule* [lab=expression] {} {{c,d} ::=}
  \and
  \inferrule* [lab=identity verity] {} {true}
  \and
  \inferrule* [lab=negation] {} {\;| \; \neg c}
  \and
  \inferrule* [lab=conjunction] {} {\;| \; c \& d}
  \and
  \inferrule* [lab=identity verity] {} {\bf{e}}
  \and
  \inferrule* [lab=negation] {} {\;| \; \bf{g_1} \; | \; ... \; | \; \bf{g_n}}
  \and
  \inferrule* [lab=conjunction] {} {\;| \; c * d}
\end{mathpar} 

The identity element, $e$ and the generators of the monoid, $g_1$,
..., $g_n$, can be considered $0$-ary operations in the same way that
we usually consider constants as $0$-ary operations. To avoid
confusion between these elements and the \emph{logical formulae} that
pick them out of the crowd, we write the logical formulae in
$\bf{boldface}$.

Now, we can write our distributive map. Surprisingly, it is exactly a
meaning for our logic!

\begin{mathpar}
  \inferrule* {} {\meaningof{true} = L(m)}
  \and
  \inferrule* {} {\meaningof{\neg c} = L(m) \backslash c}
  \and 
  \inferrule* {} {\meaningof{c \& d} = \meaningof{c} \cap \meaningof{d}}
  \and
  \inferrule* {} {\meaningof{\bf{e}} = \{ m \; \in \; L(m) \; | \; m \equiv e \}}
  \and
  \inferrule* {} {\meaningof{\bf{g_i}} = \{ m \; \in \; L(m) \; | \; m \equiv g_i \}}
  \and
  \inferrule* {} {\meaningof{c*d} = \{ m \; \in \; L(m) \; | \; m \equiv m_1 * m_2, m_1 \; \in \; \meaningof{c}, m_2 \; \in \; \meaningof{d} \}}
\end{mathpar}

\paragraph{Primes: an application}
Before going any further, let's look at an example of how to use these
new operators. Suppose we wanted to pick out all the elements of the
monoid that were not expressible as a composition of other
elements. Obviously, for monoids with a finite set of generators, this
is exactly just the generators, so we could write $\bf{g_1} || ... ||
\bf{g_n}$\footnote{We get the disjunction, $||$, by the usual DeMorgan
  translation: $c || d \stackrel{def}{=} \neg( \neg c \& \neg
  d)$}. However, when the set of generators is not finite, as it is
when the monoid is the integers under multiplication, we need another
way to write this down. That's where our other operators come in
handy. A moment's thought suggests that we could say that since $true$
denotes any possible element in the monoid, an element is not a
composition using negation plus our composition formula, i.e. $\neg
(true * true)$. This is a little overkill, however. We just want to
eliminate non-trivial compositions. We know how to express the
identity element, that's $\bf{e}$, so we are interested in those
elements that are not the identity, i.e. $\neg \bf{e}$. Then a formula
that eliminates compositions of non-trivial elements is spelled out
$\neg (\neg e * \neg e)$. Finally, we want to eliminate the identity
as a solution. So, we arrive at $\neg (\neg e * \neg e) \& \neg
e$. There, that formula picks out the \emph{primes} of \emph{any}
monoid.

\paragraph{Summary}

What have we done? We've illustrated a specific distributive map, one
that pulls the set monad through the monoid monad. We've shown that
this particular distributive map coincides with giving a semantics to
a particular logic, one whose structure is derived solely from the
shape of the collection monad, i.e. set, and the shape of the term
language, in this case monoid.

\subsubsection{Iterating the design pattern}

The whole point of working in this manner is that by virtue of its
compositional structure it provides a much higher level of abstraction
and greater opportunities for reuse. To illustrate the point, we will
now iterate the construction using our toy language, the
$lambda$-calculus, as the term language. As we saw in chapter 1, the
$lambda$-calculus also has a generators and relations
presentation. Unlike a monoid, however, the lambda calculus has
another piece of machinery: reduction! In addition to structural
equivalence of terms (which is a bi-directional relation) there is the
$beta$-reduction rule that captures the \emph{behavioral} aspect of
the lambda calculus.

It is key to understand this underlying structure of language
definitions. In essence, when a DSL is purely about structure it is
presented entirely in terms of generators (read a grammar) and
relations (like the monoid laws). When the DSL is also about behavior,
i.e. the terms in the language somehow express some kind of
computation, then the language has a third component, some kind of
reduction relation. \footnote{In some sense this is one of the central
  contributions of the theory of computation back to
  mathematics. Algebraists have known for a long time about generators
  and relations presentations of algebraic structures (of which
  algebraic data types are a subset). This collective wisdom is
  studied, for example, in the field of universal
  algebra. Computational models like the $lambda$-calculus and more
  recently the process calculi, like Milner's $\pi$-calculus or
  Cardelli and Gordon's ambient calculus, take this presentation one
  step further and add a set of conditional rewrite rules to express
  the computational content of the model. It was Milner who first
  recognized this particular decomposition of language definitions in
  his seminal paper, Functions as Processes, where he reformulated the
  presentation $\pi$-calculus along these lines.} This organization,
this common factoring of the specification of a language, makes it
possible to factor code that handles a wide range of semantic
features. The logic we derive below provides a great example.

\section{Searching for programs}

TBD

% \section{Existence problems}
% We begin with some metamathematics.
% All problems about the existence of maps can be cast into one of the
% following two forms, which are in a sense mutually dual.

% \noindent
% {\bf The Extension Problem}\index{extension problem} \    %%% NB index entry tag
% Given an inclusion $ A \stackrel{i}{\hookrightarrow} X $, and a map
% $ A \stackrel{f}{\rightarrow} Y $,
% does there exist a map $f^{\dagger}:X\to Y$ such that
% $f^{\dagger}$ agrees with $f$ on $A$?

% Here the appropriate source category for maps should be clear from the
% context and, moreover, commutativity through a
% candidate $f^{\dagger}$ is precisely
% the restriction requirement; that is,
% $$f^{\dagger}   :  f^{\dagger}\circ i = f^{\dagger}|_A = f\,. $$
% If such an $f^{\dagger}$ exists\footnote{${}^{\dagger}$ suggests striving
% for perfection, crusading}, then it is called an {\bf
% extension}\index{extension!of a map|bi} of $f$ and is said to {\bf
% extend}\index{extend|bi} $f$. In any diagrams, the presence of
% a dotted arrow or an arrow carrying a ? indicates a pious hope, in no way
% begging the question of its existence. Note that we shall usually
% omit $\circ$ from composite maps.

% \noindent
% {\bf The Lifting Problem}\index{lifting problem} \
% Given a pair of maps $E \stackrel{p}{\rightarrow}B$ and $X \stackrel{f}
% {\rightarrow} B $,
% does there exist a map $f^{\circ} : X \to E$, with
% $pf^{\circ} = f  $?


% That {\em all\/} existence problems about maps are essentially of one
% type or
% the other from these two is seen as follows. Evidently, all existence problems
% are representable by triangular diagrams\index{triangular diagrams} and it
% is easily seen that there are only these six possibilities:
% \begin{center}\begin{picture}(300,70)  %augch2 75
% \put(5,60){\vector(1,0){30}}
% \put(55,60){\vector(1,0){30}}
% \put(135,60){\vector(-1,0){30}}
% \put(185,60){\vector(-1,0){30}}
% \put(235,60){\vector(-1,0){30}}
% \put(285,60){\vector(-1,0){30}}
% \put(0,55){\vector(0,-1){30}}
% \put(50,55){\vector(0,-1){30}}
% \put(100,25){\vector(0,1){30}}
% \put(150,25){\vector(0,1){30}}
% \put(200,55){\vector(0,-1){30}}
% \put(250,55){\vector(0,-1){30}}
% \put(28,33){\small ?}
% \put(78,33){\small ?}
% \put(128,33){\small ?}
% \put(178,33){\small ?}
% \put(228,33){\small ?}
% \put(278,33){\small ?}
% \put(10,3){\bf 1}
% \put(60,3){\bf 2}
% \put(110,3){\bf 3}
% \put(160,3){\bf 4}
% \put(210,3){\bf 5}
% \put(260,3){\bf 6}
% \put(35,55){\vector(-1,-1){30}}
% \put(155,25){\vector(1,1){30}}
% \put(135,55){\vector(-1,-1){30}}
% \put(55,25){\vector(1,1){30}}
% \put(235,55){\vector(-1,-1){30}}
% \put(255,25){\vector(1,1){30}}
% \end{picture}\end{center}



% \begin{figure}
% \begin{picture}(300,220)(0,0)
% \put(-20,-20){\resizebox{20 cm}{!}{\includegraphics{3dpdf}}}
% \put(260,-10){\resizebox{15 cm}{!}{\includegraphics{contpdf}}}
% \put(220,80){$\beta$}
% \put(400,-10){$N$}
% \put(260,170){$\beta$}
% \put(90,15){$N$}
% \end{picture}
% \caption{{\em The log-gamma family of densities with central mean
% $<N> \, = \frac{1}{2}$ as a surface and as a contour plot. }}
% \label{pdf}
% \end{figure}

\newpage

%ch.tex


\chapter{The semantic web}
\begin{center}
{\small\em Where are we; how did we get here; and where are we going?}
\end{center}

TBD
\section{How our web framework enables different kinds of application queries}

\subsubsection{An alternative presentation}

If you recall, there's an alternative way to present monads that are
algebras, like our monoid monad. Algebras are presented in terms of
generators and relations. In our case the generators presentation is
really just a grammar for monoid expressions.

\begin{mathpar}
  \inferrule* [lab=expression] {} {{m,n} ::=}
  \and
  \inferrule* [lab=identity element] {} {e}
  \and
  \inferrule* [lab=generators] {} {\;| \; g_1 \; | \; ... \; | \; g_n}
  \and
  \inferrule* [lab=monoid-multiplication] {} {\;| \; m * n}
\end{mathpar} 

This is subject to the following constraints, meaning that we will
treat syntactic expressions of certain forms as denoting the same
element of the monoid. To emphasize the nearly purely syntactic role
of these constraints we will use a different symbol for the
constraints. We also use the same symbol, $\equiv$, for the smallest equivalence
relation respecting these constraints.

\begin{mathpar}
  \inferrule* [lab=identity laws] {} {m * e \equiv m \equiv e * m}
  \and
  \inferrule* [lab=associativity] {} {m_1 * (m_2 * m_3) \equiv (m_1 * m_2) * m_3}
\end{mathpar} 

\paragraph{Logic: the set monad as an algebra}
In a similar manner, there is a language associated with the monad of
sets \emph{considered as an algebra}. This language is very familiar
to most programmers.

\begin{mathpar}
  \inferrule* [lab=expression] {} {{c,d} ::=}
  \and
  \inferrule* [lab=identity verity] {} {true}
  \and
  \inferrule* [lab=negation] {} {\;| \; \neg c}
  \and
  \inferrule* [lab=conjunction] {} {\;| \; c \& d}
\end{mathpar} 

Now, if we had a specific set in hand, say $L$ (which we'll call a
universe in the sequel), we can interpret the expressions in the this
language, aka formulae, in terms of operations on subsets of that
set. As with our compiler for the concrete syntax of the
$lambda$-calculus in chapter 1, we can express this translation very
compactly as

\begin{mathpar}
  \inferrule* {} {\meaningof{true} = L}
  \and
  \inferrule* {} {\meaningof{\neg c} = L \backslash c}
  \and 
  \inferrule* {} {\meaningof{c \& d} = \meaningof{c} \cap \meaningof{d}}
\end{mathpar}

Now, what's happening when we pull the monoid monad through the set
monad via a distributive map is this. First, the monoid monad
furnishes the universe, $L$, as the set of expressions generated by
the grammar. We'll denote this by $L(m)$. Then, we enrich the set of
formulae by the operations of the monoid \emph{acting on sets}.

\begin{mathpar}
  \inferrule* [lab=expression] {} {{c,d} ::=}
  \and
  \inferrule* [lab=identity verity] {} {true}
  \and
  \inferrule* [lab=negation] {} {\;| \; \neg c}
  \and
  \inferrule* [lab=conjunction] {} {\;| \; c \& d}
  \and
  \inferrule* [lab=identity verity] {} {\bf{e}}
  \and
  \inferrule* [lab=negation] {} {\;| \; \bf{g_1} \; | \; ... \; | \; \bf{g_n}}
  \and
  \inferrule* [lab=conjunction] {} {\;| \; c * d}
\end{mathpar} 

The identity element, $e$ and the generators of the monoid, $g_1$,
..., $g_n$, can be considered $0$-ary operations in the same way that
we usually consider constants as $0$-ary operations. To avoid
confusion between these elements and the \emph{logical formulae} that
pick them out of the crowd, we write the logical formulae in
$\bf{boldface}$.

Now, we can write our distributive map. Surprisingly, it is exactly a
meaning for our logic!

\begin{mathpar}
  \inferrule* {} {\meaningof{true} = L(m)}
  \and
  \inferrule* {} {\meaningof{\neg c} = L(m) \backslash c}
  \and 
  \inferrule* {} {\meaningof{c \& d} = \meaningof{c} \cap \meaningof{d}}
  \and
  \inferrule* {} {\meaningof{\bf{e}} = \{ m \; \in \; L(m) \; | \; m \equiv e \}}
  \and
  \inferrule* {} {\meaningof{\bf{g_i}} = \{ m \; \in \; L(m) \; | \; m \equiv g_i \}}
  \and
  \inferrule* {} {\meaningof{c*d} = \{ m \; \in \; L(m) \; | \; m \equiv m_1 * m_2, m_1 \; \in \; \meaningof{c}, m_2 \; \in \; \meaningof{d} \}}
\end{mathpar}

\paragraph{Primes: an application}
Before going any further, let's look at an example of how to use these
new operators. Suppose we wanted to pick out all the elements of the
monoid that were not expressible as a composition of other
elements. Obviously, for monoids with a finite set of generators, this
is exactly just the generators, so we could write $\bf{g_1} || ... ||
\bf{g_n}$\footnote{We get the disjunction, $||$, by the usual DeMorgan
  translation: $c || d \stackrel{def}{=} \neg( \neg c \& \neg
  d)$}. However, when the set of generators is not finite, as it is
when the monoid is the integers under multiplication, we need another
way to write this down. That's where our other operators come in
handy. A moment's thought suggests that we could say that since $true$
denotes any possible element in the monoid, an element is not a
composition using negation plus our composition formula, i.e. $\neg
(true * true)$. This is a little overkill, however. We just want to
eliminate non-trivial compositions. We know how to express the
identity element, that's $\bf{e}$, so we are interested in those
elements that are not the identity, i.e. $\neg \bf{e}$. Then a formula
that eliminates compositions of non-trivial elements is spelled out
$\neg (\neg e * \neg e)$. Finally, we want to eliminate the identity
as a solution. So, we arrive at $\neg (\neg e * \neg e) \& \neg
e$. There, that formula picks out the \emph{primes} of \emph{any}
monoid.

\paragraph{Summary}

What have we done? We've illustrated a specific distributive map, one
that pulls the set monad through the monoid monad. We've shown that
this particular distributive map coincides with giving a semantics to
a particular logic, one whose structure is derived solely from the
shape of the collection monad, i.e. set, and the shape of the term
language, in this case monoid.

\subsubsection{Iterating the design pattern}

The whole point of working in this manner is that by virtue of its
compositional structure it provides a much higher level of abstraction
and greater opportunities for reuse. To illustrate the point, we will
now iterate the construction using our toy language, the
$lambda$-calculus, as the term language. As we saw in chapter 1, the
$lambda$-calculus also has a generators and relations
presentation. Unlike a monoid, however, the lambda calculus has
another piece of machinery: reduction! In addition to structural
equivalence of terms (which is a bi-directional relation) there is the
$beta$-reduction rule that captures the \emph{behavioral} aspect of
the lambda calculus.

It is key to understand this underlying structure of language
definitions. In essence, when a DSL is purely about structure it is
presented entirely in terms of generators (read a grammar) and
relations (like the monoid laws). When the DSL is also about behavior,
i.e. the terms in the language somehow express some kind of
computation, then the language has a third component, some kind of
reduction relation. \footnote{In some sense this is one of the central
  contributions of the theory of computation back to
  mathematics. Algebraists have known for a long time about generators
  and relations presentations of algebraic structures (of which
  algebraic data types are a subset). This collective wisdom is
  studied, for example, in the field of universal
  algebra. Computational models like the $lambda$-calculus and more
  recently the process calculi, like Milner's $\pi$-calculus or
  Cardelli and Gordon's ambient calculus, take this presentation one
  step further and add a set of conditional rewrite rules to express
  the computational content of the model. It was Milner who first
  recognized this particular decomposition of language definitions in
  his seminal paper, Functions as Processes, where he reformulated the
  presentation $\pi$-calculus along these lines.} This organization,
this common factoring of the specification of a language, makes it
possible to factor code that handles a wide range of semantic
features. The logic we derive below provides a great example.

\section{Searching for programs}

TBD

% \section{Existence problems}
% We begin with some metamathematics.
% All problems about the existence of maps can be cast into one of the
% following two forms, which are in a sense mutually dual.

% \noindent
% {\bf The Extension Problem}\index{extension problem} \    %%% NB index entry tag
% Given an inclusion $ A \stackrel{i}{\hookrightarrow} X $, and a map
% $ A \stackrel{f}{\rightarrow} Y $,
% does there exist a map $f^{\dagger}:X\to Y$ such that
% $f^{\dagger}$ agrees with $f$ on $A$?

% Here the appropriate source category for maps should be clear from the
% context and, moreover, commutativity through a
% candidate $f^{\dagger}$ is precisely
% the restriction requirement; that is,
% $$f^{\dagger}   :  f^{\dagger}\circ i = f^{\dagger}|_A = f\,. $$
% If such an $f^{\dagger}$ exists\footnote{${}^{\dagger}$ suggests striving
% for perfection, crusading}, then it is called an {\bf
% extension}\index{extension!of a map|bi} of $f$ and is said to {\bf
% extend}\index{extend|bi} $f$. In any diagrams, the presence of
% a dotted arrow or an arrow carrying a ? indicates a pious hope, in no way
% begging the question of its existence. Note that we shall usually
% omit $\circ$ from composite maps.

% \noindent
% {\bf The Lifting Problem}\index{lifting problem} \
% Given a pair of maps $E \stackrel{p}{\rightarrow}B$ and $X \stackrel{f}
% {\rightarrow} B $,
% does there exist a map $f^{\circ} : X \to E$, with
% $pf^{\circ} = f  $?


% That {\em all\/} existence problems about maps are essentially of one
% type or
% the other from these two is seen as follows. Evidently, all existence problems
% are representable by triangular diagrams\index{triangular diagrams} and it
% is easily seen that there are only these six possibilities:
% \begin{center}\begin{picture}(300,70)  %augch2 75
% \put(5,60){\vector(1,0){30}}
% \put(55,60){\vector(1,0){30}}
% \put(135,60){\vector(-1,0){30}}
% \put(185,60){\vector(-1,0){30}}
% \put(235,60){\vector(-1,0){30}}
% \put(285,60){\vector(-1,0){30}}
% \put(0,55){\vector(0,-1){30}}
% \put(50,55){\vector(0,-1){30}}
% \put(100,25){\vector(0,1){30}}
% \put(150,25){\vector(0,1){30}}
% \put(200,55){\vector(0,-1){30}}
% \put(250,55){\vector(0,-1){30}}
% \put(28,33){\small ?}
% \put(78,33){\small ?}
% \put(128,33){\small ?}
% \put(178,33){\small ?}
% \put(228,33){\small ?}
% \put(278,33){\small ?}
% \put(10,3){\bf 1}
% \put(60,3){\bf 2}
% \put(110,3){\bf 3}
% \put(160,3){\bf 4}
% \put(210,3){\bf 5}
% \put(260,3){\bf 6}
% \put(35,55){\vector(-1,-1){30}}
% \put(155,25){\vector(1,1){30}}
% \put(135,55){\vector(-1,-1){30}}
% \put(55,25){\vector(1,1){30}}
% \put(235,55){\vector(-1,-1){30}}
% \put(255,25){\vector(1,1){30}}
% \end{picture}\end{center}



% \begin{figure}
% \begin{picture}(300,220)(0,0)
% \put(-20,-20){\resizebox{20 cm}{!}{\includegraphics{3dpdf}}}
% \put(260,-10){\resizebox{15 cm}{!}{\includegraphics{contpdf}}}
% \put(220,80){$\beta$}
% \put(400,-10){$N$}
% \put(260,170){$\beta$}
% \put(90,15){$N$}
% \end{picture}
% \caption{{\em The log-gamma family of densities with central mean
% $<N> \, = \frac{1}{2}$ as a surface and as a contour plot. }}
% \label{pdf}
% \end{figure}

\newpage

%ch.tex


\chapter{The semantic web}
\begin{center}
{\small\em Where are we; how did we get here; and where are we going?}
\end{center}

TBD
\section{How our web framework enables different kinds of application queries}

\subsubsection{An alternative presentation}

If you recall, there's an alternative way to present monads that are
algebras, like our monoid monad. Algebras are presented in terms of
generators and relations. In our case the generators presentation is
really just a grammar for monoid expressions.

\begin{mathpar}
  \inferrule* [lab=expression] {} {{m,n} ::=}
  \and
  \inferrule* [lab=identity element] {} {e}
  \and
  \inferrule* [lab=generators] {} {\;| \; g_1 \; | \; ... \; | \; g_n}
  \and
  \inferrule* [lab=monoid-multiplication] {} {\;| \; m * n}
\end{mathpar} 

This is subject to the following constraints, meaning that we will
treat syntactic expressions of certain forms as denoting the same
element of the monoid. To emphasize the nearly purely syntactic role
of these constraints we will use a different symbol for the
constraints. We also use the same symbol, $\equiv$, for the smallest equivalence
relation respecting these constraints.

\begin{mathpar}
  \inferrule* [lab=identity laws] {} {m * e \equiv m \equiv e * m}
  \and
  \inferrule* [lab=associativity] {} {m_1 * (m_2 * m_3) \equiv (m_1 * m_2) * m_3}
\end{mathpar} 

\paragraph{Logic: the set monad as an algebra}
In a similar manner, there is a language associated with the monad of
sets \emph{considered as an algebra}. This language is very familiar
to most programmers.

\begin{mathpar}
  \inferrule* [lab=expression] {} {{c,d} ::=}
  \and
  \inferrule* [lab=identity verity] {} {true}
  \and
  \inferrule* [lab=negation] {} {\;| \; \neg c}
  \and
  \inferrule* [lab=conjunction] {} {\;| \; c \& d}
\end{mathpar} 

Now, if we had a specific set in hand, say $L$ (which we'll call a
universe in the sequel), we can interpret the expressions in the this
language, aka formulae, in terms of operations on subsets of that
set. As with our compiler for the concrete syntax of the
$lambda$-calculus in chapter 1, we can express this translation very
compactly as

\begin{mathpar}
  \inferrule* {} {\meaningof{true} = L}
  \and
  \inferrule* {} {\meaningof{\neg c} = L \backslash c}
  \and 
  \inferrule* {} {\meaningof{c \& d} = \meaningof{c} \cap \meaningof{d}}
\end{mathpar}

Now, what's happening when we pull the monoid monad through the set
monad via a distributive map is this. First, the monoid monad
furnishes the universe, $L$, as the set of expressions generated by
the grammar. We'll denote this by $L(m)$. Then, we enrich the set of
formulae by the operations of the monoid \emph{acting on sets}.

\begin{mathpar}
  \inferrule* [lab=expression] {} {{c,d} ::=}
  \and
  \inferrule* [lab=identity verity] {} {true}
  \and
  \inferrule* [lab=negation] {} {\;| \; \neg c}
  \and
  \inferrule* [lab=conjunction] {} {\;| \; c \& d}
  \and
  \inferrule* [lab=identity verity] {} {\bf{e}}
  \and
  \inferrule* [lab=negation] {} {\;| \; \bf{g_1} \; | \; ... \; | \; \bf{g_n}}
  \and
  \inferrule* [lab=conjunction] {} {\;| \; c * d}
\end{mathpar} 

The identity element, $e$ and the generators of the monoid, $g_1$,
..., $g_n$, can be considered $0$-ary operations in the same way that
we usually consider constants as $0$-ary operations. To avoid
confusion between these elements and the \emph{logical formulae} that
pick them out of the crowd, we write the logical formulae in
$\bf{boldface}$.

Now, we can write our distributive map. Surprisingly, it is exactly a
meaning for our logic!

\begin{mathpar}
  \inferrule* {} {\meaningof{true} = L(m)}
  \and
  \inferrule* {} {\meaningof{\neg c} = L(m) \backslash c}
  \and 
  \inferrule* {} {\meaningof{c \& d} = \meaningof{c} \cap \meaningof{d}}
  \and
  \inferrule* {} {\meaningof{\bf{e}} = \{ m \; \in \; L(m) \; | \; m \equiv e \}}
  \and
  \inferrule* {} {\meaningof{\bf{g_i}} = \{ m \; \in \; L(m) \; | \; m \equiv g_i \}}
  \and
  \inferrule* {} {\meaningof{c*d} = \{ m \; \in \; L(m) \; | \; m \equiv m_1 * m_2, m_1 \; \in \; \meaningof{c}, m_2 \; \in \; \meaningof{d} \}}
\end{mathpar}

\paragraph{Primes: an application}
Before going any further, let's look at an example of how to use these
new operators. Suppose we wanted to pick out all the elements of the
monoid that were not expressible as a composition of other
elements. Obviously, for monoids with a finite set of generators, this
is exactly just the generators, so we could write $\bf{g_1} || ... ||
\bf{g_n}$\footnote{We get the disjunction, $||$, by the usual DeMorgan
  translation: $c || d \stackrel{def}{=} \neg( \neg c \& \neg
  d)$}. However, when the set of generators is not finite, as it is
when the monoid is the integers under multiplication, we need another
way to write this down. That's where our other operators come in
handy. A moment's thought suggests that we could say that since $true$
denotes any possible element in the monoid, an element is not a
composition using negation plus our composition formula, i.e. $\neg
(true * true)$. This is a little overkill, however. We just want to
eliminate non-trivial compositions. We know how to express the
identity element, that's $\bf{e}$, so we are interested in those
elements that are not the identity, i.e. $\neg \bf{e}$. Then a formula
that eliminates compositions of non-trivial elements is spelled out
$\neg (\neg e * \neg e)$. Finally, we want to eliminate the identity
as a solution. So, we arrive at $\neg (\neg e * \neg e) \& \neg
e$. There, that formula picks out the \emph{primes} of \emph{any}
monoid.

\paragraph{Summary}

What have we done? We've illustrated a specific distributive map, one
that pulls the set monad through the monoid monad. We've shown that
this particular distributive map coincides with giving a semantics to
a particular logic, one whose structure is derived solely from the
shape of the collection monad, i.e. set, and the shape of the term
language, in this case monoid.

\subsubsection{Iterating the design pattern}

The whole point of working in this manner is that by virtue of its
compositional structure it provides a much higher level of abstraction
and greater opportunities for reuse. To illustrate the point, we will
now iterate the construction using our toy language, the
$lambda$-calculus, as the term language. As we saw in chapter 1, the
$lambda$-calculus also has a generators and relations
presentation. Unlike a monoid, however, the lambda calculus has
another piece of machinery: reduction! In addition to structural
equivalence of terms (which is a bi-directional relation) there is the
$beta$-reduction rule that captures the \emph{behavioral} aspect of
the lambda calculus.

It is key to understand this underlying structure of language
definitions. In essence, when a DSL is purely about structure it is
presented entirely in terms of generators (read a grammar) and
relations (like the monoid laws). When the DSL is also about behavior,
i.e. the terms in the language somehow express some kind of
computation, then the language has a third component, some kind of
reduction relation. \footnote{In some sense this is one of the central
  contributions of the theory of computation back to
  mathematics. Algebraists have known for a long time about generators
  and relations presentations of algebraic structures (of which
  algebraic data types are a subset). This collective wisdom is
  studied, for example, in the field of universal
  algebra. Computational models like the $lambda$-calculus and more
  recently the process calculi, like Milner's $\pi$-calculus or
  Cardelli and Gordon's ambient calculus, take this presentation one
  step further and add a set of conditional rewrite rules to express
  the computational content of the model. It was Milner who first
  recognized this particular decomposition of language definitions in
  his seminal paper, Functions as Processes, where he reformulated the
  presentation $\pi$-calculus along these lines.} This organization,
this common factoring of the specification of a language, makes it
possible to factor code that handles a wide range of semantic
features. The logic we derive below provides a great example.

\section{Searching for programs}

TBD

% \section{Existence problems}
% We begin with some metamathematics.
% All problems about the existence of maps can be cast into one of the
% following two forms, which are in a sense mutually dual.

% \noindent
% {\bf The Extension Problem}\index{extension problem} \    %%% NB index entry tag
% Given an inclusion $ A \stackrel{i}{\hookrightarrow} X $, and a map
% $ A \stackrel{f}{\rightarrow} Y $,
% does there exist a map $f^{\dagger}:X\to Y$ such that
% $f^{\dagger}$ agrees with $f$ on $A$?

% Here the appropriate source category for maps should be clear from the
% context and, moreover, commutativity through a
% candidate $f^{\dagger}$ is precisely
% the restriction requirement; that is,
% $$f^{\dagger}   :  f^{\dagger}\circ i = f^{\dagger}|_A = f\,. $$
% If such an $f^{\dagger}$ exists\footnote{${}^{\dagger}$ suggests striving
% for perfection, crusading}, then it is called an {\bf
% extension}\index{extension!of a map|bi} of $f$ and is said to {\bf
% extend}\index{extend|bi} $f$. In any diagrams, the presence of
% a dotted arrow or an arrow carrying a ? indicates a pious hope, in no way
% begging the question of its existence. Note that we shall usually
% omit $\circ$ from composite maps.

% \noindent
% {\bf The Lifting Problem}\index{lifting problem} \
% Given a pair of maps $E \stackrel{p}{\rightarrow}B$ and $X \stackrel{f}
% {\rightarrow} B $,
% does there exist a map $f^{\circ} : X \to E$, with
% $pf^{\circ} = f  $?


% That {\em all\/} existence problems about maps are essentially of one
% type or
% the other from these two is seen as follows. Evidently, all existence problems
% are representable by triangular diagrams\index{triangular diagrams} and it
% is easily seen that there are only these six possibilities:
% \begin{center}\begin{picture}(300,70)  %augch2 75
% \put(5,60){\vector(1,0){30}}
% \put(55,60){\vector(1,0){30}}
% \put(135,60){\vector(-1,0){30}}
% \put(185,60){\vector(-1,0){30}}
% \put(235,60){\vector(-1,0){30}}
% \put(285,60){\vector(-1,0){30}}
% \put(0,55){\vector(0,-1){30}}
% \put(50,55){\vector(0,-1){30}}
% \put(100,25){\vector(0,1){30}}
% \put(150,25){\vector(0,1){30}}
% \put(200,55){\vector(0,-1){30}}
% \put(250,55){\vector(0,-1){30}}
% \put(28,33){\small ?}
% \put(78,33){\small ?}
% \put(128,33){\small ?}
% \put(178,33){\small ?}
% \put(228,33){\small ?}
% \put(278,33){\small ?}
% \put(10,3){\bf 1}
% \put(60,3){\bf 2}
% \put(110,3){\bf 3}
% \put(160,3){\bf 4}
% \put(210,3){\bf 5}
% \put(260,3){\bf 6}
% \put(35,55){\vector(-1,-1){30}}
% \put(155,25){\vector(1,1){30}}
% \put(135,55){\vector(-1,-1){30}}
% \put(55,25){\vector(1,1){30}}
% \put(235,55){\vector(-1,-1){30}}
% \put(255,25){\vector(1,1){30}}
% \end{picture}\end{center}



% \begin{figure}
% \begin{picture}(300,220)(0,0)
% \put(-20,-20){\resizebox{20 cm}{!}{\includegraphics{3dpdf}}}
% \put(260,-10){\resizebox{15 cm}{!}{\includegraphics{contpdf}}}
% \put(220,80){$\beta$}
% \put(400,-10){$N$}
% \put(260,170){$\beta$}
% \put(90,15){$N$}
% \end{picture}
% \caption{{\em The log-gamma family of densities with central mean
% $<N> \, = \frac{1}{2}$ as a surface and as a contour plot. }}
% \label{pdf}
% \end{figure}

\newpage

%ch.tex


\chapter{The semantic web}
\begin{center}
{\small\em Where are we; how did we get here; and where are we going?}
\end{center}

TBD
\section{How our web framework enables different kinds of application queries}

\subsubsection{An alternative presentation}

If you recall, there's an alternative way to present monads that are
algebras, like our monoid monad. Algebras are presented in terms of
generators and relations. In our case the generators presentation is
really just a grammar for monoid expressions.

\begin{mathpar}
  \inferrule* [lab=expression] {} {{m,n} ::=}
  \and
  \inferrule* [lab=identity element] {} {e}
  \and
  \inferrule* [lab=generators] {} {\;| \; g_1 \; | \; ... \; | \; g_n}
  \and
  \inferrule* [lab=monoid-multiplication] {} {\;| \; m * n}
\end{mathpar} 

This is subject to the following constraints, meaning that we will
treat syntactic expressions of certain forms as denoting the same
element of the monoid. To emphasize the nearly purely syntactic role
of these constraints we will use a different symbol for the
constraints. We also use the same symbol, $\equiv$, for the smallest equivalence
relation respecting these constraints.

\begin{mathpar}
  \inferrule* [lab=identity laws] {} {m * e \equiv m \equiv e * m}
  \and
  \inferrule* [lab=associativity] {} {m_1 * (m_2 * m_3) \equiv (m_1 * m_2) * m_3}
\end{mathpar} 

\paragraph{Logic: the set monad as an algebra}
In a similar manner, there is a language associated with the monad of
sets \emph{considered as an algebra}. This language is very familiar
to most programmers.

\begin{mathpar}
  \inferrule* [lab=expression] {} {{c,d} ::=}
  \and
  \inferrule* [lab=identity verity] {} {true}
  \and
  \inferrule* [lab=negation] {} {\;| \; \neg c}
  \and
  \inferrule* [lab=conjunction] {} {\;| \; c \& d}
\end{mathpar} 

Now, if we had a specific set in hand, say $L$ (which we'll call a
universe in the sequel), we can interpret the expressions in the this
language, aka formulae, in terms of operations on subsets of that
set. As with our compiler for the concrete syntax of the
$lambda$-calculus in chapter 1, we can express this translation very
compactly as

\begin{mathpar}
  \inferrule* {} {\meaningof{true} = L}
  \and
  \inferrule* {} {\meaningof{\neg c} = L \backslash c}
  \and 
  \inferrule* {} {\meaningof{c \& d} = \meaningof{c} \cap \meaningof{d}}
\end{mathpar}

Now, what's happening when we pull the monoid monad through the set
monad via a distributive map is this. First, the monoid monad
furnishes the universe, $L$, as the set of expressions generated by
the grammar. We'll denote this by $L(m)$. Then, we enrich the set of
formulae by the operations of the monoid \emph{acting on sets}.

\begin{mathpar}
  \inferrule* [lab=expression] {} {{c,d} ::=}
  \and
  \inferrule* [lab=identity verity] {} {true}
  \and
  \inferrule* [lab=negation] {} {\;| \; \neg c}
  \and
  \inferrule* [lab=conjunction] {} {\;| \; c \& d}
  \and
  \inferrule* [lab=identity verity] {} {\bf{e}}
  \and
  \inferrule* [lab=negation] {} {\;| \; \bf{g_1} \; | \; ... \; | \; \bf{g_n}}
  \and
  \inferrule* [lab=conjunction] {} {\;| \; c * d}
\end{mathpar} 

The identity element, $e$ and the generators of the monoid, $g_1$,
..., $g_n$, can be considered $0$-ary operations in the same way that
we usually consider constants as $0$-ary operations. To avoid
confusion between these elements and the \emph{logical formulae} that
pick them out of the crowd, we write the logical formulae in
$\bf{boldface}$.

Now, we can write our distributive map. Surprisingly, it is exactly a
meaning for our logic!

\begin{mathpar}
  \inferrule* {} {\meaningof{true} = L(m)}
  \and
  \inferrule* {} {\meaningof{\neg c} = L(m) \backslash c}
  \and 
  \inferrule* {} {\meaningof{c \& d} = \meaningof{c} \cap \meaningof{d}}
  \and
  \inferrule* {} {\meaningof{\bf{e}} = \{ m \; \in \; L(m) \; | \; m \equiv e \}}
  \and
  \inferrule* {} {\meaningof{\bf{g_i}} = \{ m \; \in \; L(m) \; | \; m \equiv g_i \}}
  \and
  \inferrule* {} {\meaningof{c*d} = \{ m \; \in \; L(m) \; | \; m \equiv m_1 * m_2, m_1 \; \in \; \meaningof{c}, m_2 \; \in \; \meaningof{d} \}}
\end{mathpar}

\paragraph{Primes: an application}
Before going any further, let's look at an example of how to use these
new operators. Suppose we wanted to pick out all the elements of the
monoid that were not expressible as a composition of other
elements. Obviously, for monoids with a finite set of generators, this
is exactly just the generators, so we could write $\bf{g_1} || ... ||
\bf{g_n}$\footnote{We get the disjunction, $||$, by the usual DeMorgan
  translation: $c || d \stackrel{def}{=} \neg( \neg c \& \neg
  d)$}. However, when the set of generators is not finite, as it is
when the monoid is the integers under multiplication, we need another
way to write this down. That's where our other operators come in
handy. A moment's thought suggests that we could say that since $true$
denotes any possible element in the monoid, an element is not a
composition using negation plus our composition formula, i.e. $\neg
(true * true)$. This is a little overkill, however. We just want to
eliminate non-trivial compositions. We know how to express the
identity element, that's $\bf{e}$, so we are interested in those
elements that are not the identity, i.e. $\neg \bf{e}$. Then a formula
that eliminates compositions of non-trivial elements is spelled out
$\neg (\neg e * \neg e)$. Finally, we want to eliminate the identity
as a solution. So, we arrive at $\neg (\neg e * \neg e) \& \neg
e$. There, that formula picks out the \emph{primes} of \emph{any}
monoid.

\paragraph{Summary}

What have we done? We've illustrated a specific distributive map, one
that pulls the set monad through the monoid monad. We've shown that
this particular distributive map coincides with giving a semantics to
a particular logic, one whose structure is derived solely from the
shape of the collection monad, i.e. set, and the shape of the term
language, in this case monoid.

\subsubsection{Iterating the design pattern}

The whole point of working in this manner is that by virtue of its
compositional structure it provides a much higher level of abstraction
and greater opportunities for reuse. To illustrate the point, we will
now iterate the construction using our toy language, the
$lambda$-calculus, as the term language. As we saw in chapter 1, the
$lambda$-calculus also has a generators and relations
presentation. Unlike a monoid, however, the lambda calculus has
another piece of machinery: reduction! In addition to structural
equivalence of terms (which is a bi-directional relation) there is the
$beta$-reduction rule that captures the \emph{behavioral} aspect of
the lambda calculus.

It is key to understand this underlying structure of language
definitions. In essence, when a DSL is purely about structure it is
presented entirely in terms of generators (read a grammar) and
relations (like the monoid laws). When the DSL is also about behavior,
i.e. the terms in the language somehow express some kind of
computation, then the language has a third component, some kind of
reduction relation. \footnote{In some sense this is one of the central
  contributions of the theory of computation back to
  mathematics. Algebraists have known for a long time about generators
  and relations presentations of algebraic structures (of which
  algebraic data types are a subset). This collective wisdom is
  studied, for example, in the field of universal
  algebra. Computational models like the $lambda$-calculus and more
  recently the process calculi, like Milner's $\pi$-calculus or
  Cardelli and Gordon's ambient calculus, take this presentation one
  step further and add a set of conditional rewrite rules to express
  the computational content of the model. It was Milner who first
  recognized this particular decomposition of language definitions in
  his seminal paper, Functions as Processes, where he reformulated the
  presentation $\pi$-calculus along these lines.} This organization,
this common factoring of the specification of a language, makes it
possible to factor code that handles a wide range of semantic
features. The logic we derive below provides a great example.

\section{Searching for programs}

TBD

% \section{Existence problems}
% We begin with some metamathematics.
% All problems about the existence of maps can be cast into one of the
% following two forms, which are in a sense mutually dual.

% \noindent
% {\bf The Extension Problem}\index{extension problem} \    %%% NB index entry tag
% Given an inclusion $ A \stackrel{i}{\hookrightarrow} X $, and a map
% $ A \stackrel{f}{\rightarrow} Y $,
% does there exist a map $f^{\dagger}:X\to Y$ such that
% $f^{\dagger}$ agrees with $f$ on $A$?

% Here the appropriate source category for maps should be clear from the
% context and, moreover, commutativity through a
% candidate $f^{\dagger}$ is precisely
% the restriction requirement; that is,
% $$f^{\dagger}   :  f^{\dagger}\circ i = f^{\dagger}|_A = f\,. $$
% If such an $f^{\dagger}$ exists\footnote{${}^{\dagger}$ suggests striving
% for perfection, crusading}, then it is called an {\bf
% extension}\index{extension!of a map|bi} of $f$ and is said to {\bf
% extend}\index{extend|bi} $f$. In any diagrams, the presence of
% a dotted arrow or an arrow carrying a ? indicates a pious hope, in no way
% begging the question of its existence. Note that we shall usually
% omit $\circ$ from composite maps.

% \noindent
% {\bf The Lifting Problem}\index{lifting problem} \
% Given a pair of maps $E \stackrel{p}{\rightarrow}B$ and $X \stackrel{f}
% {\rightarrow} B $,
% does there exist a map $f^{\circ} : X \to E$, with
% $pf^{\circ} = f  $?


% That {\em all\/} existence problems about maps are essentially of one
% type or
% the other from these two is seen as follows. Evidently, all existence problems
% are representable by triangular diagrams\index{triangular diagrams} and it
% is easily seen that there are only these six possibilities:
% \begin{center}\begin{picture}(300,70)  %augch2 75
% \put(5,60){\vector(1,0){30}}
% \put(55,60){\vector(1,0){30}}
% \put(135,60){\vector(-1,0){30}}
% \put(185,60){\vector(-1,0){30}}
% \put(235,60){\vector(-1,0){30}}
% \put(285,60){\vector(-1,0){30}}
% \put(0,55){\vector(0,-1){30}}
% \put(50,55){\vector(0,-1){30}}
% \put(100,25){\vector(0,1){30}}
% \put(150,25){\vector(0,1){30}}
% \put(200,55){\vector(0,-1){30}}
% \put(250,55){\vector(0,-1){30}}
% \put(28,33){\small ?}
% \put(78,33){\small ?}
% \put(128,33){\small ?}
% \put(178,33){\small ?}
% \put(228,33){\small ?}
% \put(278,33){\small ?}
% \put(10,3){\bf 1}
% \put(60,3){\bf 2}
% \put(110,3){\bf 3}
% \put(160,3){\bf 4}
% \put(210,3){\bf 5}
% \put(260,3){\bf 6}
% \put(35,55){\vector(-1,-1){30}}
% \put(155,25){\vector(1,1){30}}
% \put(135,55){\vector(-1,-1){30}}
% \put(55,25){\vector(1,1){30}}
% \put(235,55){\vector(-1,-1){30}}
% \put(255,25){\vector(1,1){30}}
% \end{picture}\end{center}



% \begin{figure}
% \begin{picture}(300,220)(0,0)
% \put(-20,-20){\resizebox{20 cm}{!}{\includegraphics{3dpdf}}}
% \put(260,-10){\resizebox{15 cm}{!}{\includegraphics{contpdf}}}
% \put(220,80){$\beta$}
% \put(400,-10){$N$}
% \put(260,170){$\beta$}
% \put(90,15){$N$}
% \end{picture}
% \caption{{\em The log-gamma family of densities with central mean
% $<N> \, = \frac{1}{2}$ as a surface and as a contour plot. }}
% \label{pdf}
% \end{figure}

\newpage

%ch.tex


\chapter{The semantic web}
\begin{center}
{\small\em Where are we; how did we get here; and where are we going?}
\end{center}

TBD
\section{How our web framework enables different kinds of application queries}

\subsubsection{An alternative presentation}

If you recall, there's an alternative way to present monads that are
algebras, like our monoid monad. Algebras are presented in terms of
generators and relations. In our case the generators presentation is
really just a grammar for monoid expressions.

\begin{mathpar}
  \inferrule* [lab=expression] {} {{m,n} ::=}
  \and
  \inferrule* [lab=identity element] {} {e}
  \and
  \inferrule* [lab=generators] {} {\;| \; g_1 \; | \; ... \; | \; g_n}
  \and
  \inferrule* [lab=monoid-multiplication] {} {\;| \; m * n}
\end{mathpar} 

This is subject to the following constraints, meaning that we will
treat syntactic expressions of certain forms as denoting the same
element of the monoid. To emphasize the nearly purely syntactic role
of these constraints we will use a different symbol for the
constraints. We also use the same symbol, $\equiv$, for the smallest equivalence
relation respecting these constraints.

\begin{mathpar}
  \inferrule* [lab=identity laws] {} {m * e \equiv m \equiv e * m}
  \and
  \inferrule* [lab=associativity] {} {m_1 * (m_2 * m_3) \equiv (m_1 * m_2) * m_3}
\end{mathpar} 

\paragraph{Logic: the set monad as an algebra}
In a similar manner, there is a language associated with the monad of
sets \emph{considered as an algebra}. This language is very familiar
to most programmers.

\begin{mathpar}
  \inferrule* [lab=expression] {} {{c,d} ::=}
  \and
  \inferrule* [lab=identity verity] {} {true}
  \and
  \inferrule* [lab=negation] {} {\;| \; \neg c}
  \and
  \inferrule* [lab=conjunction] {} {\;| \; c \& d}
\end{mathpar} 

Now, if we had a specific set in hand, say $L$ (which we'll call a
universe in the sequel), we can interpret the expressions in the this
language, aka formulae, in terms of operations on subsets of that
set. As with our compiler for the concrete syntax of the
$lambda$-calculus in chapter 1, we can express this translation very
compactly as

\begin{mathpar}
  \inferrule* {} {\meaningof{true} = L}
  \and
  \inferrule* {} {\meaningof{\neg c} = L \backslash c}
  \and 
  \inferrule* {} {\meaningof{c \& d} = \meaningof{c} \cap \meaningof{d}}
\end{mathpar}

Now, what's happening when we pull the monoid monad through the set
monad via a distributive map is this. First, the monoid monad
furnishes the universe, $L$, as the set of expressions generated by
the grammar. We'll denote this by $L(m)$. Then, we enrich the set of
formulae by the operations of the monoid \emph{acting on sets}.

\begin{mathpar}
  \inferrule* [lab=expression] {} {{c,d} ::=}
  \and
  \inferrule* [lab=identity verity] {} {true}
  \and
  \inferrule* [lab=negation] {} {\;| \; \neg c}
  \and
  \inferrule* [lab=conjunction] {} {\;| \; c \& d}
  \and
  \inferrule* [lab=identity verity] {} {\bf{e}}
  \and
  \inferrule* [lab=negation] {} {\;| \; \bf{g_1} \; | \; ... \; | \; \bf{g_n}}
  \and
  \inferrule* [lab=conjunction] {} {\;| \; c * d}
\end{mathpar} 

The identity element, $e$ and the generators of the monoid, $g_1$,
..., $g_n$, can be considered $0$-ary operations in the same way that
we usually consider constants as $0$-ary operations. To avoid
confusion between these elements and the \emph{logical formulae} that
pick them out of the crowd, we write the logical formulae in
$\bf{boldface}$.

Now, we can write our distributive map. Surprisingly, it is exactly a
meaning for our logic!

\begin{mathpar}
  \inferrule* {} {\meaningof{true} = L(m)}
  \and
  \inferrule* {} {\meaningof{\neg c} = L(m) \backslash c}
  \and 
  \inferrule* {} {\meaningof{c \& d} = \meaningof{c} \cap \meaningof{d}}
  \and
  \inferrule* {} {\meaningof{\bf{e}} = \{ m \; \in \; L(m) \; | \; m \equiv e \}}
  \and
  \inferrule* {} {\meaningof{\bf{g_i}} = \{ m \; \in \; L(m) \; | \; m \equiv g_i \}}
  \and
  \inferrule* {} {\meaningof{c*d} = \{ m \; \in \; L(m) \; | \; m \equiv m_1 * m_2, m_1 \; \in \; \meaningof{c}, m_2 \; \in \; \meaningof{d} \}}
\end{mathpar}

\paragraph{Primes: an application}
Before going any further, let's look at an example of how to use these
new operators. Suppose we wanted to pick out all the elements of the
monoid that were not expressible as a composition of other
elements. Obviously, for monoids with a finite set of generators, this
is exactly just the generators, so we could write $\bf{g_1} || ... ||
\bf{g_n}$\footnote{We get the disjunction, $||$, by the usual DeMorgan
  translation: $c || d \stackrel{def}{=} \neg( \neg c \& \neg
  d)$}. However, when the set of generators is not finite, as it is
when the monoid is the integers under multiplication, we need another
way to write this down. That's where our other operators come in
handy. A moment's thought suggests that we could say that since $true$
denotes any possible element in the monoid, an element is not a
composition using negation plus our composition formula, i.e. $\neg
(true * true)$. This is a little overkill, however. We just want to
eliminate non-trivial compositions. We know how to express the
identity element, that's $\bf{e}$, so we are interested in those
elements that are not the identity, i.e. $\neg \bf{e}$. Then a formula
that eliminates compositions of non-trivial elements is spelled out
$\neg (\neg e * \neg e)$. Finally, we want to eliminate the identity
as a solution. So, we arrive at $\neg (\neg e * \neg e) \& \neg
e$. There, that formula picks out the \emph{primes} of \emph{any}
monoid.

\paragraph{Summary}

What have we done? We've illustrated a specific distributive map, one
that pulls the set monad through the monoid monad. We've shown that
this particular distributive map coincides with giving a semantics to
a particular logic, one whose structure is derived solely from the
shape of the collection monad, i.e. set, and the shape of the term
language, in this case monoid.

\subsubsection{Iterating the design pattern}

The whole point of working in this manner is that by virtue of its
compositional structure it provides a much higher level of abstraction
and greater opportunities for reuse. To illustrate the point, we will
now iterate the construction using our toy language, the
$lambda$-calculus, as the term language. As we saw in chapter 1, the
$lambda$-calculus also has a generators and relations
presentation. Unlike a monoid, however, the lambda calculus has
another piece of machinery: reduction! In addition to structural
equivalence of terms (which is a bi-directional relation) there is the
$beta$-reduction rule that captures the \emph{behavioral} aspect of
the lambda calculus.

It is key to understand this underlying structure of language
definitions. In essence, when a DSL is purely about structure it is
presented entirely in terms of generators (read a grammar) and
relations (like the monoid laws). When the DSL is also about behavior,
i.e. the terms in the language somehow express some kind of
computation, then the language has a third component, some kind of
reduction relation. \footnote{In some sense this is one of the central
  contributions of the theory of computation back to
  mathematics. Algebraists have known for a long time about generators
  and relations presentations of algebraic structures (of which
  algebraic data types are a subset). This collective wisdom is
  studied, for example, in the field of universal
  algebra. Computational models like the $lambda$-calculus and more
  recently the process calculi, like Milner's $\pi$-calculus or
  Cardelli and Gordon's ambient calculus, take this presentation one
  step further and add a set of conditional rewrite rules to express
  the computational content of the model. It was Milner who first
  recognized this particular decomposition of language definitions in
  his seminal paper, Functions as Processes, where he reformulated the
  presentation $\pi$-calculus along these lines.} This organization,
this common factoring of the specification of a language, makes it
possible to factor code that handles a wide range of semantic
features. The logic we derive below provides a great example.

\section{Searching for programs}

TBD

% \section{Existence problems}
% We begin with some metamathematics.
% All problems about the existence of maps can be cast into one of the
% following two forms, which are in a sense mutually dual.

% \noindent
% {\bf The Extension Problem}\index{extension problem} \    %%% NB index entry tag
% Given an inclusion $ A \stackrel{i}{\hookrightarrow} X $, and a map
% $ A \stackrel{f}{\rightarrow} Y $,
% does there exist a map $f^{\dagger}:X\to Y$ such that
% $f^{\dagger}$ agrees with $f$ on $A$?

% Here the appropriate source category for maps should be clear from the
% context and, moreover, commutativity through a
% candidate $f^{\dagger}$ is precisely
% the restriction requirement; that is,
% $$f^{\dagger}   :  f^{\dagger}\circ i = f^{\dagger}|_A = f\,. $$
% If such an $f^{\dagger}$ exists\footnote{${}^{\dagger}$ suggests striving
% for perfection, crusading}, then it is called an {\bf
% extension}\index{extension!of a map|bi} of $f$ and is said to {\bf
% extend}\index{extend|bi} $f$. In any diagrams, the presence of
% a dotted arrow or an arrow carrying a ? indicates a pious hope, in no way
% begging the question of its existence. Note that we shall usually
% omit $\circ$ from composite maps.

% \noindent
% {\bf The Lifting Problem}\index{lifting problem} \
% Given a pair of maps $E \stackrel{p}{\rightarrow}B$ and $X \stackrel{f}
% {\rightarrow} B $,
% does there exist a map $f^{\circ} : X \to E$, with
% $pf^{\circ} = f  $?


% That {\em all\/} existence problems about maps are essentially of one
% type or
% the other from these two is seen as follows. Evidently, all existence problems
% are representable by triangular diagrams\index{triangular diagrams} and it
% is easily seen that there are only these six possibilities:
% \begin{center}\begin{picture}(300,70)  %augch2 75
% \put(5,60){\vector(1,0){30}}
% \put(55,60){\vector(1,0){30}}
% \put(135,60){\vector(-1,0){30}}
% \put(185,60){\vector(-1,0){30}}
% \put(235,60){\vector(-1,0){30}}
% \put(285,60){\vector(-1,0){30}}
% \put(0,55){\vector(0,-1){30}}
% \put(50,55){\vector(0,-1){30}}
% \put(100,25){\vector(0,1){30}}
% \put(150,25){\vector(0,1){30}}
% \put(200,55){\vector(0,-1){30}}
% \put(250,55){\vector(0,-1){30}}
% \put(28,33){\small ?}
% \put(78,33){\small ?}
% \put(128,33){\small ?}
% \put(178,33){\small ?}
% \put(228,33){\small ?}
% \put(278,33){\small ?}
% \put(10,3){\bf 1}
% \put(60,3){\bf 2}
% \put(110,3){\bf 3}
% \put(160,3){\bf 4}
% \put(210,3){\bf 5}
% \put(260,3){\bf 6}
% \put(35,55){\vector(-1,-1){30}}
% \put(155,25){\vector(1,1){30}}
% \put(135,55){\vector(-1,-1){30}}
% \put(55,25){\vector(1,1){30}}
% \put(235,55){\vector(-1,-1){30}}
% \put(255,25){\vector(1,1){30}}
% \end{picture}\end{center}



% \begin{figure}
% \begin{picture}(300,220)(0,0)
% \put(-20,-20){\resizebox{20 cm}{!}{\includegraphics{3dpdf}}}
% \put(260,-10){\resizebox{15 cm}{!}{\includegraphics{contpdf}}}
% \put(220,80){$\beta$}
% \put(400,-10){$N$}
% \put(260,170){$\beta$}
% \put(90,15){$N$}
% \end{picture}
% \caption{{\em The log-gamma family of densities with central mean
% $<N> \, = \frac{1}{2}$ as a surface and as a contour plot. }}
% \label{pdf}
% \end{figure}

\newpage

%ch.tex


\chapter{The semantic web}
\begin{center}
{\small\em Where are we; how did we get here; and where are we going?}
\end{center}

TBD
\section{How our web framework enables different kinds of application queries}

\subsubsection{An alternative presentation}

If you recall, there's an alternative way to present monads that are
algebras, like our monoid monad. Algebras are presented in terms of
generators and relations. In our case the generators presentation is
really just a grammar for monoid expressions.

\begin{mathpar}
  \inferrule* [lab=expression] {} {{m,n} ::=}
  \and
  \inferrule* [lab=identity element] {} {e}
  \and
  \inferrule* [lab=generators] {} {\;| \; g_1 \; | \; ... \; | \; g_n}
  \and
  \inferrule* [lab=monoid-multiplication] {} {\;| \; m * n}
\end{mathpar} 

This is subject to the following constraints, meaning that we will
treat syntactic expressions of certain forms as denoting the same
element of the monoid. To emphasize the nearly purely syntactic role
of these constraints we will use a different symbol for the
constraints. We also use the same symbol, $\equiv$, for the smallest equivalence
relation respecting these constraints.

\begin{mathpar}
  \inferrule* [lab=identity laws] {} {m * e \equiv m \equiv e * m}
  \and
  \inferrule* [lab=associativity] {} {m_1 * (m_2 * m_3) \equiv (m_1 * m_2) * m_3}
\end{mathpar} 

\paragraph{Logic: the set monad as an algebra}
In a similar manner, there is a language associated with the monad of
sets \emph{considered as an algebra}. This language is very familiar
to most programmers.

\begin{mathpar}
  \inferrule* [lab=expression] {} {{c,d} ::=}
  \and
  \inferrule* [lab=identity verity] {} {true}
  \and
  \inferrule* [lab=negation] {} {\;| \; \neg c}
  \and
  \inferrule* [lab=conjunction] {} {\;| \; c \& d}
\end{mathpar} 

Now, if we had a specific set in hand, say $L$ (which we'll call a
universe in the sequel), we can interpret the expressions in the this
language, aka formulae, in terms of operations on subsets of that
set. As with our compiler for the concrete syntax of the
$lambda$-calculus in chapter 1, we can express this translation very
compactly as

\begin{mathpar}
  \inferrule* {} {\meaningof{true} = L}
  \and
  \inferrule* {} {\meaningof{\neg c} = L \backslash c}
  \and 
  \inferrule* {} {\meaningof{c \& d} = \meaningof{c} \cap \meaningof{d}}
\end{mathpar}

Now, what's happening when we pull the monoid monad through the set
monad via a distributive map is this. First, the monoid monad
furnishes the universe, $L$, as the set of expressions generated by
the grammar. We'll denote this by $L(m)$. Then, we enrich the set of
formulae by the operations of the monoid \emph{acting on sets}.

\begin{mathpar}
  \inferrule* [lab=expression] {} {{c,d} ::=}
  \and
  \inferrule* [lab=identity verity] {} {true}
  \and
  \inferrule* [lab=negation] {} {\;| \; \neg c}
  \and
  \inferrule* [lab=conjunction] {} {\;| \; c \& d}
  \and
  \inferrule* [lab=identity verity] {} {\bf{e}}
  \and
  \inferrule* [lab=negation] {} {\;| \; \bf{g_1} \; | \; ... \; | \; \bf{g_n}}
  \and
  \inferrule* [lab=conjunction] {} {\;| \; c * d}
\end{mathpar} 

The identity element, $e$ and the generators of the monoid, $g_1$,
..., $g_n$, can be considered $0$-ary operations in the same way that
we usually consider constants as $0$-ary operations. To avoid
confusion between these elements and the \emph{logical formulae} that
pick them out of the crowd, we write the logical formulae in
$\bf{boldface}$.

Now, we can write our distributive map. Surprisingly, it is exactly a
meaning for our logic!

\begin{mathpar}
  \inferrule* {} {\meaningof{true} = L(m)}
  \and
  \inferrule* {} {\meaningof{\neg c} = L(m) \backslash c}
  \and 
  \inferrule* {} {\meaningof{c \& d} = \meaningof{c} \cap \meaningof{d}}
  \and
  \inferrule* {} {\meaningof{\bf{e}} = \{ m \; \in \; L(m) \; | \; m \equiv e \}}
  \and
  \inferrule* {} {\meaningof{\bf{g_i}} = \{ m \; \in \; L(m) \; | \; m \equiv g_i \}}
  \and
  \inferrule* {} {\meaningof{c*d} = \{ m \; \in \; L(m) \; | \; m \equiv m_1 * m_2, m_1 \; \in \; \meaningof{c}, m_2 \; \in \; \meaningof{d} \}}
\end{mathpar}

\paragraph{Primes: an application}
Before going any further, let's look at an example of how to use these
new operators. Suppose we wanted to pick out all the elements of the
monoid that were not expressible as a composition of other
elements. Obviously, for monoids with a finite set of generators, this
is exactly just the generators, so we could write $\bf{g_1} || ... ||
\bf{g_n}$\footnote{We get the disjunction, $||$, by the usual DeMorgan
  translation: $c || d \stackrel{def}{=} \neg( \neg c \& \neg
  d)$}. However, when the set of generators is not finite, as it is
when the monoid is the integers under multiplication, we need another
way to write this down. That's where our other operators come in
handy. A moment's thought suggests that we could say that since $true$
denotes any possible element in the monoid, an element is not a
composition using negation plus our composition formula, i.e. $\neg
(true * true)$. This is a little overkill, however. We just want to
eliminate non-trivial compositions. We know how to express the
identity element, that's $\bf{e}$, so we are interested in those
elements that are not the identity, i.e. $\neg \bf{e}$. Then a formula
that eliminates compositions of non-trivial elements is spelled out
$\neg (\neg e * \neg e)$. Finally, we want to eliminate the identity
as a solution. So, we arrive at $\neg (\neg e * \neg e) \& \neg
e$. There, that formula picks out the \emph{primes} of \emph{any}
monoid.

\paragraph{Summary}

What have we done? We've illustrated a specific distributive map, one
that pulls the set monad through the monoid monad. We've shown that
this particular distributive map coincides with giving a semantics to
a particular logic, one whose structure is derived solely from the
shape of the collection monad, i.e. set, and the shape of the term
language, in this case monoid.

\subsubsection{Iterating the design pattern}

The whole point of working in this manner is that by virtue of its
compositional structure it provides a much higher level of abstraction
and greater opportunities for reuse. To illustrate the point, we will
now iterate the construction using our toy language, the
$lambda$-calculus, as the term language. As we saw in chapter 1, the
$lambda$-calculus also has a generators and relations
presentation. Unlike a monoid, however, the lambda calculus has
another piece of machinery: reduction! In addition to structural
equivalence of terms (which is a bi-directional relation) there is the
$beta$-reduction rule that captures the \emph{behavioral} aspect of
the lambda calculus.

It is key to understand this underlying structure of language
definitions. In essence, when a DSL is purely about structure it is
presented entirely in terms of generators (read a grammar) and
relations (like the monoid laws). When the DSL is also about behavior,
i.e. the terms in the language somehow express some kind of
computation, then the language has a third component, some kind of
reduction relation. \footnote{In some sense this is one of the central
  contributions of the theory of computation back to
  mathematics. Algebraists have known for a long time about generators
  and relations presentations of algebraic structures (of which
  algebraic data types are a subset). This collective wisdom is
  studied, for example, in the field of universal
  algebra. Computational models like the $lambda$-calculus and more
  recently the process calculi, like Milner's $\pi$-calculus or
  Cardelli and Gordon's ambient calculus, take this presentation one
  step further and add a set of conditional rewrite rules to express
  the computational content of the model. It was Milner who first
  recognized this particular decomposition of language definitions in
  his seminal paper, Functions as Processes, where he reformulated the
  presentation $\pi$-calculus along these lines.} This organization,
this common factoring of the specification of a language, makes it
possible to factor code that handles a wide range of semantic
features. The logic we derive below provides a great example.

\section{Searching for programs}

TBD

% \section{Existence problems}
% We begin with some metamathematics.
% All problems about the existence of maps can be cast into one of the
% following two forms, which are in a sense mutually dual.

% \noindent
% {\bf The Extension Problem}\index{extension problem} \    %%% NB index entry tag
% Given an inclusion $ A \stackrel{i}{\hookrightarrow} X $, and a map
% $ A \stackrel{f}{\rightarrow} Y $,
% does there exist a map $f^{\dagger}:X\to Y$ such that
% $f^{\dagger}$ agrees with $f$ on $A$?

% Here the appropriate source category for maps should be clear from the
% context and, moreover, commutativity through a
% candidate $f^{\dagger}$ is precisely
% the restriction requirement; that is,
% $$f^{\dagger}   :  f^{\dagger}\circ i = f^{\dagger}|_A = f\,. $$
% If such an $f^{\dagger}$ exists\footnote{${}^{\dagger}$ suggests striving
% for perfection, crusading}, then it is called an {\bf
% extension}\index{extension!of a map|bi} of $f$ and is said to {\bf
% extend}\index{extend|bi} $f$. In any diagrams, the presence of
% a dotted arrow or an arrow carrying a ? indicates a pious hope, in no way
% begging the question of its existence. Note that we shall usually
% omit $\circ$ from composite maps.

% \noindent
% {\bf The Lifting Problem}\index{lifting problem} \
% Given a pair of maps $E \stackrel{p}{\rightarrow}B$ and $X \stackrel{f}
% {\rightarrow} B $,
% does there exist a map $f^{\circ} : X \to E$, with
% $pf^{\circ} = f  $?


% That {\em all\/} existence problems about maps are essentially of one
% type or
% the other from these two is seen as follows. Evidently, all existence problems
% are representable by triangular diagrams\index{triangular diagrams} and it
% is easily seen that there are only these six possibilities:
% \begin{center}\begin{picture}(300,70)  %augch2 75
% \put(5,60){\vector(1,0){30}}
% \put(55,60){\vector(1,0){30}}
% \put(135,60){\vector(-1,0){30}}
% \put(185,60){\vector(-1,0){30}}
% \put(235,60){\vector(-1,0){30}}
% \put(285,60){\vector(-1,0){30}}
% \put(0,55){\vector(0,-1){30}}
% \put(50,55){\vector(0,-1){30}}
% \put(100,25){\vector(0,1){30}}
% \put(150,25){\vector(0,1){30}}
% \put(200,55){\vector(0,-1){30}}
% \put(250,55){\vector(0,-1){30}}
% \put(28,33){\small ?}
% \put(78,33){\small ?}
% \put(128,33){\small ?}
% \put(178,33){\small ?}
% \put(228,33){\small ?}
% \put(278,33){\small ?}
% \put(10,3){\bf 1}
% \put(60,3){\bf 2}
% \put(110,3){\bf 3}
% \put(160,3){\bf 4}
% \put(210,3){\bf 5}
% \put(260,3){\bf 6}
% \put(35,55){\vector(-1,-1){30}}
% \put(155,25){\vector(1,1){30}}
% \put(135,55){\vector(-1,-1){30}}
% \put(55,25){\vector(1,1){30}}
% \put(235,55){\vector(-1,-1){30}}
% \put(255,25){\vector(1,1){30}}
% \end{picture}\end{center}



% \begin{figure}
% \begin{picture}(300,220)(0,0)
% \put(-20,-20){\resizebox{20 cm}{!}{\includegraphics{3dpdf}}}
% \put(260,-10){\resizebox{15 cm}{!}{\includegraphics{contpdf}}}
% \put(220,80){$\beta$}
% \put(400,-10){$N$}
% \put(260,170){$\beta$}
% \put(90,15){$N$}
% \end{picture}
% \caption{{\em The log-gamma family of densities with central mean
% $<N> \, = \frac{1}{2}$ as a surface and as a contour plot. }}
% \label{pdf}
% \end{figure}

\newpage

%ch.tex


\chapter{The semantic web}
\begin{center}
{\small\em Where are we; how did we get here; and where are we going?}
\end{center}

TBD
\section{How our web framework enables different kinds of application queries}

\subsubsection{An alternative presentation}

If you recall, there's an alternative way to present monads that are
algebras, like our monoid monad. Algebras are presented in terms of
generators and relations. In our case the generators presentation is
really just a grammar for monoid expressions.

\begin{mathpar}
  \inferrule* [lab=expression] {} {{m,n} ::=}
  \and
  \inferrule* [lab=identity element] {} {e}
  \and
  \inferrule* [lab=generators] {} {\;| \; g_1 \; | \; ... \; | \; g_n}
  \and
  \inferrule* [lab=monoid-multiplication] {} {\;| \; m * n}
\end{mathpar} 

This is subject to the following constraints, meaning that we will
treat syntactic expressions of certain forms as denoting the same
element of the monoid. To emphasize the nearly purely syntactic role
of these constraints we will use a different symbol for the
constraints. We also use the same symbol, $\equiv$, for the smallest equivalence
relation respecting these constraints.

\begin{mathpar}
  \inferrule* [lab=identity laws] {} {m * e \equiv m \equiv e * m}
  \and
  \inferrule* [lab=associativity] {} {m_1 * (m_2 * m_3) \equiv (m_1 * m_2) * m_3}
\end{mathpar} 

\paragraph{Logic: the set monad as an algebra}
In a similar manner, there is a language associated with the monad of
sets \emph{considered as an algebra}. This language is very familiar
to most programmers.

\begin{mathpar}
  \inferrule* [lab=expression] {} {{c,d} ::=}
  \and
  \inferrule* [lab=identity verity] {} {true}
  \and
  \inferrule* [lab=negation] {} {\;| \; \neg c}
  \and
  \inferrule* [lab=conjunction] {} {\;| \; c \& d}
\end{mathpar} 

Now, if we had a specific set in hand, say $L$ (which we'll call a
universe in the sequel), we can interpret the expressions in the this
language, aka formulae, in terms of operations on subsets of that
set. As with our compiler for the concrete syntax of the
$lambda$-calculus in chapter 1, we can express this translation very
compactly as

\begin{mathpar}
  \inferrule* {} {\meaningof{true} = L}
  \and
  \inferrule* {} {\meaningof{\neg c} = L \backslash c}
  \and 
  \inferrule* {} {\meaningof{c \& d} = \meaningof{c} \cap \meaningof{d}}
\end{mathpar}

Now, what's happening when we pull the monoid monad through the set
monad via a distributive map is this. First, the monoid monad
furnishes the universe, $L$, as the set of expressions generated by
the grammar. We'll denote this by $L(m)$. Then, we enrich the set of
formulae by the operations of the monoid \emph{acting on sets}.

\begin{mathpar}
  \inferrule* [lab=expression] {} {{c,d} ::=}
  \and
  \inferrule* [lab=identity verity] {} {true}
  \and
  \inferrule* [lab=negation] {} {\;| \; \neg c}
  \and
  \inferrule* [lab=conjunction] {} {\;| \; c \& d}
  \and
  \inferrule* [lab=identity verity] {} {\bf{e}}
  \and
  \inferrule* [lab=negation] {} {\;| \; \bf{g_1} \; | \; ... \; | \; \bf{g_n}}
  \and
  \inferrule* [lab=conjunction] {} {\;| \; c * d}
\end{mathpar} 

The identity element, $e$ and the generators of the monoid, $g_1$,
..., $g_n$, can be considered $0$-ary operations in the same way that
we usually consider constants as $0$-ary operations. To avoid
confusion between these elements and the \emph{logical formulae} that
pick them out of the crowd, we write the logical formulae in
$\bf{boldface}$.

Now, we can write our distributive map. Surprisingly, it is exactly a
meaning for our logic!

\begin{mathpar}
  \inferrule* {} {\meaningof{true} = L(m)}
  \and
  \inferrule* {} {\meaningof{\neg c} = L(m) \backslash c}
  \and 
  \inferrule* {} {\meaningof{c \& d} = \meaningof{c} \cap \meaningof{d}}
  \and
  \inferrule* {} {\meaningof{\bf{e}} = \{ m \; \in \; L(m) \; | \; m \equiv e \}}
  \and
  \inferrule* {} {\meaningof{\bf{g_i}} = \{ m \; \in \; L(m) \; | \; m \equiv g_i \}}
  \and
  \inferrule* {} {\meaningof{c*d} = \{ m \; \in \; L(m) \; | \; m \equiv m_1 * m_2, m_1 \; \in \; \meaningof{c}, m_2 \; \in \; \meaningof{d} \}}
\end{mathpar}

\paragraph{Primes: an application}
Before going any further, let's look at an example of how to use these
new operators. Suppose we wanted to pick out all the elements of the
monoid that were not expressible as a composition of other
elements. Obviously, for monoids with a finite set of generators, this
is exactly just the generators, so we could write $\bf{g_1} || ... ||
\bf{g_n}$\footnote{We get the disjunction, $||$, by the usual DeMorgan
  translation: $c || d \stackrel{def}{=} \neg( \neg c \& \neg
  d)$}. However, when the set of generators is not finite, as it is
when the monoid is the integers under multiplication, we need another
way to write this down. That's where our other operators come in
handy. A moment's thought suggests that we could say that since $true$
denotes any possible element in the monoid, an element is not a
composition using negation plus our composition formula, i.e. $\neg
(true * true)$. This is a little overkill, however. We just want to
eliminate non-trivial compositions. We know how to express the
identity element, that's $\bf{e}$, so we are interested in those
elements that are not the identity, i.e. $\neg \bf{e}$. Then a formula
that eliminates compositions of non-trivial elements is spelled out
$\neg (\neg e * \neg e)$. Finally, we want to eliminate the identity
as a solution. So, we arrive at $\neg (\neg e * \neg e) \& \neg
e$. There, that formula picks out the \emph{primes} of \emph{any}
monoid.

\paragraph{Summary}

What have we done? We've illustrated a specific distributive map, one
that pulls the set monad through the monoid monad. We've shown that
this particular distributive map coincides with giving a semantics to
a particular logic, one whose structure is derived solely from the
shape of the collection monad, i.e. set, and the shape of the term
language, in this case monoid.

\subsubsection{Iterating the design pattern}

The whole point of working in this manner is that by virtue of its
compositional structure it provides a much higher level of abstraction
and greater opportunities for reuse. To illustrate the point, we will
now iterate the construction using our toy language, the
$lambda$-calculus, as the term language. As we saw in chapter 1, the
$lambda$-calculus also has a generators and relations
presentation. Unlike a monoid, however, the lambda calculus has
another piece of machinery: reduction! In addition to structural
equivalence of terms (which is a bi-directional relation) there is the
$beta$-reduction rule that captures the \emph{behavioral} aspect of
the lambda calculus.

It is key to understand this underlying structure of language
definitions. In essence, when a DSL is purely about structure it is
presented entirely in terms of generators (read a grammar) and
relations (like the monoid laws). When the DSL is also about behavior,
i.e. the terms in the language somehow express some kind of
computation, then the language has a third component, some kind of
reduction relation. \footnote{In some sense this is one of the central
  contributions of the theory of computation back to
  mathematics. Algebraists have known for a long time about generators
  and relations presentations of algebraic structures (of which
  algebraic data types are a subset). This collective wisdom is
  studied, for example, in the field of universal
  algebra. Computational models like the $lambda$-calculus and more
  recently the process calculi, like Milner's $\pi$-calculus or
  Cardelli and Gordon's ambient calculus, take this presentation one
  step further and add a set of conditional rewrite rules to express
  the computational content of the model. It was Milner who first
  recognized this particular decomposition of language definitions in
  his seminal paper, Functions as Processes, where he reformulated the
  presentation $\pi$-calculus along these lines.} This organization,
this common factoring of the specification of a language, makes it
possible to factor code that handles a wide range of semantic
features. The logic we derive below provides a great example.

\section{Searching for programs}

TBD

% \section{Existence problems}
% We begin with some metamathematics.
% All problems about the existence of maps can be cast into one of the
% following two forms, which are in a sense mutually dual.

% \noindent
% {\bf The Extension Problem}\index{extension problem} \    %%% NB index entry tag
% Given an inclusion $ A \stackrel{i}{\hookrightarrow} X $, and a map
% $ A \stackrel{f}{\rightarrow} Y $,
% does there exist a map $f^{\dagger}:X\to Y$ such that
% $f^{\dagger}$ agrees with $f$ on $A$?

% Here the appropriate source category for maps should be clear from the
% context and, moreover, commutativity through a
% candidate $f^{\dagger}$ is precisely
% the restriction requirement; that is,
% $$f^{\dagger}   :  f^{\dagger}\circ i = f^{\dagger}|_A = f\,. $$
% If such an $f^{\dagger}$ exists\footnote{${}^{\dagger}$ suggests striving
% for perfection, crusading}, then it is called an {\bf
% extension}\index{extension!of a map|bi} of $f$ and is said to {\bf
% extend}\index{extend|bi} $f$. In any diagrams, the presence of
% a dotted arrow or an arrow carrying a ? indicates a pious hope, in no way
% begging the question of its existence. Note that we shall usually
% omit $\circ$ from composite maps.

% \noindent
% {\bf The Lifting Problem}\index{lifting problem} \
% Given a pair of maps $E \stackrel{p}{\rightarrow}B$ and $X \stackrel{f}
% {\rightarrow} B $,
% does there exist a map $f^{\circ} : X \to E$, with
% $pf^{\circ} = f  $?


% That {\em all\/} existence problems about maps are essentially of one
% type or
% the other from these two is seen as follows. Evidently, all existence problems
% are representable by triangular diagrams\index{triangular diagrams} and it
% is easily seen that there are only these six possibilities:
% \begin{center}\begin{picture}(300,70)  %augch2 75
% \put(5,60){\vector(1,0){30}}
% \put(55,60){\vector(1,0){30}}
% \put(135,60){\vector(-1,0){30}}
% \put(185,60){\vector(-1,0){30}}
% \put(235,60){\vector(-1,0){30}}
% \put(285,60){\vector(-1,0){30}}
% \put(0,55){\vector(0,-1){30}}
% \put(50,55){\vector(0,-1){30}}
% \put(100,25){\vector(0,1){30}}
% \put(150,25){\vector(0,1){30}}
% \put(200,55){\vector(0,-1){30}}
% \put(250,55){\vector(0,-1){30}}
% \put(28,33){\small ?}
% \put(78,33){\small ?}
% \put(128,33){\small ?}
% \put(178,33){\small ?}
% \put(228,33){\small ?}
% \put(278,33){\small ?}
% \put(10,3){\bf 1}
% \put(60,3){\bf 2}
% \put(110,3){\bf 3}
% \put(160,3){\bf 4}
% \put(210,3){\bf 5}
% \put(260,3){\bf 6}
% \put(35,55){\vector(-1,-1){30}}
% \put(155,25){\vector(1,1){30}}
% \put(135,55){\vector(-1,-1){30}}
% \put(55,25){\vector(1,1){30}}
% \put(235,55){\vector(-1,-1){30}}
% \put(255,25){\vector(1,1){30}}
% \end{picture}\end{center}



% \begin{figure}
% \begin{picture}(300,220)(0,0)
% \put(-20,-20){\resizebox{20 cm}{!}{\includegraphics{3dpdf}}}
% \put(260,-10){\resizebox{15 cm}{!}{\includegraphics{contpdf}}}
% \put(220,80){$\beta$}
% \put(400,-10){$N$}
% \put(260,170){$\beta$}
% \put(90,15){$N$}
% \end{picture}
% \caption{{\em The log-gamma family of densities with central mean
% $<N> \, = \frac{1}{2}$ as a surface and as a contour plot. }}
% \label{pdf}
% \end{figure}

\newpage

%ch.tex


\chapter{The semantic web}
\begin{center}
{\small\em Where are we; how did we get here; and where are we going?}
\end{center}

TBD
\section{How our web framework enables different kinds of application queries}

\subsubsection{An alternative presentation}

If you recall, there's an alternative way to present monads that are
algebras, like our monoid monad. Algebras are presented in terms of
generators and relations. In our case the generators presentation is
really just a grammar for monoid expressions.

\begin{mathpar}
  \inferrule* [lab=expression] {} {{m,n} ::=}
  \and
  \inferrule* [lab=identity element] {} {e}
  \and
  \inferrule* [lab=generators] {} {\;| \; g_1 \; | \; ... \; | \; g_n}
  \and
  \inferrule* [lab=monoid-multiplication] {} {\;| \; m * n}
\end{mathpar} 

This is subject to the following constraints, meaning that we will
treat syntactic expressions of certain forms as denoting the same
element of the monoid. To emphasize the nearly purely syntactic role
of these constraints we will use a different symbol for the
constraints. We also use the same symbol, $\equiv$, for the smallest equivalence
relation respecting these constraints.

\begin{mathpar}
  \inferrule* [lab=identity laws] {} {m * e \equiv m \equiv e * m}
  \and
  \inferrule* [lab=associativity] {} {m_1 * (m_2 * m_3) \equiv (m_1 * m_2) * m_3}
\end{mathpar} 

\paragraph{Logic: the set monad as an algebra}
In a similar manner, there is a language associated with the monad of
sets \emph{considered as an algebra}. This language is very familiar
to most programmers.

\begin{mathpar}
  \inferrule* [lab=expression] {} {{c,d} ::=}
  \and
  \inferrule* [lab=identity verity] {} {true}
  \and
  \inferrule* [lab=negation] {} {\;| \; \neg c}
  \and
  \inferrule* [lab=conjunction] {} {\;| \; c \& d}
\end{mathpar} 

Now, if we had a specific set in hand, say $L$ (which we'll call a
universe in the sequel), we can interpret the expressions in the this
language, aka formulae, in terms of operations on subsets of that
set. As with our compiler for the concrete syntax of the
$lambda$-calculus in chapter 1, we can express this translation very
compactly as

\begin{mathpar}
  \inferrule* {} {\meaningof{true} = L}
  \and
  \inferrule* {} {\meaningof{\neg c} = L \backslash c}
  \and 
  \inferrule* {} {\meaningof{c \& d} = \meaningof{c} \cap \meaningof{d}}
\end{mathpar}

Now, what's happening when we pull the monoid monad through the set
monad via a distributive map is this. First, the monoid monad
furnishes the universe, $L$, as the set of expressions generated by
the grammar. We'll denote this by $L(m)$. Then, we enrich the set of
formulae by the operations of the monoid \emph{acting on sets}.

\begin{mathpar}
  \inferrule* [lab=expression] {} {{c,d} ::=}
  \and
  \inferrule* [lab=identity verity] {} {true}
  \and
  \inferrule* [lab=negation] {} {\;| \; \neg c}
  \and
  \inferrule* [lab=conjunction] {} {\;| \; c \& d}
  \and
  \inferrule* [lab=identity verity] {} {\bf{e}}
  \and
  \inferrule* [lab=negation] {} {\;| \; \bf{g_1} \; | \; ... \; | \; \bf{g_n}}
  \and
  \inferrule* [lab=conjunction] {} {\;| \; c * d}
\end{mathpar} 

The identity element, $e$ and the generators of the monoid, $g_1$,
..., $g_n$, can be considered $0$-ary operations in the same way that
we usually consider constants as $0$-ary operations. To avoid
confusion between these elements and the \emph{logical formulae} that
pick them out of the crowd, we write the logical formulae in
$\bf{boldface}$.

Now, we can write our distributive map. Surprisingly, it is exactly a
meaning for our logic!

\begin{mathpar}
  \inferrule* {} {\meaningof{true} = L(m)}
  \and
  \inferrule* {} {\meaningof{\neg c} = L(m) \backslash c}
  \and 
  \inferrule* {} {\meaningof{c \& d} = \meaningof{c} \cap \meaningof{d}}
  \and
  \inferrule* {} {\meaningof{\bf{e}} = \{ m \; \in \; L(m) \; | \; m \equiv e \}}
  \and
  \inferrule* {} {\meaningof{\bf{g_i}} = \{ m \; \in \; L(m) \; | \; m \equiv g_i \}}
  \and
  \inferrule* {} {\meaningof{c*d} = \{ m \; \in \; L(m) \; | \; m \equiv m_1 * m_2, m_1 \; \in \; \meaningof{c}, m_2 \; \in \; \meaningof{d} \}}
\end{mathpar}

\paragraph{Primes: an application}
Before going any further, let's look at an example of how to use these
new operators. Suppose we wanted to pick out all the elements of the
monoid that were not expressible as a composition of other
elements. Obviously, for monoids with a finite set of generators, this
is exactly just the generators, so we could write $\bf{g_1} || ... ||
\bf{g_n}$\footnote{We get the disjunction, $||$, by the usual DeMorgan
  translation: $c || d \stackrel{def}{=} \neg( \neg c \& \neg
  d)$}. However, when the set of generators is not finite, as it is
when the monoid is the integers under multiplication, we need another
way to write this down. That's where our other operators come in
handy. A moment's thought suggests that we could say that since $true$
denotes any possible element in the monoid, an element is not a
composition using negation plus our composition formula, i.e. $\neg
(true * true)$. This is a little overkill, however. We just want to
eliminate non-trivial compositions. We know how to express the
identity element, that's $\bf{e}$, so we are interested in those
elements that are not the identity, i.e. $\neg \bf{e}$. Then a formula
that eliminates compositions of non-trivial elements is spelled out
$\neg (\neg e * \neg e)$. Finally, we want to eliminate the identity
as a solution. So, we arrive at $\neg (\neg e * \neg e) \& \neg
e$. There, that formula picks out the \emph{primes} of \emph{any}
monoid.

\paragraph{Summary}

What have we done? We've illustrated a specific distributive map, one
that pulls the set monad through the monoid monad. We've shown that
this particular distributive map coincides with giving a semantics to
a particular logic, one whose structure is derived solely from the
shape of the collection monad, i.e. set, and the shape of the term
language, in this case monoid.

\subsubsection{Iterating the design pattern}

The whole point of working in this manner is that by virtue of its
compositional structure it provides a much higher level of abstraction
and greater opportunities for reuse. To illustrate the point, we will
now iterate the construction using our toy language, the
$lambda$-calculus, as the term language. As we saw in chapter 1, the
$lambda$-calculus also has a generators and relations
presentation. Unlike a monoid, however, the lambda calculus has
another piece of machinery: reduction! In addition to structural
equivalence of terms (which is a bi-directional relation) there is the
$beta$-reduction rule that captures the \emph{behavioral} aspect of
the lambda calculus.

It is key to understand this underlying structure of language
definitions. In essence, when a DSL is purely about structure it is
presented entirely in terms of generators (read a grammar) and
relations (like the monoid laws). When the DSL is also about behavior,
i.e. the terms in the language somehow express some kind of
computation, then the language has a third component, some kind of
reduction relation. \footnote{In some sense this is one of the central
  contributions of the theory of computation back to
  mathematics. Algebraists have known for a long time about generators
  and relations presentations of algebraic structures (of which
  algebraic data types are a subset). This collective wisdom is
  studied, for example, in the field of universal
  algebra. Computational models like the $lambda$-calculus and more
  recently the process calculi, like Milner's $\pi$-calculus or
  Cardelli and Gordon's ambient calculus, take this presentation one
  step further and add a set of conditional rewrite rules to express
  the computational content of the model. It was Milner who first
  recognized this particular decomposition of language definitions in
  his seminal paper, Functions as Processes, where he reformulated the
  presentation $\pi$-calculus along these lines.} This organization,
this common factoring of the specification of a language, makes it
possible to factor code that handles a wide range of semantic
features. The logic we derive below provides a great example.

\section{Searching for programs}

TBD

% \section{Existence problems}
% We begin with some metamathematics.
% All problems about the existence of maps can be cast into one of the
% following two forms, which are in a sense mutually dual.

% \noindent
% {\bf The Extension Problem}\index{extension problem} \    %%% NB index entry tag
% Given an inclusion $ A \stackrel{i}{\hookrightarrow} X $, and a map
% $ A \stackrel{f}{\rightarrow} Y $,
% does there exist a map $f^{\dagger}:X\to Y$ such that
% $f^{\dagger}$ agrees with $f$ on $A$?

% Here the appropriate source category for maps should be clear from the
% context and, moreover, commutativity through a
% candidate $f^{\dagger}$ is precisely
% the restriction requirement; that is,
% $$f^{\dagger}   :  f^{\dagger}\circ i = f^{\dagger}|_A = f\,. $$
% If such an $f^{\dagger}$ exists\footnote{${}^{\dagger}$ suggests striving
% for perfection, crusading}, then it is called an {\bf
% extension}\index{extension!of a map|bi} of $f$ and is said to {\bf
% extend}\index{extend|bi} $f$. In any diagrams, the presence of
% a dotted arrow or an arrow carrying a ? indicates a pious hope, in no way
% begging the question of its existence. Note that we shall usually
% omit $\circ$ from composite maps.

% \noindent
% {\bf The Lifting Problem}\index{lifting problem} \
% Given a pair of maps $E \stackrel{p}{\rightarrow}B$ and $X \stackrel{f}
% {\rightarrow} B $,
% does there exist a map $f^{\circ} : X \to E$, with
% $pf^{\circ} = f  $?


% That {\em all\/} existence problems about maps are essentially of one
% type or
% the other from these two is seen as follows. Evidently, all existence problems
% are representable by triangular diagrams\index{triangular diagrams} and it
% is easily seen that there are only these six possibilities:
% \begin{center}\begin{picture}(300,70)  %augch2 75
% \put(5,60){\vector(1,0){30}}
% \put(55,60){\vector(1,0){30}}
% \put(135,60){\vector(-1,0){30}}
% \put(185,60){\vector(-1,0){30}}
% \put(235,60){\vector(-1,0){30}}
% \put(285,60){\vector(-1,0){30}}
% \put(0,55){\vector(0,-1){30}}
% \put(50,55){\vector(0,-1){30}}
% \put(100,25){\vector(0,1){30}}
% \put(150,25){\vector(0,1){30}}
% \put(200,55){\vector(0,-1){30}}
% \put(250,55){\vector(0,-1){30}}
% \put(28,33){\small ?}
% \put(78,33){\small ?}
% \put(128,33){\small ?}
% \put(178,33){\small ?}
% \put(228,33){\small ?}
% \put(278,33){\small ?}
% \put(10,3){\bf 1}
% \put(60,3){\bf 2}
% \put(110,3){\bf 3}
% \put(160,3){\bf 4}
% \put(210,3){\bf 5}
% \put(260,3){\bf 6}
% \put(35,55){\vector(-1,-1){30}}
% \put(155,25){\vector(1,1){30}}
% \put(135,55){\vector(-1,-1){30}}
% \put(55,25){\vector(1,1){30}}
% \put(235,55){\vector(-1,-1){30}}
% \put(255,25){\vector(1,1){30}}
% \end{picture}\end{center}



% \begin{figure}
% \begin{picture}(300,220)(0,0)
% \put(-20,-20){\resizebox{20 cm}{!}{\includegraphics{3dpdf}}}
% \put(260,-10){\resizebox{15 cm}{!}{\includegraphics{contpdf}}}
% \put(220,80){$\beta$}
% \put(400,-10){$N$}
% \put(260,170){$\beta$}
% \put(90,15){$N$}
% \end{picture}
% \caption{{\em The log-gamma family of densities with central mean
% $<N> \, = \frac{1}{2}$ as a surface and as a contour plot. }}
% \label{pdf}
% \end{figure}

\newpage

%ch.tex


\chapter{The semantic web}
\begin{center}
{\small\em Where are we; how did we get here; and where are we going?}
\end{center}

TBD
\section{How our web framework enables different kinds of application queries}

\subsubsection{An alternative presentation}

If you recall, there's an alternative way to present monads that are
algebras, like our monoid monad. Algebras are presented in terms of
generators and relations. In our case the generators presentation is
really just a grammar for monoid expressions.

\begin{mathpar}
  \inferrule* [lab=expression] {} {{m,n} ::=}
  \and
  \inferrule* [lab=identity element] {} {e}
  \and
  \inferrule* [lab=generators] {} {\;| \; g_1 \; | \; ... \; | \; g_n}
  \and
  \inferrule* [lab=monoid-multiplication] {} {\;| \; m * n}
\end{mathpar} 

This is subject to the following constraints, meaning that we will
treat syntactic expressions of certain forms as denoting the same
element of the monoid. To emphasize the nearly purely syntactic role
of these constraints we will use a different symbol for the
constraints. We also use the same symbol, $\equiv$, for the smallest equivalence
relation respecting these constraints.

\begin{mathpar}
  \inferrule* [lab=identity laws] {} {m * e \equiv m \equiv e * m}
  \and
  \inferrule* [lab=associativity] {} {m_1 * (m_2 * m_3) \equiv (m_1 * m_2) * m_3}
\end{mathpar} 

\paragraph{Logic: the set monad as an algebra}
In a similar manner, there is a language associated with the monad of
sets \emph{considered as an algebra}. This language is very familiar
to most programmers.

\begin{mathpar}
  \inferrule* [lab=expression] {} {{c,d} ::=}
  \and
  \inferrule* [lab=identity verity] {} {true}
  \and
  \inferrule* [lab=negation] {} {\;| \; \neg c}
  \and
  \inferrule* [lab=conjunction] {} {\;| \; c \& d}
\end{mathpar} 

Now, if we had a specific set in hand, say $L$ (which we'll call a
universe in the sequel), we can interpret the expressions in the this
language, aka formulae, in terms of operations on subsets of that
set. As with our compiler for the concrete syntax of the
$lambda$-calculus in chapter 1, we can express this translation very
compactly as

\begin{mathpar}
  \inferrule* {} {\meaningof{true} = L}
  \and
  \inferrule* {} {\meaningof{\neg c} = L \backslash c}
  \and 
  \inferrule* {} {\meaningof{c \& d} = \meaningof{c} \cap \meaningof{d}}
\end{mathpar}

Now, what's happening when we pull the monoid monad through the set
monad via a distributive map is this. First, the monoid monad
furnishes the universe, $L$, as the set of expressions generated by
the grammar. We'll denote this by $L(m)$. Then, we enrich the set of
formulae by the operations of the monoid \emph{acting on sets}.

\begin{mathpar}
  \inferrule* [lab=expression] {} {{c,d} ::=}
  \and
  \inferrule* [lab=identity verity] {} {true}
  \and
  \inferrule* [lab=negation] {} {\;| \; \neg c}
  \and
  \inferrule* [lab=conjunction] {} {\;| \; c \& d}
  \and
  \inferrule* [lab=identity verity] {} {\bf{e}}
  \and
  \inferrule* [lab=negation] {} {\;| \; \bf{g_1} \; | \; ... \; | \; \bf{g_n}}
  \and
  \inferrule* [lab=conjunction] {} {\;| \; c * d}
\end{mathpar} 

The identity element, $e$ and the generators of the monoid, $g_1$,
..., $g_n$, can be considered $0$-ary operations in the same way that
we usually consider constants as $0$-ary operations. To avoid
confusion between these elements and the \emph{logical formulae} that
pick them out of the crowd, we write the logical formulae in
$\bf{boldface}$.

Now, we can write our distributive map. Surprisingly, it is exactly a
meaning for our logic!

\begin{mathpar}
  \inferrule* {} {\meaningof{true} = L(m)}
  \and
  \inferrule* {} {\meaningof{\neg c} = L(m) \backslash c}
  \and 
  \inferrule* {} {\meaningof{c \& d} = \meaningof{c} \cap \meaningof{d}}
  \and
  \inferrule* {} {\meaningof{\bf{e}} = \{ m \; \in \; L(m) \; | \; m \equiv e \}}
  \and
  \inferrule* {} {\meaningof{\bf{g_i}} = \{ m \; \in \; L(m) \; | \; m \equiv g_i \}}
  \and
  \inferrule* {} {\meaningof{c*d} = \{ m \; \in \; L(m) \; | \; m \equiv m_1 * m_2, m_1 \; \in \; \meaningof{c}, m_2 \; \in \; \meaningof{d} \}}
\end{mathpar}

\paragraph{Primes: an application}
Before going any further, let's look at an example of how to use these
new operators. Suppose we wanted to pick out all the elements of the
monoid that were not expressible as a composition of other
elements. Obviously, for monoids with a finite set of generators, this
is exactly just the generators, so we could write $\bf{g_1} || ... ||
\bf{g_n}$\footnote{We get the disjunction, $||$, by the usual DeMorgan
  translation: $c || d \stackrel{def}{=} \neg( \neg c \& \neg
  d)$}. However, when the set of generators is not finite, as it is
when the monoid is the integers under multiplication, we need another
way to write this down. That's where our other operators come in
handy. A moment's thought suggests that we could say that since $true$
denotes any possible element in the monoid, an element is not a
composition using negation plus our composition formula, i.e. $\neg
(true * true)$. This is a little overkill, however. We just want to
eliminate non-trivial compositions. We know how to express the
identity element, that's $\bf{e}$, so we are interested in those
elements that are not the identity, i.e. $\neg \bf{e}$. Then a formula
that eliminates compositions of non-trivial elements is spelled out
$\neg (\neg e * \neg e)$. Finally, we want to eliminate the identity
as a solution. So, we arrive at $\neg (\neg e * \neg e) \& \neg
e$. There, that formula picks out the \emph{primes} of \emph{any}
monoid.

\paragraph{Summary}

What have we done? We've illustrated a specific distributive map, one
that pulls the set monad through the monoid monad. We've shown that
this particular distributive map coincides with giving a semantics to
a particular logic, one whose structure is derived solely from the
shape of the collection monad, i.e. set, and the shape of the term
language, in this case monoid.

\subsubsection{Iterating the design pattern}

The whole point of working in this manner is that by virtue of its
compositional structure it provides a much higher level of abstraction
and greater opportunities for reuse. To illustrate the point, we will
now iterate the construction using our toy language, the
$lambda$-calculus, as the term language. As we saw in chapter 1, the
$lambda$-calculus also has a generators and relations
presentation. Unlike a monoid, however, the lambda calculus has
another piece of machinery: reduction! In addition to structural
equivalence of terms (which is a bi-directional relation) there is the
$beta$-reduction rule that captures the \emph{behavioral} aspect of
the lambda calculus.

It is key to understand this underlying structure of language
definitions. In essence, when a DSL is purely about structure it is
presented entirely in terms of generators (read a grammar) and
relations (like the monoid laws). When the DSL is also about behavior,
i.e. the terms in the language somehow express some kind of
computation, then the language has a third component, some kind of
reduction relation. \footnote{In some sense this is one of the central
  contributions of the theory of computation back to
  mathematics. Algebraists have known for a long time about generators
  and relations presentations of algebraic structures (of which
  algebraic data types are a subset). This collective wisdom is
  studied, for example, in the field of universal
  algebra. Computational models like the $lambda$-calculus and more
  recently the process calculi, like Milner's $\pi$-calculus or
  Cardelli and Gordon's ambient calculus, take this presentation one
  step further and add a set of conditional rewrite rules to express
  the computational content of the model. It was Milner who first
  recognized this particular decomposition of language definitions in
  his seminal paper, Functions as Processes, where he reformulated the
  presentation $\pi$-calculus along these lines.} This organization,
this common factoring of the specification of a language, makes it
possible to factor code that handles a wide range of semantic
features. The logic we derive below provides a great example.

\section{Searching for programs}

TBD

% \section{Existence problems}
% We begin with some metamathematics.
% All problems about the existence of maps can be cast into one of the
% following two forms, which are in a sense mutually dual.

% \noindent
% {\bf The Extension Problem}\index{extension problem} \    %%% NB index entry tag
% Given an inclusion $ A \stackrel{i}{\hookrightarrow} X $, and a map
% $ A \stackrel{f}{\rightarrow} Y $,
% does there exist a map $f^{\dagger}:X\to Y$ such that
% $f^{\dagger}$ agrees with $f$ on $A$?

% Here the appropriate source category for maps should be clear from the
% context and, moreover, commutativity through a
% candidate $f^{\dagger}$ is precisely
% the restriction requirement; that is,
% $$f^{\dagger}   :  f^{\dagger}\circ i = f^{\dagger}|_A = f\,. $$
% If such an $f^{\dagger}$ exists\footnote{${}^{\dagger}$ suggests striving
% for perfection, crusading}, then it is called an {\bf
% extension}\index{extension!of a map|bi} of $f$ and is said to {\bf
% extend}\index{extend|bi} $f$. In any diagrams, the presence of
% a dotted arrow or an arrow carrying a ? indicates a pious hope, in no way
% begging the question of its existence. Note that we shall usually
% omit $\circ$ from composite maps.

% \noindent
% {\bf The Lifting Problem}\index{lifting problem} \
% Given a pair of maps $E \stackrel{p}{\rightarrow}B$ and $X \stackrel{f}
% {\rightarrow} B $,
% does there exist a map $f^{\circ} : X \to E$, with
% $pf^{\circ} = f  $?


% That {\em all\/} existence problems about maps are essentially of one
% type or
% the other from these two is seen as follows. Evidently, all existence problems
% are representable by triangular diagrams\index{triangular diagrams} and it
% is easily seen that there are only these six possibilities:
% \begin{center}\begin{picture}(300,70)  %augch2 75
% \put(5,60){\vector(1,0){30}}
% \put(55,60){\vector(1,0){30}}
% \put(135,60){\vector(-1,0){30}}
% \put(185,60){\vector(-1,0){30}}
% \put(235,60){\vector(-1,0){30}}
% \put(285,60){\vector(-1,0){30}}
% \put(0,55){\vector(0,-1){30}}
% \put(50,55){\vector(0,-1){30}}
% \put(100,25){\vector(0,1){30}}
% \put(150,25){\vector(0,1){30}}
% \put(200,55){\vector(0,-1){30}}
% \put(250,55){\vector(0,-1){30}}
% \put(28,33){\small ?}
% \put(78,33){\small ?}
% \put(128,33){\small ?}
% \put(178,33){\small ?}
% \put(228,33){\small ?}
% \put(278,33){\small ?}
% \put(10,3){\bf 1}
% \put(60,3){\bf 2}
% \put(110,3){\bf 3}
% \put(160,3){\bf 4}
% \put(210,3){\bf 5}
% \put(260,3){\bf 6}
% \put(35,55){\vector(-1,-1){30}}
% \put(155,25){\vector(1,1){30}}
% \put(135,55){\vector(-1,-1){30}}
% \put(55,25){\vector(1,1){30}}
% \put(235,55){\vector(-1,-1){30}}
% \put(255,25){\vector(1,1){30}}
% \end{picture}\end{center}



% \begin{figure}
% \begin{picture}(300,220)(0,0)
% \put(-20,-20){\resizebox{20 cm}{!}{\includegraphics{3dpdf}}}
% \put(260,-10){\resizebox{15 cm}{!}{\includegraphics{contpdf}}}
% \put(220,80){$\beta$}
% \put(400,-10){$N$}
% \put(260,170){$\beta$}
% \put(90,15){$N$}
% \end{picture}
% \caption{{\em The log-gamma family of densities with central mean
% $<N> \, = \frac{1}{2}$ as a surface and as a contour plot. }}
% \label{pdf}
% \end{figure}

\newpage

%ch.tex


\chapter{The semantic web}
\begin{center}
{\small\em Where are we; how did we get here; and where are we going?}
\end{center}

TBD
\section{How our web framework enables different kinds of application queries}

\subsubsection{An alternative presentation}

If you recall, there's an alternative way to present monads that are
algebras, like our monoid monad. Algebras are presented in terms of
generators and relations. In our case the generators presentation is
really just a grammar for monoid expressions.

\begin{mathpar}
  \inferrule* [lab=expression] {} {{m,n} ::=}
  \and
  \inferrule* [lab=identity element] {} {e}
  \and
  \inferrule* [lab=generators] {} {\;| \; g_1 \; | \; ... \; | \; g_n}
  \and
  \inferrule* [lab=monoid-multiplication] {} {\;| \; m * n}
\end{mathpar} 

This is subject to the following constraints, meaning that we will
treat syntactic expressions of certain forms as denoting the same
element of the monoid. To emphasize the nearly purely syntactic role
of these constraints we will use a different symbol for the
constraints. We also use the same symbol, $\equiv$, for the smallest equivalence
relation respecting these constraints.

\begin{mathpar}
  \inferrule* [lab=identity laws] {} {m * e \equiv m \equiv e * m}
  \and
  \inferrule* [lab=associativity] {} {m_1 * (m_2 * m_3) \equiv (m_1 * m_2) * m_3}
\end{mathpar} 

\paragraph{Logic: the set monad as an algebra}
In a similar manner, there is a language associated with the monad of
sets \emph{considered as an algebra}. This language is very familiar
to most programmers.

\begin{mathpar}
  \inferrule* [lab=expression] {} {{c,d} ::=}
  \and
  \inferrule* [lab=identity verity] {} {true}
  \and
  \inferrule* [lab=negation] {} {\;| \; \neg c}
  \and
  \inferrule* [lab=conjunction] {} {\;| \; c \& d}
\end{mathpar} 

Now, if we had a specific set in hand, say $L$ (which we'll call a
universe in the sequel), we can interpret the expressions in the this
language, aka formulae, in terms of operations on subsets of that
set. As with our compiler for the concrete syntax of the
$lambda$-calculus in chapter 1, we can express this translation very
compactly as

\begin{mathpar}
  \inferrule* {} {\meaningof{true} = L}
  \and
  \inferrule* {} {\meaningof{\neg c} = L \backslash c}
  \and 
  \inferrule* {} {\meaningof{c \& d} = \meaningof{c} \cap \meaningof{d}}
\end{mathpar}

Now, what's happening when we pull the monoid monad through the set
monad via a distributive map is this. First, the monoid monad
furnishes the universe, $L$, as the set of expressions generated by
the grammar. We'll denote this by $L(m)$. Then, we enrich the set of
formulae by the operations of the monoid \emph{acting on sets}.

\begin{mathpar}
  \inferrule* [lab=expression] {} {{c,d} ::=}
  \and
  \inferrule* [lab=identity verity] {} {true}
  \and
  \inferrule* [lab=negation] {} {\;| \; \neg c}
  \and
  \inferrule* [lab=conjunction] {} {\;| \; c \& d}
  \and
  \inferrule* [lab=identity verity] {} {\bf{e}}
  \and
  \inferrule* [lab=negation] {} {\;| \; \bf{g_1} \; | \; ... \; | \; \bf{g_n}}
  \and
  \inferrule* [lab=conjunction] {} {\;| \; c * d}
\end{mathpar} 

The identity element, $e$ and the generators of the monoid, $g_1$,
..., $g_n$, can be considered $0$-ary operations in the same way that
we usually consider constants as $0$-ary operations. To avoid
confusion between these elements and the \emph{logical formulae} that
pick them out of the crowd, we write the logical formulae in
$\bf{boldface}$.

Now, we can write our distributive map. Surprisingly, it is exactly a
meaning for our logic!

\begin{mathpar}
  \inferrule* {} {\meaningof{true} = L(m)}
  \and
  \inferrule* {} {\meaningof{\neg c} = L(m) \backslash c}
  \and 
  \inferrule* {} {\meaningof{c \& d} = \meaningof{c} \cap \meaningof{d}}
  \and
  \inferrule* {} {\meaningof{\bf{e}} = \{ m \; \in \; L(m) \; | \; m \equiv e \}}
  \and
  \inferrule* {} {\meaningof{\bf{g_i}} = \{ m \; \in \; L(m) \; | \; m \equiv g_i \}}
  \and
  \inferrule* {} {\meaningof{c*d} = \{ m \; \in \; L(m) \; | \; m \equiv m_1 * m_2, m_1 \; \in \; \meaningof{c}, m_2 \; \in \; \meaningof{d} \}}
\end{mathpar}

\paragraph{Primes: an application}
Before going any further, let's look at an example of how to use these
new operators. Suppose we wanted to pick out all the elements of the
monoid that were not expressible as a composition of other
elements. Obviously, for monoids with a finite set of generators, this
is exactly just the generators, so we could write $\bf{g_1} || ... ||
\bf{g_n}$\footnote{We get the disjunction, $||$, by the usual DeMorgan
  translation: $c || d \stackrel{def}{=} \neg( \neg c \& \neg
  d)$}. However, when the set of generators is not finite, as it is
when the monoid is the integers under multiplication, we need another
way to write this down. That's where our other operators come in
handy. A moment's thought suggests that we could say that since $true$
denotes any possible element in the monoid, an element is not a
composition using negation plus our composition formula, i.e. $\neg
(true * true)$. This is a little overkill, however. We just want to
eliminate non-trivial compositions. We know how to express the
identity element, that's $\bf{e}$, so we are interested in those
elements that are not the identity, i.e. $\neg \bf{e}$. Then a formula
that eliminates compositions of non-trivial elements is spelled out
$\neg (\neg e * \neg e)$. Finally, we want to eliminate the identity
as a solution. So, we arrive at $\neg (\neg e * \neg e) \& \neg
e$. There, that formula picks out the \emph{primes} of \emph{any}
monoid.

\paragraph{Summary}

What have we done? We've illustrated a specific distributive map, one
that pulls the set monad through the monoid monad. We've shown that
this particular distributive map coincides with giving a semantics to
a particular logic, one whose structure is derived solely from the
shape of the collection monad, i.e. set, and the shape of the term
language, in this case monoid.

\subsubsection{Iterating the design pattern}

The whole point of working in this manner is that by virtue of its
compositional structure it provides a much higher level of abstraction
and greater opportunities for reuse. To illustrate the point, we will
now iterate the construction using our toy language, the
$lambda$-calculus, as the term language. As we saw in chapter 1, the
$lambda$-calculus also has a generators and relations
presentation. Unlike a monoid, however, the lambda calculus has
another piece of machinery: reduction! In addition to structural
equivalence of terms (which is a bi-directional relation) there is the
$beta$-reduction rule that captures the \emph{behavioral} aspect of
the lambda calculus.

It is key to understand this underlying structure of language
definitions. In essence, when a DSL is purely about structure it is
presented entirely in terms of generators (read a grammar) and
relations (like the monoid laws). When the DSL is also about behavior,
i.e. the terms in the language somehow express some kind of
computation, then the language has a third component, some kind of
reduction relation. \footnote{In some sense this is one of the central
  contributions of the theory of computation back to
  mathematics. Algebraists have known for a long time about generators
  and relations presentations of algebraic structures (of which
  algebraic data types are a subset). This collective wisdom is
  studied, for example, in the field of universal
  algebra. Computational models like the $lambda$-calculus and more
  recently the process calculi, like Milner's $\pi$-calculus or
  Cardelli and Gordon's ambient calculus, take this presentation one
  step further and add a set of conditional rewrite rules to express
  the computational content of the model. It was Milner who first
  recognized this particular decomposition of language definitions in
  his seminal paper, Functions as Processes, where he reformulated the
  presentation $\pi$-calculus along these lines.} This organization,
this common factoring of the specification of a language, makes it
possible to factor code that handles a wide range of semantic
features. The logic we derive below provides a great example.

\section{Searching for programs}

TBD

% \section{Existence problems}
% We begin with some metamathematics.
% All problems about the existence of maps can be cast into one of the
% following two forms, which are in a sense mutually dual.

% \noindent
% {\bf The Extension Problem}\index{extension problem} \    %%% NB index entry tag
% Given an inclusion $ A \stackrel{i}{\hookrightarrow} X $, and a map
% $ A \stackrel{f}{\rightarrow} Y $,
% does there exist a map $f^{\dagger}:X\to Y$ such that
% $f^{\dagger}$ agrees with $f$ on $A$?

% Here the appropriate source category for maps should be clear from the
% context and, moreover, commutativity through a
% candidate $f^{\dagger}$ is precisely
% the restriction requirement; that is,
% $$f^{\dagger}   :  f^{\dagger}\circ i = f^{\dagger}|_A = f\,. $$
% If such an $f^{\dagger}$ exists\footnote{${}^{\dagger}$ suggests striving
% for perfection, crusading}, then it is called an {\bf
% extension}\index{extension!of a map|bi} of $f$ and is said to {\bf
% extend}\index{extend|bi} $f$. In any diagrams, the presence of
% a dotted arrow or an arrow carrying a ? indicates a pious hope, in no way
% begging the question of its existence. Note that we shall usually
% omit $\circ$ from composite maps.

% \noindent
% {\bf The Lifting Problem}\index{lifting problem} \
% Given a pair of maps $E \stackrel{p}{\rightarrow}B$ and $X \stackrel{f}
% {\rightarrow} B $,
% does there exist a map $f^{\circ} : X \to E$, with
% $pf^{\circ} = f  $?


% That {\em all\/} existence problems about maps are essentially of one
% type or
% the other from these two is seen as follows. Evidently, all existence problems
% are representable by triangular diagrams\index{triangular diagrams} and it
% is easily seen that there are only these six possibilities:
% \begin{center}\begin{picture}(300,70)  %augch2 75
% \put(5,60){\vector(1,0){30}}
% \put(55,60){\vector(1,0){30}}
% \put(135,60){\vector(-1,0){30}}
% \put(185,60){\vector(-1,0){30}}
% \put(235,60){\vector(-1,0){30}}
% \put(285,60){\vector(-1,0){30}}
% \put(0,55){\vector(0,-1){30}}
% \put(50,55){\vector(0,-1){30}}
% \put(100,25){\vector(0,1){30}}
% \put(150,25){\vector(0,1){30}}
% \put(200,55){\vector(0,-1){30}}
% \put(250,55){\vector(0,-1){30}}
% \put(28,33){\small ?}
% \put(78,33){\small ?}
% \put(128,33){\small ?}
% \put(178,33){\small ?}
% \put(228,33){\small ?}
% \put(278,33){\small ?}
% \put(10,3){\bf 1}
% \put(60,3){\bf 2}
% \put(110,3){\bf 3}
% \put(160,3){\bf 4}
% \put(210,3){\bf 5}
% \put(260,3){\bf 6}
% \put(35,55){\vector(-1,-1){30}}
% \put(155,25){\vector(1,1){30}}
% \put(135,55){\vector(-1,-1){30}}
% \put(55,25){\vector(1,1){30}}
% \put(235,55){\vector(-1,-1){30}}
% \put(255,25){\vector(1,1){30}}
% \end{picture}\end{center}



% \begin{figure}
% \begin{picture}(300,220)(0,0)
% \put(-20,-20){\resizebox{20 cm}{!}{\includegraphics{3dpdf}}}
% \put(260,-10){\resizebox{15 cm}{!}{\includegraphics{contpdf}}}
% \put(220,80){$\beta$}
% \put(400,-10){$N$}
% \put(260,170){$\beta$}
% \put(90,15){$N$}
% \end{picture}
% \caption{{\em The log-gamma family of densities with central mean
% $<N> \, = \frac{1}{2}$ as a surface and as a contour plot. }}
% \label{pdf}
% \end{figure}

\newpage



\documentclass[12pt,leqno]{book}
\usepackage{amsmath,amssymb,amsfonts} % Typical maths resource packages
\usepackage{graphics}                 % Packages to allow inclusion of graphics
\usepackage{color}                    % For creating coloured text and background
\usepackage{hyperref}                 % For creating hyperlinks in cross references
\usepackage{makeidx}                  % For indexing
\usepackage{listings}                 % For code listing
\usepackage{mathpartir}               % For grammars, rules, etc
\usepackage{bcprules}                 % For other kinds of rules

\lstloadlanguages{Scala,Java,Haskell,XML,bash,HTML,SQL}

\parindent 1cm
\parskip 0.2cm
\topmargin 0.2cm
\oddsidemargin 1cm
\evensidemargin 0.5cm
\textwidth 15cm
\textheight 21cm

\newtheorem{theorem}{Theorem}[section]
\newtheorem{proposition}[theorem]{Proposition}
\newtheorem{corollary}[theorem]{Corollary}
\newtheorem{lemma}[theorem]{Lemma}
\newtheorem{remark}[theorem]{Remark}
\newtheorem{definition}[theorem]{Definition}


\def\R{\mathbb{ R}}
\def\S{\mathbb{ S}}
\def\I{\mathbb{ I}}

\def\Scala{\texttt{Scala}}
\def\ScalaCheck{\texttt{ScalaCheck}}
\def\Haskell{\texttt{Haskell}}
\def\XML{\texttt{XML}}


\makeindex


\title{Pro Scala: Monadic Design Patterns for the Web}

\author{L.G. Meredith  \\
{\small\em \copyright \  Draft date \today }}

 \date{ }
\begin{document}
\lstset{language=Haskell}
\maketitle
 \addcontentsline{toc}{chapter}{Contents}
\pagenumbering{roman}
\tableofcontents
\listoffigures
\listoftables
\chapter*{Preface}\normalsize
  \addcontentsline{toc}{chapter}{Preface}
\pagestyle{plain}
% The book root file {\tt bookex.tex} gives a basic example of how to
% use \LaTeX \ for preparation of a book. Note that all
% \LaTeX \ commands begin with a
% backslash.

% Each
% Chapter, Appendix and the Index is made as a {\tt *.tex} file and is
% called in by the {\tt include} command---thus {\tt ch1.tex} is
% the name here of the file containing Chapter~1. The inclusion of any
% particular file can be suppressed by prefixing the line by a
% percent sign.


%  Do not put an {\tt end{document}} command at the end of chapter files;
% just one such command is needed at the end of the book.

% Note the tag used to make an index entry. You may need to consult Lamport's
% book~\cite{lamport} for details of the procedure to make the index input
% file; \LaTeX \ will create a pre-index by listing all the tagged
% items in the file {\tt bookex.idx} then you edit this into
% a {\tt theindex} environment, as {\tt index.tex}.

The book you hold in your hands, Dear Reader, is not at all what you expected...



\pagestyle{headings}
\pagenumbering{arabic}

%ch.tex


\chapter{The semantic web}
\begin{center}
{\small\em Where are we; how did we get here; and where are we going?}
\end{center}

\input{chapters/ten/semantic-web}
\input{chapters/ten/new-queries}
\input{chapters/ten/search-for-behavior}

% \section{Existence problems}
% We begin with some metamathematics.
% All problems about the existence of maps can be cast into one of the
% following two forms, which are in a sense mutually dual.

% \noindent
% {\bf The Extension Problem}\index{extension problem} \    %%% NB index entry tag
% Given an inclusion $ A \stackrel{i}{\hookrightarrow} X $, and a map
% $ A \stackrel{f}{\rightarrow} Y $,
% does there exist a map $f^{\dagger}:X\to Y$ such that
% $f^{\dagger}$ agrees with $f$ on $A$?

% Here the appropriate source category for maps should be clear from the
% context and, moreover, commutativity through a
% candidate $f^{\dagger}$ is precisely
% the restriction requirement; that is,
% $$f^{\dagger}   :  f^{\dagger}\circ i = f^{\dagger}|_A = f\,. $$
% If such an $f^{\dagger}$ exists\footnote{${}^{\dagger}$ suggests striving
% for perfection, crusading}, then it is called an {\bf
% extension}\index{extension!of a map|bi} of $f$ and is said to {\bf
% extend}\index{extend|bi} $f$. In any diagrams, the presence of
% a dotted arrow or an arrow carrying a ? indicates a pious hope, in no way
% begging the question of its existence. Note that we shall usually
% omit $\circ$ from composite maps.

% \noindent
% {\bf The Lifting Problem}\index{lifting problem} \
% Given a pair of maps $E \stackrel{p}{\rightarrow}B$ and $X \stackrel{f}
% {\rightarrow} B $,
% does there exist a map $f^{\circ} : X \to E$, with
% $pf^{\circ} = f  $?


% That {\em all\/} existence problems about maps are essentially of one
% type or
% the other from these two is seen as follows. Evidently, all existence problems
% are representable by triangular diagrams\index{triangular diagrams} and it
% is easily seen that there are only these six possibilities:
% \begin{center}\begin{picture}(300,70)  %augch2 75
% \put(5,60){\vector(1,0){30}}
% \put(55,60){\vector(1,0){30}}
% \put(135,60){\vector(-1,0){30}}
% \put(185,60){\vector(-1,0){30}}
% \put(235,60){\vector(-1,0){30}}
% \put(285,60){\vector(-1,0){30}}
% \put(0,55){\vector(0,-1){30}}
% \put(50,55){\vector(0,-1){30}}
% \put(100,25){\vector(0,1){30}}
% \put(150,25){\vector(0,1){30}}
% \put(200,55){\vector(0,-1){30}}
% \put(250,55){\vector(0,-1){30}}
% \put(28,33){\small ?}
% \put(78,33){\small ?}
% \put(128,33){\small ?}
% \put(178,33){\small ?}
% \put(228,33){\small ?}
% \put(278,33){\small ?}
% \put(10,3){\bf 1}
% \put(60,3){\bf 2}
% \put(110,3){\bf 3}
% \put(160,3){\bf 4}
% \put(210,3){\bf 5}
% \put(260,3){\bf 6}
% \put(35,55){\vector(-1,-1){30}}
% \put(155,25){\vector(1,1){30}}
% \put(135,55){\vector(-1,-1){30}}
% \put(55,25){\vector(1,1){30}}
% \put(235,55){\vector(-1,-1){30}}
% \put(255,25){\vector(1,1){30}}
% \end{picture}\end{center}



% \begin{figure}
% \begin{picture}(300,220)(0,0)
% \put(-20,-20){\resizebox{20 cm}{!}{\includegraphics{3dpdf}}}
% \put(260,-10){\resizebox{15 cm}{!}{\includegraphics{contpdf}}}
% \put(220,80){$\beta$}
% \put(400,-10){$N$}
% \put(260,170){$\beta$}
% \put(90,15){$N$}
% \end{picture}
% \caption{{\em The log-gamma family of densities with central mean
% $<N> \, = \frac{1}{2}$ as a surface and as a contour plot. }}
% \label{pdf}
% \end{figure}

\newpage

%ch.tex


\chapter{The semantic web}
\begin{center}
{\small\em Where are we; how did we get here; and where are we going?}
\end{center}

\input{chapters/ten/semantic-web}
\input{chapters/ten/new-queries}
\input{chapters/ten/search-for-behavior}

% \section{Existence problems}
% We begin with some metamathematics.
% All problems about the existence of maps can be cast into one of the
% following two forms, which are in a sense mutually dual.

% \noindent
% {\bf The Extension Problem}\index{extension problem} \    %%% NB index entry tag
% Given an inclusion $ A \stackrel{i}{\hookrightarrow} X $, and a map
% $ A \stackrel{f}{\rightarrow} Y $,
% does there exist a map $f^{\dagger}:X\to Y$ such that
% $f^{\dagger}$ agrees with $f$ on $A$?

% Here the appropriate source category for maps should be clear from the
% context and, moreover, commutativity through a
% candidate $f^{\dagger}$ is precisely
% the restriction requirement; that is,
% $$f^{\dagger}   :  f^{\dagger}\circ i = f^{\dagger}|_A = f\,. $$
% If such an $f^{\dagger}$ exists\footnote{${}^{\dagger}$ suggests striving
% for perfection, crusading}, then it is called an {\bf
% extension}\index{extension!of a map|bi} of $f$ and is said to {\bf
% extend}\index{extend|bi} $f$. In any diagrams, the presence of
% a dotted arrow or an arrow carrying a ? indicates a pious hope, in no way
% begging the question of its existence. Note that we shall usually
% omit $\circ$ from composite maps.

% \noindent
% {\bf The Lifting Problem}\index{lifting problem} \
% Given a pair of maps $E \stackrel{p}{\rightarrow}B$ and $X \stackrel{f}
% {\rightarrow} B $,
% does there exist a map $f^{\circ} : X \to E$, with
% $pf^{\circ} = f  $?


% That {\em all\/} existence problems about maps are essentially of one
% type or
% the other from these two is seen as follows. Evidently, all existence problems
% are representable by triangular diagrams\index{triangular diagrams} and it
% is easily seen that there are only these six possibilities:
% \begin{center}\begin{picture}(300,70)  %augch2 75
% \put(5,60){\vector(1,0){30}}
% \put(55,60){\vector(1,0){30}}
% \put(135,60){\vector(-1,0){30}}
% \put(185,60){\vector(-1,0){30}}
% \put(235,60){\vector(-1,0){30}}
% \put(285,60){\vector(-1,0){30}}
% \put(0,55){\vector(0,-1){30}}
% \put(50,55){\vector(0,-1){30}}
% \put(100,25){\vector(0,1){30}}
% \put(150,25){\vector(0,1){30}}
% \put(200,55){\vector(0,-1){30}}
% \put(250,55){\vector(0,-1){30}}
% \put(28,33){\small ?}
% \put(78,33){\small ?}
% \put(128,33){\small ?}
% \put(178,33){\small ?}
% \put(228,33){\small ?}
% \put(278,33){\small ?}
% \put(10,3){\bf 1}
% \put(60,3){\bf 2}
% \put(110,3){\bf 3}
% \put(160,3){\bf 4}
% \put(210,3){\bf 5}
% \put(260,3){\bf 6}
% \put(35,55){\vector(-1,-1){30}}
% \put(155,25){\vector(1,1){30}}
% \put(135,55){\vector(-1,-1){30}}
% \put(55,25){\vector(1,1){30}}
% \put(235,55){\vector(-1,-1){30}}
% \put(255,25){\vector(1,1){30}}
% \end{picture}\end{center}



% \begin{figure}
% \begin{picture}(300,220)(0,0)
% \put(-20,-20){\resizebox{20 cm}{!}{\includegraphics{3dpdf}}}
% \put(260,-10){\resizebox{15 cm}{!}{\includegraphics{contpdf}}}
% \put(220,80){$\beta$}
% \put(400,-10){$N$}
% \put(260,170){$\beta$}
% \put(90,15){$N$}
% \end{picture}
% \caption{{\em The log-gamma family of densities with central mean
% $<N> \, = \frac{1}{2}$ as a surface and as a contour plot. }}
% \label{pdf}
% \end{figure}

\newpage

%ch.tex


\chapter{The semantic web}
\begin{center}
{\small\em Where are we; how did we get here; and where are we going?}
\end{center}

\input{chapters/ten/semantic-web}
\input{chapters/ten/new-queries}
\input{chapters/ten/search-for-behavior}

% \section{Existence problems}
% We begin with some metamathematics.
% All problems about the existence of maps can be cast into one of the
% following two forms, which are in a sense mutually dual.

% \noindent
% {\bf The Extension Problem}\index{extension problem} \    %%% NB index entry tag
% Given an inclusion $ A \stackrel{i}{\hookrightarrow} X $, and a map
% $ A \stackrel{f}{\rightarrow} Y $,
% does there exist a map $f^{\dagger}:X\to Y$ such that
% $f^{\dagger}$ agrees with $f$ on $A$?

% Here the appropriate source category for maps should be clear from the
% context and, moreover, commutativity through a
% candidate $f^{\dagger}$ is precisely
% the restriction requirement; that is,
% $$f^{\dagger}   :  f^{\dagger}\circ i = f^{\dagger}|_A = f\,. $$
% If such an $f^{\dagger}$ exists\footnote{${}^{\dagger}$ suggests striving
% for perfection, crusading}, then it is called an {\bf
% extension}\index{extension!of a map|bi} of $f$ and is said to {\bf
% extend}\index{extend|bi} $f$. In any diagrams, the presence of
% a dotted arrow or an arrow carrying a ? indicates a pious hope, in no way
% begging the question of its existence. Note that we shall usually
% omit $\circ$ from composite maps.

% \noindent
% {\bf The Lifting Problem}\index{lifting problem} \
% Given a pair of maps $E \stackrel{p}{\rightarrow}B$ and $X \stackrel{f}
% {\rightarrow} B $,
% does there exist a map $f^{\circ} : X \to E$, with
% $pf^{\circ} = f  $?


% That {\em all\/} existence problems about maps are essentially of one
% type or
% the other from these two is seen as follows. Evidently, all existence problems
% are representable by triangular diagrams\index{triangular diagrams} and it
% is easily seen that there are only these six possibilities:
% \begin{center}\begin{picture}(300,70)  %augch2 75
% \put(5,60){\vector(1,0){30}}
% \put(55,60){\vector(1,0){30}}
% \put(135,60){\vector(-1,0){30}}
% \put(185,60){\vector(-1,0){30}}
% \put(235,60){\vector(-1,0){30}}
% \put(285,60){\vector(-1,0){30}}
% \put(0,55){\vector(0,-1){30}}
% \put(50,55){\vector(0,-1){30}}
% \put(100,25){\vector(0,1){30}}
% \put(150,25){\vector(0,1){30}}
% \put(200,55){\vector(0,-1){30}}
% \put(250,55){\vector(0,-1){30}}
% \put(28,33){\small ?}
% \put(78,33){\small ?}
% \put(128,33){\small ?}
% \put(178,33){\small ?}
% \put(228,33){\small ?}
% \put(278,33){\small ?}
% \put(10,3){\bf 1}
% \put(60,3){\bf 2}
% \put(110,3){\bf 3}
% \put(160,3){\bf 4}
% \put(210,3){\bf 5}
% \put(260,3){\bf 6}
% \put(35,55){\vector(-1,-1){30}}
% \put(155,25){\vector(1,1){30}}
% \put(135,55){\vector(-1,-1){30}}
% \put(55,25){\vector(1,1){30}}
% \put(235,55){\vector(-1,-1){30}}
% \put(255,25){\vector(1,1){30}}
% \end{picture}\end{center}



% \begin{figure}
% \begin{picture}(300,220)(0,0)
% \put(-20,-20){\resizebox{20 cm}{!}{\includegraphics{3dpdf}}}
% \put(260,-10){\resizebox{15 cm}{!}{\includegraphics{contpdf}}}
% \put(220,80){$\beta$}
% \put(400,-10){$N$}
% \put(260,170){$\beta$}
% \put(90,15){$N$}
% \end{picture}
% \caption{{\em The log-gamma family of densities with central mean
% $<N> \, = \frac{1}{2}$ as a surface and as a contour plot. }}
% \label{pdf}
% \end{figure}

\newpage

%ch.tex


\chapter{The semantic web}
\begin{center}
{\small\em Where are we; how did we get here; and where are we going?}
\end{center}

\input{chapters/ten/semantic-web}
\input{chapters/ten/new-queries}
\input{chapters/ten/search-for-behavior}

% \section{Existence problems}
% We begin with some metamathematics.
% All problems about the existence of maps can be cast into one of the
% following two forms, which are in a sense mutually dual.

% \noindent
% {\bf The Extension Problem}\index{extension problem} \    %%% NB index entry tag
% Given an inclusion $ A \stackrel{i}{\hookrightarrow} X $, and a map
% $ A \stackrel{f}{\rightarrow} Y $,
% does there exist a map $f^{\dagger}:X\to Y$ such that
% $f^{\dagger}$ agrees with $f$ on $A$?

% Here the appropriate source category for maps should be clear from the
% context and, moreover, commutativity through a
% candidate $f^{\dagger}$ is precisely
% the restriction requirement; that is,
% $$f^{\dagger}   :  f^{\dagger}\circ i = f^{\dagger}|_A = f\,. $$
% If such an $f^{\dagger}$ exists\footnote{${}^{\dagger}$ suggests striving
% for perfection, crusading}, then it is called an {\bf
% extension}\index{extension!of a map|bi} of $f$ and is said to {\bf
% extend}\index{extend|bi} $f$. In any diagrams, the presence of
% a dotted arrow or an arrow carrying a ? indicates a pious hope, in no way
% begging the question of its existence. Note that we shall usually
% omit $\circ$ from composite maps.

% \noindent
% {\bf The Lifting Problem}\index{lifting problem} \
% Given a pair of maps $E \stackrel{p}{\rightarrow}B$ and $X \stackrel{f}
% {\rightarrow} B $,
% does there exist a map $f^{\circ} : X \to E$, with
% $pf^{\circ} = f  $?


% That {\em all\/} existence problems about maps are essentially of one
% type or
% the other from these two is seen as follows. Evidently, all existence problems
% are representable by triangular diagrams\index{triangular diagrams} and it
% is easily seen that there are only these six possibilities:
% \begin{center}\begin{picture}(300,70)  %augch2 75
% \put(5,60){\vector(1,0){30}}
% \put(55,60){\vector(1,0){30}}
% \put(135,60){\vector(-1,0){30}}
% \put(185,60){\vector(-1,0){30}}
% \put(235,60){\vector(-1,0){30}}
% \put(285,60){\vector(-1,0){30}}
% \put(0,55){\vector(0,-1){30}}
% \put(50,55){\vector(0,-1){30}}
% \put(100,25){\vector(0,1){30}}
% \put(150,25){\vector(0,1){30}}
% \put(200,55){\vector(0,-1){30}}
% \put(250,55){\vector(0,-1){30}}
% \put(28,33){\small ?}
% \put(78,33){\small ?}
% \put(128,33){\small ?}
% \put(178,33){\small ?}
% \put(228,33){\small ?}
% \put(278,33){\small ?}
% \put(10,3){\bf 1}
% \put(60,3){\bf 2}
% \put(110,3){\bf 3}
% \put(160,3){\bf 4}
% \put(210,3){\bf 5}
% \put(260,3){\bf 6}
% \put(35,55){\vector(-1,-1){30}}
% \put(155,25){\vector(1,1){30}}
% \put(135,55){\vector(-1,-1){30}}
% \put(55,25){\vector(1,1){30}}
% \put(235,55){\vector(-1,-1){30}}
% \put(255,25){\vector(1,1){30}}
% \end{picture}\end{center}



% \begin{figure}
% \begin{picture}(300,220)(0,0)
% \put(-20,-20){\resizebox{20 cm}{!}{\includegraphics{3dpdf}}}
% \put(260,-10){\resizebox{15 cm}{!}{\includegraphics{contpdf}}}
% \put(220,80){$\beta$}
% \put(400,-10){$N$}
% \put(260,170){$\beta$}
% \put(90,15){$N$}
% \end{picture}
% \caption{{\em The log-gamma family of densities with central mean
% $<N> \, = \frac{1}{2}$ as a surface and as a contour plot. }}
% \label{pdf}
% \end{figure}

\newpage

%ch.tex


\chapter{The semantic web}
\begin{center}
{\small\em Where are we; how did we get here; and where are we going?}
\end{center}

\input{chapters/ten/semantic-web}
\input{chapters/ten/new-queries}
\input{chapters/ten/search-for-behavior}

% \section{Existence problems}
% We begin with some metamathematics.
% All problems about the existence of maps can be cast into one of the
% following two forms, which are in a sense mutually dual.

% \noindent
% {\bf The Extension Problem}\index{extension problem} \    %%% NB index entry tag
% Given an inclusion $ A \stackrel{i}{\hookrightarrow} X $, and a map
% $ A \stackrel{f}{\rightarrow} Y $,
% does there exist a map $f^{\dagger}:X\to Y$ such that
% $f^{\dagger}$ agrees with $f$ on $A$?

% Here the appropriate source category for maps should be clear from the
% context and, moreover, commutativity through a
% candidate $f^{\dagger}$ is precisely
% the restriction requirement; that is,
% $$f^{\dagger}   :  f^{\dagger}\circ i = f^{\dagger}|_A = f\,. $$
% If such an $f^{\dagger}$ exists\footnote{${}^{\dagger}$ suggests striving
% for perfection, crusading}, then it is called an {\bf
% extension}\index{extension!of a map|bi} of $f$ and is said to {\bf
% extend}\index{extend|bi} $f$. In any diagrams, the presence of
% a dotted arrow or an arrow carrying a ? indicates a pious hope, in no way
% begging the question of its existence. Note that we shall usually
% omit $\circ$ from composite maps.

% \noindent
% {\bf The Lifting Problem}\index{lifting problem} \
% Given a pair of maps $E \stackrel{p}{\rightarrow}B$ and $X \stackrel{f}
% {\rightarrow} B $,
% does there exist a map $f^{\circ} : X \to E$, with
% $pf^{\circ} = f  $?


% That {\em all\/} existence problems about maps are essentially of one
% type or
% the other from these two is seen as follows. Evidently, all existence problems
% are representable by triangular diagrams\index{triangular diagrams} and it
% is easily seen that there are only these six possibilities:
% \begin{center}\begin{picture}(300,70)  %augch2 75
% \put(5,60){\vector(1,0){30}}
% \put(55,60){\vector(1,0){30}}
% \put(135,60){\vector(-1,0){30}}
% \put(185,60){\vector(-1,0){30}}
% \put(235,60){\vector(-1,0){30}}
% \put(285,60){\vector(-1,0){30}}
% \put(0,55){\vector(0,-1){30}}
% \put(50,55){\vector(0,-1){30}}
% \put(100,25){\vector(0,1){30}}
% \put(150,25){\vector(0,1){30}}
% \put(200,55){\vector(0,-1){30}}
% \put(250,55){\vector(0,-1){30}}
% \put(28,33){\small ?}
% \put(78,33){\small ?}
% \put(128,33){\small ?}
% \put(178,33){\small ?}
% \put(228,33){\small ?}
% \put(278,33){\small ?}
% \put(10,3){\bf 1}
% \put(60,3){\bf 2}
% \put(110,3){\bf 3}
% \put(160,3){\bf 4}
% \put(210,3){\bf 5}
% \put(260,3){\bf 6}
% \put(35,55){\vector(-1,-1){30}}
% \put(155,25){\vector(1,1){30}}
% \put(135,55){\vector(-1,-1){30}}
% \put(55,25){\vector(1,1){30}}
% \put(235,55){\vector(-1,-1){30}}
% \put(255,25){\vector(1,1){30}}
% \end{picture}\end{center}



% \begin{figure}
% \begin{picture}(300,220)(0,0)
% \put(-20,-20){\resizebox{20 cm}{!}{\includegraphics{3dpdf}}}
% \put(260,-10){\resizebox{15 cm}{!}{\includegraphics{contpdf}}}
% \put(220,80){$\beta$}
% \put(400,-10){$N$}
% \put(260,170){$\beta$}
% \put(90,15){$N$}
% \end{picture}
% \caption{{\em The log-gamma family of densities with central mean
% $<N> \, = \frac{1}{2}$ as a surface and as a contour plot. }}
% \label{pdf}
% \end{figure}

\newpage

%ch.tex


\chapter{The semantic web}
\begin{center}
{\small\em Where are we; how did we get here; and where are we going?}
\end{center}

\input{chapters/ten/semantic-web}
\input{chapters/ten/new-queries}
\input{chapters/ten/search-for-behavior}

% \section{Existence problems}
% We begin with some metamathematics.
% All problems about the existence of maps can be cast into one of the
% following two forms, which are in a sense mutually dual.

% \noindent
% {\bf The Extension Problem}\index{extension problem} \    %%% NB index entry tag
% Given an inclusion $ A \stackrel{i}{\hookrightarrow} X $, and a map
% $ A \stackrel{f}{\rightarrow} Y $,
% does there exist a map $f^{\dagger}:X\to Y$ such that
% $f^{\dagger}$ agrees with $f$ on $A$?

% Here the appropriate source category for maps should be clear from the
% context and, moreover, commutativity through a
% candidate $f^{\dagger}$ is precisely
% the restriction requirement; that is,
% $$f^{\dagger}   :  f^{\dagger}\circ i = f^{\dagger}|_A = f\,. $$
% If such an $f^{\dagger}$ exists\footnote{${}^{\dagger}$ suggests striving
% for perfection, crusading}, then it is called an {\bf
% extension}\index{extension!of a map|bi} of $f$ and is said to {\bf
% extend}\index{extend|bi} $f$. In any diagrams, the presence of
% a dotted arrow or an arrow carrying a ? indicates a pious hope, in no way
% begging the question of its existence. Note that we shall usually
% omit $\circ$ from composite maps.

% \noindent
% {\bf The Lifting Problem}\index{lifting problem} \
% Given a pair of maps $E \stackrel{p}{\rightarrow}B$ and $X \stackrel{f}
% {\rightarrow} B $,
% does there exist a map $f^{\circ} : X \to E$, with
% $pf^{\circ} = f  $?


% That {\em all\/} existence problems about maps are essentially of one
% type or
% the other from these two is seen as follows. Evidently, all existence problems
% are representable by triangular diagrams\index{triangular diagrams} and it
% is easily seen that there are only these six possibilities:
% \begin{center}\begin{picture}(300,70)  %augch2 75
% \put(5,60){\vector(1,0){30}}
% \put(55,60){\vector(1,0){30}}
% \put(135,60){\vector(-1,0){30}}
% \put(185,60){\vector(-1,0){30}}
% \put(235,60){\vector(-1,0){30}}
% \put(285,60){\vector(-1,0){30}}
% \put(0,55){\vector(0,-1){30}}
% \put(50,55){\vector(0,-1){30}}
% \put(100,25){\vector(0,1){30}}
% \put(150,25){\vector(0,1){30}}
% \put(200,55){\vector(0,-1){30}}
% \put(250,55){\vector(0,-1){30}}
% \put(28,33){\small ?}
% \put(78,33){\small ?}
% \put(128,33){\small ?}
% \put(178,33){\small ?}
% \put(228,33){\small ?}
% \put(278,33){\small ?}
% \put(10,3){\bf 1}
% \put(60,3){\bf 2}
% \put(110,3){\bf 3}
% \put(160,3){\bf 4}
% \put(210,3){\bf 5}
% \put(260,3){\bf 6}
% \put(35,55){\vector(-1,-1){30}}
% \put(155,25){\vector(1,1){30}}
% \put(135,55){\vector(-1,-1){30}}
% \put(55,25){\vector(1,1){30}}
% \put(235,55){\vector(-1,-1){30}}
% \put(255,25){\vector(1,1){30}}
% \end{picture}\end{center}



% \begin{figure}
% \begin{picture}(300,220)(0,0)
% \put(-20,-20){\resizebox{20 cm}{!}{\includegraphics{3dpdf}}}
% \put(260,-10){\resizebox{15 cm}{!}{\includegraphics{contpdf}}}
% \put(220,80){$\beta$}
% \put(400,-10){$N$}
% \put(260,170){$\beta$}
% \put(90,15){$N$}
% \end{picture}
% \caption{{\em The log-gamma family of densities with central mean
% $<N> \, = \frac{1}{2}$ as a surface and as a contour plot. }}
% \label{pdf}
% \end{figure}

\newpage

%ch.tex


\chapter{The semantic web}
\begin{center}
{\small\em Where are we; how did we get here; and where are we going?}
\end{center}

\input{chapters/ten/semantic-web}
\input{chapters/ten/new-queries}
\input{chapters/ten/search-for-behavior}

% \section{Existence problems}
% We begin with some metamathematics.
% All problems about the existence of maps can be cast into one of the
% following two forms, which are in a sense mutually dual.

% \noindent
% {\bf The Extension Problem}\index{extension problem} \    %%% NB index entry tag
% Given an inclusion $ A \stackrel{i}{\hookrightarrow} X $, and a map
% $ A \stackrel{f}{\rightarrow} Y $,
% does there exist a map $f^{\dagger}:X\to Y$ such that
% $f^{\dagger}$ agrees with $f$ on $A$?

% Here the appropriate source category for maps should be clear from the
% context and, moreover, commutativity through a
% candidate $f^{\dagger}$ is precisely
% the restriction requirement; that is,
% $$f^{\dagger}   :  f^{\dagger}\circ i = f^{\dagger}|_A = f\,. $$
% If such an $f^{\dagger}$ exists\footnote{${}^{\dagger}$ suggests striving
% for perfection, crusading}, then it is called an {\bf
% extension}\index{extension!of a map|bi} of $f$ and is said to {\bf
% extend}\index{extend|bi} $f$. In any diagrams, the presence of
% a dotted arrow or an arrow carrying a ? indicates a pious hope, in no way
% begging the question of its existence. Note that we shall usually
% omit $\circ$ from composite maps.

% \noindent
% {\bf The Lifting Problem}\index{lifting problem} \
% Given a pair of maps $E \stackrel{p}{\rightarrow}B$ and $X \stackrel{f}
% {\rightarrow} B $,
% does there exist a map $f^{\circ} : X \to E$, with
% $pf^{\circ} = f  $?


% That {\em all\/} existence problems about maps are essentially of one
% type or
% the other from these two is seen as follows. Evidently, all existence problems
% are representable by triangular diagrams\index{triangular diagrams} and it
% is easily seen that there are only these six possibilities:
% \begin{center}\begin{picture}(300,70)  %augch2 75
% \put(5,60){\vector(1,0){30}}
% \put(55,60){\vector(1,0){30}}
% \put(135,60){\vector(-1,0){30}}
% \put(185,60){\vector(-1,0){30}}
% \put(235,60){\vector(-1,0){30}}
% \put(285,60){\vector(-1,0){30}}
% \put(0,55){\vector(0,-1){30}}
% \put(50,55){\vector(0,-1){30}}
% \put(100,25){\vector(0,1){30}}
% \put(150,25){\vector(0,1){30}}
% \put(200,55){\vector(0,-1){30}}
% \put(250,55){\vector(0,-1){30}}
% \put(28,33){\small ?}
% \put(78,33){\small ?}
% \put(128,33){\small ?}
% \put(178,33){\small ?}
% \put(228,33){\small ?}
% \put(278,33){\small ?}
% \put(10,3){\bf 1}
% \put(60,3){\bf 2}
% \put(110,3){\bf 3}
% \put(160,3){\bf 4}
% \put(210,3){\bf 5}
% \put(260,3){\bf 6}
% \put(35,55){\vector(-1,-1){30}}
% \put(155,25){\vector(1,1){30}}
% \put(135,55){\vector(-1,-1){30}}
% \put(55,25){\vector(1,1){30}}
% \put(235,55){\vector(-1,-1){30}}
% \put(255,25){\vector(1,1){30}}
% \end{picture}\end{center}



% \begin{figure}
% \begin{picture}(300,220)(0,0)
% \put(-20,-20){\resizebox{20 cm}{!}{\includegraphics{3dpdf}}}
% \put(260,-10){\resizebox{15 cm}{!}{\includegraphics{contpdf}}}
% \put(220,80){$\beta$}
% \put(400,-10){$N$}
% \put(260,170){$\beta$}
% \put(90,15){$N$}
% \end{picture}
% \caption{{\em The log-gamma family of densities with central mean
% $<N> \, = \frac{1}{2}$ as a surface and as a contour plot. }}
% \label{pdf}
% \end{figure}

\newpage

%ch.tex


\chapter{The semantic web}
\begin{center}
{\small\em Where are we; how did we get here; and where are we going?}
\end{center}

\input{chapters/ten/semantic-web}
\input{chapters/ten/new-queries}
\input{chapters/ten/search-for-behavior}

% \section{Existence problems}
% We begin with some metamathematics.
% All problems about the existence of maps can be cast into one of the
% following two forms, which are in a sense mutually dual.

% \noindent
% {\bf The Extension Problem}\index{extension problem} \    %%% NB index entry tag
% Given an inclusion $ A \stackrel{i}{\hookrightarrow} X $, and a map
% $ A \stackrel{f}{\rightarrow} Y $,
% does there exist a map $f^{\dagger}:X\to Y$ such that
% $f^{\dagger}$ agrees with $f$ on $A$?

% Here the appropriate source category for maps should be clear from the
% context and, moreover, commutativity through a
% candidate $f^{\dagger}$ is precisely
% the restriction requirement; that is,
% $$f^{\dagger}   :  f^{\dagger}\circ i = f^{\dagger}|_A = f\,. $$
% If such an $f^{\dagger}$ exists\footnote{${}^{\dagger}$ suggests striving
% for perfection, crusading}, then it is called an {\bf
% extension}\index{extension!of a map|bi} of $f$ and is said to {\bf
% extend}\index{extend|bi} $f$. In any diagrams, the presence of
% a dotted arrow or an arrow carrying a ? indicates a pious hope, in no way
% begging the question of its existence. Note that we shall usually
% omit $\circ$ from composite maps.

% \noindent
% {\bf The Lifting Problem}\index{lifting problem} \
% Given a pair of maps $E \stackrel{p}{\rightarrow}B$ and $X \stackrel{f}
% {\rightarrow} B $,
% does there exist a map $f^{\circ} : X \to E$, with
% $pf^{\circ} = f  $?


% That {\em all\/} existence problems about maps are essentially of one
% type or
% the other from these two is seen as follows. Evidently, all existence problems
% are representable by triangular diagrams\index{triangular diagrams} and it
% is easily seen that there are only these six possibilities:
% \begin{center}\begin{picture}(300,70)  %augch2 75
% \put(5,60){\vector(1,0){30}}
% \put(55,60){\vector(1,0){30}}
% \put(135,60){\vector(-1,0){30}}
% \put(185,60){\vector(-1,0){30}}
% \put(235,60){\vector(-1,0){30}}
% \put(285,60){\vector(-1,0){30}}
% \put(0,55){\vector(0,-1){30}}
% \put(50,55){\vector(0,-1){30}}
% \put(100,25){\vector(0,1){30}}
% \put(150,25){\vector(0,1){30}}
% \put(200,55){\vector(0,-1){30}}
% \put(250,55){\vector(0,-1){30}}
% \put(28,33){\small ?}
% \put(78,33){\small ?}
% \put(128,33){\small ?}
% \put(178,33){\small ?}
% \put(228,33){\small ?}
% \put(278,33){\small ?}
% \put(10,3){\bf 1}
% \put(60,3){\bf 2}
% \put(110,3){\bf 3}
% \put(160,3){\bf 4}
% \put(210,3){\bf 5}
% \put(260,3){\bf 6}
% \put(35,55){\vector(-1,-1){30}}
% \put(155,25){\vector(1,1){30}}
% \put(135,55){\vector(-1,-1){30}}
% \put(55,25){\vector(1,1){30}}
% \put(235,55){\vector(-1,-1){30}}
% \put(255,25){\vector(1,1){30}}
% \end{picture}\end{center}



% \begin{figure}
% \begin{picture}(300,220)(0,0)
% \put(-20,-20){\resizebox{20 cm}{!}{\includegraphics{3dpdf}}}
% \put(260,-10){\resizebox{15 cm}{!}{\includegraphics{contpdf}}}
% \put(220,80){$\beta$}
% \put(400,-10){$N$}
% \put(260,170){$\beta$}
% \put(90,15){$N$}
% \end{picture}
% \caption{{\em The log-gamma family of densities with central mean
% $<N> \, = \frac{1}{2}$ as a surface and as a contour plot. }}
% \label{pdf}
% \end{figure}

\newpage

%ch.tex


\chapter{The semantic web}
\begin{center}
{\small\em Where are we; how did we get here; and where are we going?}
\end{center}

\input{chapters/ten/semantic-web}
\input{chapters/ten/new-queries}
\input{chapters/ten/search-for-behavior}

% \section{Existence problems}
% We begin with some metamathematics.
% All problems about the existence of maps can be cast into one of the
% following two forms, which are in a sense mutually dual.

% \noindent
% {\bf The Extension Problem}\index{extension problem} \    %%% NB index entry tag
% Given an inclusion $ A \stackrel{i}{\hookrightarrow} X $, and a map
% $ A \stackrel{f}{\rightarrow} Y $,
% does there exist a map $f^{\dagger}:X\to Y$ such that
% $f^{\dagger}$ agrees with $f$ on $A$?

% Here the appropriate source category for maps should be clear from the
% context and, moreover, commutativity through a
% candidate $f^{\dagger}$ is precisely
% the restriction requirement; that is,
% $$f^{\dagger}   :  f^{\dagger}\circ i = f^{\dagger}|_A = f\,. $$
% If such an $f^{\dagger}$ exists\footnote{${}^{\dagger}$ suggests striving
% for perfection, crusading}, then it is called an {\bf
% extension}\index{extension!of a map|bi} of $f$ and is said to {\bf
% extend}\index{extend|bi} $f$. In any diagrams, the presence of
% a dotted arrow or an arrow carrying a ? indicates a pious hope, in no way
% begging the question of its existence. Note that we shall usually
% omit $\circ$ from composite maps.

% \noindent
% {\bf The Lifting Problem}\index{lifting problem} \
% Given a pair of maps $E \stackrel{p}{\rightarrow}B$ and $X \stackrel{f}
% {\rightarrow} B $,
% does there exist a map $f^{\circ} : X \to E$, with
% $pf^{\circ} = f  $?


% That {\em all\/} existence problems about maps are essentially of one
% type or
% the other from these two is seen as follows. Evidently, all existence problems
% are representable by triangular diagrams\index{triangular diagrams} and it
% is easily seen that there are only these six possibilities:
% \begin{center}\begin{picture}(300,70)  %augch2 75
% \put(5,60){\vector(1,0){30}}
% \put(55,60){\vector(1,0){30}}
% \put(135,60){\vector(-1,0){30}}
% \put(185,60){\vector(-1,0){30}}
% \put(235,60){\vector(-1,0){30}}
% \put(285,60){\vector(-1,0){30}}
% \put(0,55){\vector(0,-1){30}}
% \put(50,55){\vector(0,-1){30}}
% \put(100,25){\vector(0,1){30}}
% \put(150,25){\vector(0,1){30}}
% \put(200,55){\vector(0,-1){30}}
% \put(250,55){\vector(0,-1){30}}
% \put(28,33){\small ?}
% \put(78,33){\small ?}
% \put(128,33){\small ?}
% \put(178,33){\small ?}
% \put(228,33){\small ?}
% \put(278,33){\small ?}
% \put(10,3){\bf 1}
% \put(60,3){\bf 2}
% \put(110,3){\bf 3}
% \put(160,3){\bf 4}
% \put(210,3){\bf 5}
% \put(260,3){\bf 6}
% \put(35,55){\vector(-1,-1){30}}
% \put(155,25){\vector(1,1){30}}
% \put(135,55){\vector(-1,-1){30}}
% \put(55,25){\vector(1,1){30}}
% \put(235,55){\vector(-1,-1){30}}
% \put(255,25){\vector(1,1){30}}
% \end{picture}\end{center}



% \begin{figure}
% \begin{picture}(300,220)(0,0)
% \put(-20,-20){\resizebox{20 cm}{!}{\includegraphics{3dpdf}}}
% \put(260,-10){\resizebox{15 cm}{!}{\includegraphics{contpdf}}}
% \put(220,80){$\beta$}
% \put(400,-10){$N$}
% \put(260,170){$\beta$}
% \put(90,15){$N$}
% \end{picture}
% \caption{{\em The log-gamma family of densities with central mean
% $<N> \, = \frac{1}{2}$ as a surface and as a contour plot. }}
% \label{pdf}
% \end{figure}

\newpage

%ch.tex


\chapter{The semantic web}
\begin{center}
{\small\em Where are we; how did we get here; and where are we going?}
\end{center}

\input{chapters/ten/semantic-web}
\input{chapters/ten/new-queries}
\input{chapters/ten/search-for-behavior}

% \section{Existence problems}
% We begin with some metamathematics.
% All problems about the existence of maps can be cast into one of the
% following two forms, which are in a sense mutually dual.

% \noindent
% {\bf The Extension Problem}\index{extension problem} \    %%% NB index entry tag
% Given an inclusion $ A \stackrel{i}{\hookrightarrow} X $, and a map
% $ A \stackrel{f}{\rightarrow} Y $,
% does there exist a map $f^{\dagger}:X\to Y$ such that
% $f^{\dagger}$ agrees with $f$ on $A$?

% Here the appropriate source category for maps should be clear from the
% context and, moreover, commutativity through a
% candidate $f^{\dagger}$ is precisely
% the restriction requirement; that is,
% $$f^{\dagger}   :  f^{\dagger}\circ i = f^{\dagger}|_A = f\,. $$
% If such an $f^{\dagger}$ exists\footnote{${}^{\dagger}$ suggests striving
% for perfection, crusading}, then it is called an {\bf
% extension}\index{extension!of a map|bi} of $f$ and is said to {\bf
% extend}\index{extend|bi} $f$. In any diagrams, the presence of
% a dotted arrow or an arrow carrying a ? indicates a pious hope, in no way
% begging the question of its existence. Note that we shall usually
% omit $\circ$ from composite maps.

% \noindent
% {\bf The Lifting Problem}\index{lifting problem} \
% Given a pair of maps $E \stackrel{p}{\rightarrow}B$ and $X \stackrel{f}
% {\rightarrow} B $,
% does there exist a map $f^{\circ} : X \to E$, with
% $pf^{\circ} = f  $?


% That {\em all\/} existence problems about maps are essentially of one
% type or
% the other from these two is seen as follows. Evidently, all existence problems
% are representable by triangular diagrams\index{triangular diagrams} and it
% is easily seen that there are only these six possibilities:
% \begin{center}\begin{picture}(300,70)  %augch2 75
% \put(5,60){\vector(1,0){30}}
% \put(55,60){\vector(1,0){30}}
% \put(135,60){\vector(-1,0){30}}
% \put(185,60){\vector(-1,0){30}}
% \put(235,60){\vector(-1,0){30}}
% \put(285,60){\vector(-1,0){30}}
% \put(0,55){\vector(0,-1){30}}
% \put(50,55){\vector(0,-1){30}}
% \put(100,25){\vector(0,1){30}}
% \put(150,25){\vector(0,1){30}}
% \put(200,55){\vector(0,-1){30}}
% \put(250,55){\vector(0,-1){30}}
% \put(28,33){\small ?}
% \put(78,33){\small ?}
% \put(128,33){\small ?}
% \put(178,33){\small ?}
% \put(228,33){\small ?}
% \put(278,33){\small ?}
% \put(10,3){\bf 1}
% \put(60,3){\bf 2}
% \put(110,3){\bf 3}
% \put(160,3){\bf 4}
% \put(210,3){\bf 5}
% \put(260,3){\bf 6}
% \put(35,55){\vector(-1,-1){30}}
% \put(155,25){\vector(1,1){30}}
% \put(135,55){\vector(-1,-1){30}}
% \put(55,25){\vector(1,1){30}}
% \put(235,55){\vector(-1,-1){30}}
% \put(255,25){\vector(1,1){30}}
% \end{picture}\end{center}



% \begin{figure}
% \begin{picture}(300,220)(0,0)
% \put(-20,-20){\resizebox{20 cm}{!}{\includegraphics{3dpdf}}}
% \put(260,-10){\resizebox{15 cm}{!}{\includegraphics{contpdf}}}
% \put(220,80){$\beta$}
% \put(400,-10){$N$}
% \put(260,170){$\beta$}
% \put(90,15){$N$}
% \end{picture}
% \caption{{\em The log-gamma family of densities with central mean
% $<N> \, = \frac{1}{2}$ as a surface and as a contour plot. }}
% \label{pdf}
% \end{figure}

\newpage



\documentclass[12pt,leqno]{book}
\usepackage{amsmath,amssymb,amsfonts} % Typical maths resource packages
\usepackage{graphics}                 % Packages to allow inclusion of graphics
\usepackage{color}                    % For creating coloured text and background
\usepackage{hyperref}                 % For creating hyperlinks in cross references
\usepackage{makeidx}                  % For indexing
\usepackage{listings}                 % For code listing
\usepackage{mathpartir}               % For grammars, rules, etc
\usepackage{bcprules}                 % For other kinds of rules

\lstloadlanguages{Scala,Java,Haskell,XML,bash,HTML,SQL}

\parindent 1cm
\parskip 0.2cm
\topmargin 0.2cm
\oddsidemargin 1cm
\evensidemargin 0.5cm
\textwidth 15cm
\textheight 21cm

\include{local/local}

\makeindex


\title{Pro Scala: Monadic Design Patterns for the Web}

\author{L.G. Meredith  \\
{\small\em \copyright \  Draft date \today }}

 \date{ }
\begin{document}
\lstset{language=Haskell}
\maketitle
 \addcontentsline{toc}{chapter}{Contents}
\pagenumbering{roman}
\tableofcontents
\listoffigures
\listoftables
\chapter*{Preface}\normalsize
  \addcontentsline{toc}{chapter}{Preface}
\pagestyle{plain}
% The book root file {\tt bookex.tex} gives a basic example of how to
% use \LaTeX \ for preparation of a book. Note that all
% \LaTeX \ commands begin with a
% backslash.

% Each
% Chapter, Appendix and the Index is made as a {\tt *.tex} file and is
% called in by the {\tt include} command---thus {\tt ch1.tex} is
% the name here of the file containing Chapter~1. The inclusion of any
% particular file can be suppressed by prefixing the line by a
% percent sign.


%  Do not put an {\tt end{document}} command at the end of chapter files;
% just one such command is needed at the end of the book.

% Note the tag used to make an index entry. You may need to consult Lamport's
% book~\cite{lamport} for details of the procedure to make the index input
% file; \LaTeX \ will create a pre-index by listing all the tagged
% items in the file {\tt bookex.idx} then you edit this into
% a {\tt theindex} environment, as {\tt index.tex}.

The book you hold in your hands, Dear Reader, is not at all what you expected...



\pagestyle{headings}
\pagenumbering{arabic}

\include{chapters/one/ch}
\include{chapters/two/ch}
\include{chapters/three/ch}
\include{chapters/four/ch}
\include{chapters/five/ch}
\include{chapters/six/ch}
\include{chapters/seven/ch}
\include{chapters/eight/ch}
\include{chapters/nine/ch}
\include{chapters/ten/ch}

\include{bibliography/monadic}

%\include{index/index}
  \addcontentsline{toc}{chapter}{Index}
\end{document}


%\index{monad!categorical}
\index{monad!haskell}
  \addcontentsline{toc}{chapter}{Index}
\end{document}


%\index{monad!categorical}
\index{monad!haskell}
  \addcontentsline{toc}{chapter}{Index}
\end{document}


%\index{monad!categorical}
\index{monad!haskell}
  \addcontentsline{toc}{chapter}{Index}
\end{document}
